% AI stuff
\newcommand{\domain}{\mathds{D}}
\newcommand{\entities}{\texttt{ent}}
\newcommand{\consistsWith}{\texttt{cw}}
\newcommand{\charFunML}{\charactheristicFunction}
\newcommand{\aCircuit}{\mathcal{C}}

% Computational complexity stuff
\newcommand{\NP}{\textsc{NP}}
\newcommand{\NPhard}{\textsc{NP-hard}}
\newcommand{\NPcomplete}{\textsc{NP-complete}}
\newcommand{\sharpPhard}{\textsc{\#P-hard}}

% Sets and probability stuff
\newcommand{\R}{\mathds{R}}
\newcommand{\perm}{perm}
\newcommand{\aDistribution}{\mathcal{D}}
\newcommand{\expectancy}{\mathds{E}}
% Important names and stuff
\newcommand{\SHAPscore}{\textsc{SHAP-score}}
\newcommand{\aBayesianDistribution}{\mathcal{B}}
\newcommand{\set}[1]{ \{ #1 \} }

% Game Theory
\newcommand{\charactheristicFunction}{\nu}
\newcommand{\players}{\mathcal{I}}
\newcommand{\Shap}{Shap}
\newcommand{\assym}{\Shap^{assym}}

% DAG and Assym Notation
\newcommand{\toOr}{\pi} %topologicalOrder
\newcommand{\rel}{R} %relation
\newcommand{\union}{union}
\newcommand{\equivalenceClass}{[\pi]_{\rel}} %equivalenceClass
\newcommand{\eqClassSize}{eqClassTopos}
\newcommand{\equivalenceClassRep}{\mathsf{Rep}(\equivalenceClass)}
\newcommand{\numEqCl}{\#EC} %equivalenceClass formula
\newcommand{\unrEqCl}{UnrEC} %unrelated equivalenceClass formula
\newcommand{\heuristicASVFormula}{\frac{1}{|\topo(G)|} \sum_{\equivalenceClass \in eqCl(G, x_i)} (\charactheristicFunction(\toOr_{<i} \cup \{x_i\}) - \charactheristicFunction(\toOr_{<i})) \cdot |\equivalenceClass|} %New Formula
\newcommand{\eqClassSizes}{eqClassSizes} %unrelated equivalenceClass sizes formula
\newcommand{\leftPossibleOrders}{\#LO} %Left possible orders

%% Drawing graphs

\newcommand{\drawUnrelatedTreeWithColor}[5]{
    \node[circle, draw=#5] (#1) at (#2, #3) {#4};
    \pgfmathsetmacro{\x}{#2-2}
    \pgfmathsetmacro{\y}{#3-0.3} 
    \draw[#5, wiggly] (#2, \y)
        -- ++(-1,-1.6) 
        -- ++(2,0) 
        -- cycle;
    \node[text=#5] at (\x, \y) {#4 subtree};
}

\newcommand{\drawUnrelatedTree}[4]{
    \drawUnrelatedTreeWithColor{#1}{#2}{#3}{#4}{red}
}

\newcommand{\drawUnrelatedTreeWithTag}[6]{
    \drawUnrelatedTreeWithColor{#1}{#2}{#3}{#4}{#5}

    % Texto de nodos disponibles (más abajo del contorno)
    \pgfmathsetmacro{\ysubtree}{#3 - 0.3}
    \pgfmathsetmacro{\ytext}{\ysubtree - 2.0}
    \node[text=#5] at (#2, \ytext) {#6};
}


% Comments and stuff
%\definecolor{darkred}{rgb}{0.55, 0.0, 0.0}
%\usepackage[dvipsnames,svgnames,x11names]{xcolor}
\usepackage[backgroundcolor=orange, textcolor=black, textsize=tiny]{todonotes}
\newcommand{\santi}[1]{\todo[inline,caption={},color=blue!30]{{\bf Santi:} #1}}
\newcommand{\echu}[1]{\todo[inline,caption={},color=orange!30]{{\bf Echu:} #1}}
\newcommand{\sergio}[1]{\todo[inline,caption={},color=green!30, size=\footnotesize]{{\bf Sergio:} #1}} 
\newcommand{\sidesergio}[1]{\todo[caption={},color=green!30, size=\footnotesize]{{\bf Sergio:} #1}}
\newcommand{\aMechanism}[0]{\mathcal{D}}
\newcommand{\anAlgorithm}[0]{\mathcal{A}}

% Graphs and networks
\newcommand{\topo}{topos}
\newcommand{\numTopo}{\#topos}
\newcommand{\aBayesianNetwork}{N}
\newcommand{\parents}{\textit{Parents}}
\newcommand{\dtrees}{\emph{dtrees}}
\newcommand{\dtree}{\emph{dtree}}
\newcommand{\childNetwork}{\emph{Child}}
\newcommand{\cancerNetwork}{\emph{Cancer}}
\newcommand{\variableNetwork}[1]{\texttt{#1}}
\newcommand{\intersectionNode}{i_{\cap}}
\newcommand{\subtree}{s}


% Theorems %Traducidos 18/3
\newtheorem{example}{Ejemplo}
\newtheorem{theorem}{Teorema}
\theoremstyle{plain} %Nota amsthm es necesario para Theoremstyle
% \newtheorem{proposition}[theorem]{Proposition} Si lo hago de este modo todos compartirían el mismo contador y no quiero que eso pase. 
\newtheorem{proposition}{Proposición} 
\newtheorem{lemma}{Lema}
\newtheorem{corollary}{Corolario}
\newtheorem{observation}{Observación}  
\newtheorem{formula}{Fórmula}  
\newtheorem{sketch}{Boceto} 
\newtheorem{acknowledgements}{Agradecimientos}
%
\newtheorem{fact}{Hecho}
%\theoremstyle{definition}  % Para que no sea en italics
%\newtheorem{definition}[theorem]{Definition}

%Definition 

\theoremstyle{definition} % The style more suitable for definitions, examples, etc.
\newtheorem{definition}{Definición}[section] % This will not share the counter with theorems