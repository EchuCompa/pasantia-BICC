% AI stuff
\newcommand{\domain}{\mathds{D}}
\newcommand{\entities}{\texttt{ent}}
\newcommand{\consistsWith}{\texttt{cw}}
\newcommand{\charFunML}{\charactheristicFunction}
\newcommand{\aCircuit}{\mathcal{C}}

% Computational complexity stuff
\newcommand{\NP}{\textsc{NP}}
\newcommand{\NPhard}{\textsc{NP-hard}}
\newcommand{\NPcomplete}{\textsc{NP-complete}}
\newcommand{\sharpPhard}{\textsc{\#P-hard}}

% Sets and probability stuff
\newcommand{\R}{\mathds{R}}
\newcommand{\perm}{perm}
\newcommand{\aDistribution}{\mathcal{D}}
\newcommand{\expectancy}{\mathds{E}}
% Important names and stuff
\newcommand{\SHAPscore}{\textsc{SHAP-score}}
\newcommand{\aBayesianDistribution}{\mathcal{B}}
\newcommand{\set}[1]{ \{ #1 \} }

% Game Theory
\newcommand{\charactheristicFunction}{\nu}
\newcommand{\players}{\mathcal{I}}
\newcommand{\Shap}{Shap}
\newcommand{\assym}{\Shap^{assym}}

% DAG and Assym Notation
\newcommand{\toOr}{\pi} %topologicalOrder
\newcommand{\rel}{R} %relation
\newcommand{\eqCl}{[\pi]\rel} %equivalenceClass
\newcommand{\numEqCl}{\#EC} %equivalenceClass formula
\newcommand{\unrEqCl}{\#UnrEC} %unrelated equivalenceClass formula
\newcommand{\heuristicASVFormula}{\sum_{\eqCl \in eqCl(G, x_i), \toOr \in \eqCl} \left(\charactheristicFunction(\toOr_{<i} \cup \{x_i\}) - \charactheristicFunction(\toOr{<i})
\right)* \#\eqCl} %New Formula
\newcommand{\eqClassSizes}{eqClassSizes} %unrelated equivalenceClass sizes formula
\newcommand{\leftPossibleOrders}{\#LO} %Left possible orders






% Comments and stuff
%\definecolor{darkred}{rgb}{0.55, 0.0, 0.0}
%\usepackage[dvipsnames,svgnames,x11names]{xcolor}
\usepackage[backgroundcolor=orange, textcolor=black, textsize=tiny]{todonotes}
\newcommand{\santi}[1]{\todo[inline,caption={},color=blue!30]{{\bf Santi:} #1}}
\newcommand{\echu}[1]{\todo[inline,caption={},color=blue!30]{{\bf Echu:} #1}}
\newcommand{\sergio}[1]{\todo[inline,caption={},color=green!30, size=\footnotesize]{{\bf Sergio:} #1}} 
\newcommand{\sidesergio}[1]{\todo[caption={},color=green!30, size=\footnotesize]{{\bf Sergio:} #1}}

% Graphs and networks
\newcommand{\topo}{topo}
\newcommand{\numTopo}{\#topo}
\newcommand{\aBayesianNetwork}{N}
\newcommand{\parents}{\textit{Parents}}

% Theorems
\newtheorem{example}{Example}
\newtheorem{theorem}{Theorem}
\theoremstyle{plain} %Nota amsthm es necesario para Theoremstyle
\newtheorem{proposition}[theorem]{Proposition}
\newtheorem{lemma}[theorem]{Lemma}
\newtheorem{corollary}[theorem]{Corollary}
\newtheorem{observation}[theorem]{Observation}  
\newtheorem{sketch}[theorem]{Sketch} 
\newtheorem{acknowledgements}[theorem]{Acknowledgements}
%
\newtheorem{fact}[theorem]{Fact}
%\theoremstyle{definition}  % Para que no sea en italics
%\newtheorem{definition}[theorem]{Definition}

%Definition 

\theoremstyle{definition} % The style more suitable for definitions, examples, etc.
\newtheorem{definition}{Definition}[section] % This will not share the counter with theorems