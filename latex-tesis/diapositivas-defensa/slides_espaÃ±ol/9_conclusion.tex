\begin{frame}{Conclusiones}
	\dificultyLevel{1}
	\begin{itemize}[<+- | alert@+>]
		\item Se optimiz\'o el c\'alculo de ASV en datos con distribuciones bayesianas y árboles de decisión.
		\item Se demostr\'o la tratabilidad para Naive Bayes.
		\item Se desarroll\'o un algoritmo exacto eficiente para la predicci\'on promedio en árboles de decisión.
		\item Se defini\'o una heur\'istica basada en clases de equivalencia para reducir las evaluaciones.
		\item Se construy\'o un algoritmo de sampleo de órdenes topológicos con performance tratable en grafos con grados acotados.
		%\item En la pr\'actica: la heur\'istica reduce llamadas a $\charactheristicFunction$, aunque requiere optimizaci\'on.
	\end{itemize}
\end{frame}

\begin{frame}{Conclusiones}
	\dificultyLevel{1}
	\begin{itemize}[<+- | alert@+>]
		\item Se implement\'o una versi\'on exacta y otra aproximada para ASV.
		\item Se comprobó empiricamente que las clases de equivalencia proporcionan una mejora significativa. 
		\item El principal aporte es la optimizaci\'on de ASV mediante clases de equivalencia respecto de los órdenes topológicos.
	\end{itemize}
\end{frame}

\begin{frame}{Trabajo Futuro}
	\dificultyLevel{1}
	\begin{itemize}[<+- | alert@+>]
		\item Generalizar algoritmo de clases de equivalencia a \emph{polytrees}.
		%\item Optimizar el algoritmo de conteo de órdenes topológicos.
		\item Implementar nuevas estrategias de sampleo y conteo. % \cite{HUBER2006420, efficientCountingOfToposorts}.
		\item Extender la implementación de ASV para modelos y distribuciones arbitrarios.
		\item Estudiar propiedades de complejidad del sampleo y conteo de órdenes tópologicos.
		\item Explorar algoritmos alternativos para enumerar órdenes tópologicos.
		
	\end{itemize}
\end{frame}

