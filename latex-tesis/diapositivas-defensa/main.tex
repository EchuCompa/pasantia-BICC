%For the multiple rows for the dots in the headline bar
	%\PassOptionsToPackage{subsection=false}{beamerouterthememiniframes} %Para que no aparezca la subsección en el headr
	%\usetheme[compress]{Berlin}
	%\useoutertheme{miniframes}

\documentclass{beamer}
\usepackage[T1]{fontenc}
\usepackage[spanish,activeacute]{babel}
\usepackage{Ritsumeikan} %Paquete de estilo que debería analizar porque modificar el header para que no tenga suficiente espacio para poner las subsecciones en 2 puntos. 
%\usetheme{Madrid}
\PassOptionsToPackage{subsection=false}{beamerouterthememiniframes} 
\useoutertheme{miniframes}
\usepackage{tikz}
\usepackage{dsfont, amsmath, amsthm}
\usetikzlibrary{arrows.meta}
\usepackage{amssymb}
\usepackage{booktabs} % For better table formatting
\usepackage{graphicx} % Required for inserting images
\usepackage{hyperref}
\hypersetup{
    colorlinks=true,
    urlcolor=blue,
    linkcolor=blue,
    citecolor=green
}
%linkColor = citas internas cómo ref
\usepackage{tikz}
\usetikzlibrary{positioning}
\usetikzlibrary{arrows.meta}
\usetikzlibrary{decorations.pathmorphing} %For the wiggly lines and more
\usepackage{algorithm}
\usepackage{algpseudocode}
\usepackage{pgf} % For arithmetic operations in TikZ
\usepackage{comment}
\usepackage{booktabs}
\usepackage{subcaption}
\usepackage{multirow}
\usepackage{url} %For writing paths
\usepackage[normalem]{ulem}
\newcommand{\scc}[1]{{\color{red}#1}}
\usepackage{float}

%Redefiniendo los entornos

\newtheorem{mydef}{Definición}
\newtheorem{mythm}{Teorema}
\setbeamercolor{alerted text}{fg=orange}

%gets rid of bottom navigation bars
\setbeamertemplate{footline}[frame number]{}
%gets rid of bottom navigation symbols
\setbeamertemplate{navigation symbols}{}


% other packages
\usepackage{latexsym,xcolor,multicol,calligra}
\usepackage{pstricks,listings,stackengine}
\usepackage{tcolorbox}
\newtheorem{thm}{Lema}

\usepackage{forest}
\usepackage{caption}
\definecolor{level1color}{RGB}{255,200,200}
\definecolor{level2color}{RGB}{200,255,200}
\definecolor{level3color}{RGB}{200,200,255}


\usepackage[authoryear,round]{natbib}   % carga natbib en modo autor-año para bibliografia
\bibliographystyle{abbrvnat} 

% Definiciones de estilos de nodo y arista
\tikzset{
	nodo/.style={
		shape=circle,
		draw=black,
		line width=1,
		minimum size=7mm
	},
	nodito/.style={
		shape=circle,
		draw=black,
		line width=1,
		minimum size=5mm
	},
	arista/.style={
		line width=1,
		-{Latex[length=3mm]}
	},
	aristita/.style={
		line width=1,
		-{Latex[length=2mm]}
	},
	mySnake/.style={
		decorate, 
		decoration={snake, amplitude=.4mm, segment length=4mm, post length=1mm}
	},
	wiggly/.style={
		decorate, 
		decoration={snake, amplitude=.2mm, segment length=2mm}}
}

\usepackage{lipsum}
% AI stuff
\newcommand{\domain}{\mathds{D}}
\newcommand{\entities}{\texttt{ent}}
\newcommand{\consistsWith}{\texttt{cw}}
\newcommand{\charFunML}{\charactheristicFunction}
\newcommand{\aCircuit}{\mathcal{C}}

% Computational complexity stuff
\newcommand{\NP}{\textsc{NP}}
\newcommand{\NPhard}{\textsc{NP-hard}}
\newcommand{\NPcomplete}{\textsc{NP-complete}}
\newcommand{\sharpPhard}{\textsc{\#P-hard}}

% Sets and probability stuff
\newcommand{\R}{\mathds{R}}
\newcommand{\perm}{perm}
\newcommand{\aDistribution}{\mathcal{D}}
\newcommand{\expectancy}{\mathds{E}}
% Important names and stuff
\newcommand{\SHAPscore}{\textsc{SHAP-score}}
\newcommand{\aBayesianDistribution}{\mathcal{B}}
\newcommand{\set}[1]{ \{ #1 \} }

% Game Theory
\newcommand{\charactheristicFunction}{\nu}
\newcommand{\players}{\mathcal{I}}
\newcommand{\Shap}{Shap}
\newcommand{\assym}{\Shap^{assym}}

% DAG and Assym Notation
\newcommand{\toOr}{\pi} %topologicalOrder
\newcommand{\rel}{R} %relation
\newcommand{\eqCl}{[\pi]\rel} %equivalenceClass
\newcommand{\numEqCl}{\#EC} %equivalenceClass formula
\newcommand{\unrEqCl}{\#UnrEC} %unrelated equivalenceClass formula
\newcommand{\heuristicASVFormula}{\sum_{\eqCl \in eqCl(G, x_i), \toOr \in \eqCl} \left(\charactheristicFunction(\toOr_{<i} \cup \{x_i\}) - \charactheristicFunction(\toOr{<i})
\right)* \#\eqCl} %New Formula
\newcommand{\eqClassSizes}{eqClassSizes} %unrelated equivalenceClass sizes formula
\newcommand{\leftPossibleOrders}{\#LO} %Left possible orders






% Comments and stuff
%\definecolor{darkred}{rgb}{0.55, 0.0, 0.0}
%\usepackage[dvipsnames,svgnames,x11names]{xcolor}
\usepackage[backgroundcolor=orange, textcolor=black, textsize=tiny]{todonotes}
\newcommand{\santi}[1]{\todo[inline,caption={},color=blue!30]{{\bf Santi:} #1}}
\newcommand{\echu}[1]{\todo[inline,caption={},color=blue!30]{{\bf Echu:} #1}}
\newcommand{\sergio}[1]{\todo[inline,caption={},color=green!30, size=\footnotesize]{{\bf Sergio:} #1}} 
\newcommand{\sidesergio}[1]{\todo[caption={},color=green!30, size=\footnotesize]{{\bf Sergio:} #1}}

% Graphs and networks
\newcommand{\topo}{topo}
\newcommand{\numTopo}{\#topo}
\newcommand{\aBayesianNetwork}{N}
\newcommand{\parents}{\textit{Parents}}

% Theorems
\newtheorem{example}{Example}
\newtheorem{theorem}{Theorem}
\theoremstyle{plain} %Nota amsthm es necesario para Theoremstyle
\newtheorem{proposition}[theorem]{Proposition}
\newtheorem{lemma}[theorem]{Lemma}
\newtheorem{corollary}[theorem]{Corollary}
\newtheorem{observation}[theorem]{Observation}  
\newtheorem{sketch}[theorem]{Sketch} 
\newtheorem{acknowledgements}[theorem]{Acknowledgements}
%
\newtheorem{fact}[theorem]{Fact}
%\theoremstyle{definition}  % Para que no sea en italics
%\newtheorem{definition}[theorem]{Definition}

%Definition 

\theoremstyle{definition} % The style more suitable for definitions, examples, etc.
\newtheorem{definition}{Definition}[section] % This will not share the counter with theorems



\title{Optimización de ASV para árboles de decisión}

\institute[UBA]{
\begin{columns}
    \column{0.45\textwidth}
    \centering
    Director:\\
    Santiago Cifuentes
    \column{0.45\textwidth}
    \centering
    Co-director:\\
    Sergio Abriola
\end{columns}
\vspace{1em}
Departamento de Computación\\
Facultad de Ciencias Exactas y Naturales\\
Universidad de Buenos Aires
}

\begin{document}

\begin{frame}
    \titlepage
    \vspace*{-0.6cm}
    \begin{figure}[htpb]
        \begin{center}
            \includegraphics[keepaspectratio, scale=0.4]{pic/ubalogo.png}
        \end{center}
    \end{figure}
\end{frame}

\begin{frame}    
\tableofcontents[sectionstyle=show,
subsectionstyle=show/shaded/hide,
subsubsectionstyle=show/shaded/hide]
\end{frame}

\begin{frame}{Dificultad tesis}
	Vamos a utilizar pelotas de voley para medir la dificultad de las diapositivas. 
	
	  \begin{itemize}
	  	\item 1 pelota de voley: Family friendly
	  	\item 2 pelotas de voley: Prestando atención se llega
	  	\item 3 pelotas de voley: Con grafos y ganas alcanza
	  	\item $<$4 pelotas  de voley: Esta es para los jurados
	  	
	  	\end{itemize}
	  
	  \begin{tikzpicture}[remember picture,overlay]
		\foreach \i in {1,...,1} {
			\node[anchor=north east, yshift=-90, xshift={-20px-1.5em*\i}] at (current page.north east) {\includegraphics[width=2em]{pic/voleyball.png}};
		}
	\end{tikzpicture}%
	
	  \begin{tikzpicture}[remember picture,overlay]
	\foreach \i in {1,...,2} {
		\node[anchor=north east, yshift=-110, xshift={-20px-1.5em*\i}] at (current page.north east) {\includegraphics[width=2em]{pic/voleyball.png}};
	}
	\end{tikzpicture}%
	
	  \begin{tikzpicture}[remember picture,overlay]
	\foreach \i in {1,...,3} {
		\node[anchor=north east, yshift=-130, xshift={-20px-1.5em*\i}] at (current page.north east) {\includegraphics[width=2em]{pic/voleyball.png}};
	}
	\end{tikzpicture}%
	
	  \begin{tikzpicture}[remember picture,overlay]
	\foreach \i in {1,...,4} {
		\node[anchor=north east, yshift=-150, xshift={-20px-1.5em*\i}] at (current page.north east) {\includegraphics[width=2em]{pic/voleyball.png}};
	}
	\end{tikzpicture}%
	
	
	

\end{frame}

\section{Introducción}

\begin{comment}
    Introducción
        XAI y las distintas métricas que existen
            Feature attribution, counterfactual, rule based, sufficient reason, etc
        Algoritmos de feature attribution
        Código fuente 
\end{comment}

Las capacidades de los distintos modelos de inteligencia artificial (IA) han ido creciendo exponencialmente en los últimos años. Tareas que creíamos imposibles de realizar a través de estas técnicas, como la resolución de problemas matemáticos complejos, hoy en día pueden ser resueltas por los LLMs \cite{frontierMath}. No sólo aumentó la complejidad de las tareas que pueden resolver, sino también la complejidad y el tamaño de sus arquitecturas. Los modelos de hace unos años tenían 1e8 parámetros, mientras que hoy en día ya llegan a 1e12. \cite{EpochLargeScaleModels2024}. El problema que esto genera es que cada vez es más difícil entender cómo funcionan y cómo obtienen las respuestas que nos dan. Esto es algo sumamente importante, ya que si uno está utilizando estas técnicas para diagnósticos médicos u otras situaciones igual de delicadas, es fundamental  entender como los modelos alcanzan sus conclusiones . Ahí es donde entra en juego el área de XAI, Explainable Artificial Intelligence.

El objetivo principal de esta área de investigación consiste en encontrar una explicación para las decisiones o predicciones de los distintos modelos de IA, para poder entender el razonamiento detrás de las mismas. Dentro de esta área hay dos ramas principales, la explicabilidad y la interpretabilidad. La primera se centra en obtener estas explicaciones, y es el área en la cual nos centramos en este trabajo. La segunda en entender a los modelos y su representación interna \cite{interpretabilityPaper}. Estas explicaciones no son útiles únicamente en tanto permiten que los usuarios entiendan el porqué de la respuesta, sino también debido a que ayudan a detectar errores o sesgos indeseados que se hayan generado durante el entrenamiento. 

Dentro del área de explicabilidad hay varios tipos de métodos \cite{interpretabilityBook}, los cuales pueden agruparse de acuerdo a distintos ejes:
\begin{itemize}
    \item \textit{Modelo}
        \begin{itemize}
            \item Agnóstico: Son los métodos que se pueden aplicar a todo tipo de modelos. Por ejemplo en SHAP, el método que describiremos más adelante, su valor no depende de la implementación interna del modelo. %\santi{A qué te referís acá?} 
            \item Específico: Son los métodos que tienen una implementación que está acoplada al modelo que buscan explicar, como en el caso de \textit{Tree-Shap} , una versión de SHAP desarrollada para árboles de decisión, que en su implementación utiliza los caminos del árbol para calcular su valor, aprovechando propiedades de la estructura del modelo.
            %\santi{Acá creo que sería mejor usar como ejemplo que en el caso de árboles podés tomar el camino como justificación. Eso es claramente único al modelo, mientras que Tree-shap es un concepto agnóstico al modelo pero que se calcula de forma piola en árboles.} 
        \end{itemize}
    \item \textit{Alcance}
        \begin{itemize}
            \item Explicación global: Busca descubrir cualidades o comportamientos que el modelo tenga en todas sus predicciones. Por ejemplo, para entender por qué un modelo le otorga un préstamo a un individuo, se observa que el nivel de ingresos siempre es un feature relevante. %.\santi{Acá siento que queda medio vago lo de explicar en su totalidad. Una alternativa es "busca descubrir cualidades o comportamientos que el modelo tenga en todas sus predicciones, como por ejemplo si el modelo mira o no la cantidad de metros cuadrados de una casa, pensando en los datasets de Housing".}
            \item Explicación local: Busca descubrir cualidades o comportamiento que el modelo tenga para un conjunto reducido de instancias del dataset. Por ejemplo, para entender por qué un modelo le otorga un préstamo a un individuo, podemos encontrar que para las personas casadas, la cantidad de hijos es un feature que toma más relevancia.  %\santi{Idem anterior, ser mas precisos y ejemplificar.}
        \end{itemize}
    \item \textit{Tipo de explicación:} Hay varias categorías del tipo de explicación que puede tener un método, por ejemplo \textit{feature importance}, el cual asigna valores a cada feature. También hay \textit{interpretaciones visuales} como saliency maps o correlations plots. Incluso hay métodos que generan un modelo más simple a partir del modelo original, como \textit{Lime}. 

    Cabe destacar que no existe un consenso absoluto sobre qué constituye una "explicación" en el contexto de la inteligencia artificial. Muchas de las técnicas mencionadas —como feature attribution o interpretaciones visuales— son aproximaciones pragmáticas a un problema mucho más amplio y complejo, vinculado a cuestiones filosóficas sobre comprensión, causalidad e interpretación humana \cite{MILLER20191, lipton2017mythosmodelinterpretability}. En este sentido, lo que hoy se considera una explicación en XAI responde más a criterios de utilidad y simplicidad interpretativa que a una definición formal y universalmente aceptada.

    %\santi{Ojo que está incompleto. Btw, en esta parte podrías agregar un párrafo cuestionando qué es una explicación, y podemos buscar citas a discusiones más filosóficas. Onda, que quede claro que feature attribution y etc son una aproximación a una ambición filosófica mucho mayor.}\sergio{agreed}
\end{itemize}

En esta tesis nos centramos en \textit{SHAP} \cite{shapOriginalPaper}, el cual es un método que, en principio, provee explicaciones agnósticas al modelo, con un alcance local y que es del tipo de feature attribution. Esto significa que para una predicción individual que realiza el modelo, \textit{SHAP} va a devolver un valor asociado a cada feature de la instancia. Decimos que es agnóstico, puesto que su valor depende simplemente del output del modelo, por lo que no esta acoplado a su representación interna. Aún así, \textit{SHAP} actualmente es un framework que engloba varios tipos de métodos, ya que hay distintas aproximaciones según cuál sea tu tipo de modelo. Por ejemplo, existen métodos como DeepExplainer, TreeExplainer y LinearExplainer dentro del  \textcolor{blue}{\href{https://shap.readthedocs.io/en/latest/index.html}{framework}}  SHAP, los cuales se utilizan para modelos específicos, pero también hay otros métodos como KernelExplainer, que son agnósticos al modelo.

%\santi{Pero no es este el motivo por el cual decimos que es agnóstico, no? Decimos que es agnóstico porque su valor depende únicamente del mappeo de entradas a salidas. O sea, es una propiedad del mappeo subyacente al modelo, ignorando la arquitectura específica. Entiendo que la idea original es esa, y estas versiones que se nombras después son aproximaciones imperfectas (pq muchas veces nisiquiera coinciden con los shapley o SHAP) para modelos particulares.}

Uno de los problemas principales de este framework es lo costoso que es calcularlo, y en particular, hay estudios que analizan esta dificultad desde el punto de vista de la complejidad computacional. Por ejemplo, se demostró que calcular SHAP para un clasificador trivial es NP-completo, o intratable, para datos con distribuciones más o igual de complejas que una Naive Bayes \cite{arenas2021tractability}.
A raíz de esto, nos resultó interesante analizar si existía alguna variación de SHAP, la cual pueda ser calculada en tiempo polinomial para una distribución de red bayesiana. Además, actualmente hay trabajos similares a este, que buscan analizar la complejidad de SHAP según la variante, el modelo y la distribución \cite{marzouk2025computationaltractabilitymanyshapley} %\santi{Hace poco salió este paper, habría que agregarlo en algún lado -> On the Computational Tractability of the (Many) Shapley Values} 

SHAP está basado en los Shapley values \cite{shapley1953value}, un valor de teoría de juegos.  Una variación de los mismos son los assymetric shapley values (ASV), los cuales pueden introducir el factor de causalidad al cálculo de los mismos. De esta manera, se define ASV \cite{frye2019asymmetric}, un framework similar a SHAP, el cual además tiene en cuenta el grafo causal de los datos. El objetivo es esta tesis es lograr calcular ASV en tiempo polinomial de forma exacta y aproximada para datos con distribuciones de redes bayesianas.

\subsection{Explicaciones basadas en causalidad: Un ejemplo comparativo entre SHAP y ASV} \label{asvCaseExample}
%\echu{¿Juega el ejemplo así?}

En esta sección presentamos un ejemplo tomado de la sección 4.1 del artículo original de ASV \cite{frye2019asymmetric}, con el objetivo de ilustrar como la incorporación de información causal puede enriquecer las explicaciones de un modelo. Se utiliza como caso de estudio datos del conjunto Census Income de la UCI \cite{dua2017uci}, en el cual se entrena un modelo (en este caso una red neuronal) para predecir si el ingreso de un individuo supera los \$50\,000. Dado que algunas de las variables demográficas presentes en este conjunto (por ejemplo, la edad) son claramente causas de otras (por ejemplo, el nivel educativo), es especialmente pertinente considerar información causal al interpretar la predicción del modelo. El conjunto $A$ son las variables definidas como los ancestros causales, con $A= \set{age, sex, native \ country, race}$ y el resto de variables son descendientes de estas. Luego en base a estas variables tenemos 

Para este experimento se calculan dos variaciones de SHAP, una \emph{off-manifold} y otra \emph{on-manifold}, la diferencia es la función de probabilidad que ambas usan. En el enfoque \emph{on-manifold} se respetan las restricciones que hay entre las distintas variables, por ejemplo a la hora de calcular SHAP se va a descartar cualquier instancia tal que $age=3 \land occupation = teacher$, ya que eso no es posible. Pero en el enfoque \emph{off-manifold} no va a haber ninguna restricción a la hora de combinar los distintos valores de cada feature. Estas restricciones se obtienen a través de las correlaciones entre las distintas variables del dataset. Además, vamos a calcular los valores globales para SHAP y ASV, esto se realiza haciendo el promedio entre los valores de SHAP y ASV para cada una de las instancias. 
%\santi{Este párrafo queda raro. El lector todavía no sabe que hay una distribución involucrada. Por otro lado, hace unos párrafos dijiste que Shap era local, pero ahora estás asignando scores globales.}

\begin{figure}
    \centering
    \includegraphics[width=0.5\linewidth]{img/asvPaperPlotExample.png}
    \caption{Valores globales de los Shapley Values y ASV para el modelo entrenado para el Census Income dataset.}
    \label{fig:asvPaperPlotExample}
\end{figure}

La Figura \ref{fig:asvPaperPlotExample} muestra como la variable \emph{género} presenta un Shapley value relativamente pequeño en el análisis off-manifold. En cambio, al incorporar el conocimiento causal mediante ASV, se observa que dicha variable recibe un valor significativamente mayor. Este hallazgo evidencia que, a pesar de que el género tenga una influencia moderada en el modelo cuando se considera de forma aislada, su papel en la explicación se amplifica una vez que se tiene en cuenta su relación causal con otras variables (como el estado civil o la relación actual). De esta forma, los ASVs proporcionan una medida más precisa y contrastada de la contribución de cada variable, permitiendo detectar aspectos de discriminación o sesgo que de otro modo pasarían desapercibidos.

Este ejemplo evidencia cuándo es útil incorporar información causal en la explicación de modelos: en situaciones donde se conoce una relación de causa y efecto entre las variables, el uso de ASVs no solo mejora la interpretabilidad de la explicación, sino que también aporta una perspectiva más fiel al proceso generador de los datos, contribuyendo así a la construcción de modelos más transparentes y confiables.

%\sergio{En general me parece que hay que poner más motivación y ejemplos en esta sección. Estos también pueden servir para ser referenciados y hacer más concretos los conceptos que se introducen después, como ser en la sección 2.}

%\santi{Agreed. Haría un ejemplito con una predicción, y un típico gráfico de esos que muestran los SHAP scores. Idealmente es un ejemplo donde por no tener cuenta correlaciones se asignan mal los scores. Dps cuando introducis ASV mostras que usando la red causal mejora el ranking (i.e. refleja mejor la importancia de los features)}


\subsection{Código fuente}

Todos los algoritmos presentados en este trabajo fueron implementados. El código fuente puede ser encontrado online en el siguiente repositorio: 

\begin{table}[H]
\centering
\begin{tabular}{ll}
\toprule
\textbf{Repository} & \textbf{Repository} \\
\midrule
Source code & \url{https://github.com/EchuCompa/pasantia-BICC} \\
\end{tabular}
\end{table}

\subsection{Introducción a Shapley y ASV}

\begin{comment}
    SHAP
    Complejidad de SHAP
        Resultados previos
        ASV + Grafo Causal y Toposorts
        Nuestro objetivo era calcular ASV en tiempo polinomial en Árboles de Decisión
\end{comment}

\subsection{SHAP}

%\santi{Hay que comentar acá o arriba que vamos a trabajar con clasificadores binarios y con features binarios, justificando esta decisión. Podemos decir que es por simplicidad, o bien que todo lo que hacemos generaliza. Este párrafo es super importante porque la restricción es fuerte.}

Sea $X$ un conjunto finito de features. Una entidad $e$ sobre $X$ es una función $e: X \to \{0,1\}$, tal que para una feature $x$, $e(x)$ indica el valor que la entidad $e$ toma en $x$. Utilizamos clasificadores binarios y features binarios para esta definición, ya que esta restricción nos permite simplificar la notación y las ideas presentadas, sin perder generalidad en los resultados. En particular, las técnicas y algoritmos que desarrollamos en este trabajo pueden extenderse naturalmente al caso de clasificadores multiclase. El dominio de las instancias que va a tomar nuestro clasificador lo denotamos como $\entities(X)$ \footnote{Podríamos considerar un codominio no binario pero finito $\domain_x$ para cada $x \in X$ y adaptar todas las definiciones}, el cual es el conjunto de todas las posibles $2^{|X|}$ entidades. El espacio de probabilidad para el conjunto \(\entities(X)\) va a estar dado por \(\Pr\). Así es como podemos definir a un clasificador binario $M: \entities(X) \to \{0,1\}$ sobre entidades \footnote{Aquí también podríamos considerar modelos con un codominio finito} como una función, la cual dada una entidad $e$, $M(e)$ indica la clase asignada por el clasificador a $e$. %\santi{Este párrafo está desordenado.} Rta: Ahí lo reordene un poco
%\santi{Comentar algo de estas definiciones. No digo literal una figura con un modelo, pero un chamuyo justificando las formalizaciones estilo: dada una feature $x$, el valor $e(x)$ indica el valor que la entidad $e$ toma en $x$. Dado un modelo, $M(e)$ indica la clase a la cual el mismo clasifica a la entidad $e$.}

Un \textit{feature attribution score} para un modelo $M$ y una entidad $e$ es una función $\phi : X \to \R$, tal que $\phi(x)$ indica el \textit{puntaje} o \textit{relevancia} del feature $x$ con respecto a la predicción $M(e)$. Uno de los puntajes de feature attribution más destacados es él \SHAPscore{} \cite{shapOriginalPaper}, que se basa en los Shapley values \cite{shapley1953value} de la teoría de juegos cooperativos. En ese contexto, los Shapley values representan el esquema único de distribución de la ganancia obtenida por una coalición de jugadores que satisface ciertas propiedades deseables. Una interpretación de los mismos es una función que nos dice cuánto aporta cada jugador al valor total que obtiene la coalición. %\sidesergio{En caso que no aparezca ya en la Intro, motivar un poco más esto, o dar algún ejemplo}

Más formalmente, sea $\players$ un conjunto finito de jugadores, y definimos una \textit{función característica} para $\players$ como una función $\charactheristicFunction : \mathcal{P}(\players) \to \R$, que asigna un valor a cada posible \textit{coalición} de jugadores (es decir, subconjuntos de los jugadores). Por ejemplo, si los jugadores son features, se podría tomar $v(S)$ como la predicción promedio del modelo cuando los features de $S$ se dejan fijos con ciertos valores. Esta valuación daría un valor mayor en la medida que los valores fijos para las features de $S$ estén más correlacionados con que el modelo acepte. Los Shapley values $\{\phi_i\}_{i \in \players}$ son las únicas funciones que toman como entrada funciones características y devuelven valores reales que satisfacen las siguientes propiedades:

%\santi{Agregaría intuición de qué es esto.}

\begin{itemize}
    \item \textbf{Eficiencia}: toda la ganancia es distribuida.
    \begin{align*}
        \sum_{i \in \players} \phi_i(\charactheristicFunction) = \charactheristicFunction(\players)
    \end{align*}

    \item \textbf{Simetría}: cualquier par de jugadores $i,j \in \players$ que contribuyan igual reciben la misma recompensa.
    \begin{align*}
        \forall i,j \in \players : \left( \bigwedge_{S \subseteq \players \setminus \{i,j\}} \charactheristicFunction(S \cup \{i\}) = \charactheristicFunction(S \cup \{j\}) \right) \implies \phi_i(\charactheristicFunction) = \phi_j(\charactheristicFunction)
    \end{align*}

    \item \textbf{Linealidad}: si dos juegos se combinan, entonces la solución a ese nuevo juego es la suma de las soluciones de los originales. Si un juego es multiplicado por un escalar, entonces la solución también se multiplica por él. %\sidesergio{Unificar acá: siempre hablar de juegos o de funciones características. Posiblemente mejor hablar de juegos porque el nombre función característica es confuso}
    \begin{align*}
        \forall a \in \R : 
        \phi_i(a \charactheristicFunction_1 + \charactheristicFunction_2) =  a \phi_i(\charactheristicFunction_1) + \phi_i(\charactheristicFunction_2)
    \end{align*}

    \item \textbf{Jugador nulo}: si un jugador no contribuye en ninguna coalición, entonces no recibe recompensa.
    \begin{align*}
        \forall i \in \players : \left( \bigwedge_{S \subseteq \players \setminus \{i\}} \charactheristicFunction(S) = \charactheristicFunction(S \cup \{i\}) \right) \implies \phi_i(\charactheristicFunction) = 0
    \end{align*}
\end{itemize}

Además, existe una forma cerrada para estas funciones. Dado un conjunto finito $A$, sea $perm(A)$ el conjunto de todas sus permutaciones\footnote{Formalmente, $perm(A) = \{(a_1, a_2, \dots, a_n) \mid \{a_1, a_2, \dots, a_n\} = A\}$, es decir, el conjunto de todas las secuencias que se pueden formar reordenando los elementos de $A$.}, y dada $\pi \in \perm(A)$ denotamos como $\pi_{<a}$ al conjunto $\{a' \in A : \pi(a') < \pi(a)\}$. Entonces:

\begin{align}\label{eq:shapley_values_by_perm}
    \phi_i(\charactheristicFunction) = \frac{1}{|\players|!} \sum_{\pi \in \perm(\players)} \left[\charactheristicFunction(\pi_{<i} \cup {i}) - \charactheristicFunction(\pi_{<i})\right]
\end{align}

Intuitivamente, esta función considera todos los órdenes posibles en que los jugadores llegan al juego y utiliza la contribución que $i$ proporciona al llegar. Se puede demostrar que:

\begin{align*}
    \phi_i(\charactheristicFunction) = \sum_{S \subseteq \players \setminus \{i\}} c_{|S|}\left[\charactheristicFunction(S \cup {i}) - \charactheristicFunction(S)\right]
\end{align*}

donde $c_m = \frac{m! (|\players|-m-1!)}{|\players|!}$.\\

La analogía con el aprendizaje automático surge al entender el conjunto de \(n\) features \(X\) como jugadores, y la función característica \(\charactheristicFunction\) como la predicción promedio al considerar un subconjunto de estos features fijados. En el ejemplo que vimos en la subsección \ref{asvCaseExample}, $X$ serían los features del dataset (age, sex, etc.) y $\charactheristicFunction$ sería la predicción promedio que utiliza a la  red neuronal, que evalúa si el ingreso de una persona es superior a \$50\,000 para realizar la predicción. Dados \(M\) y \(e\), definimos el conjunto de entidades consistentes con (\textit{consistent with}) \(e\) teniendo en cuenta el subconjunto de features \(S \subseteq X\) como \(\consistsWith(e, S) = \{e' \in \entities(X) : e'(s) = e(s) \text{ para } s \in S\}\). Definimos a la probabilidad condicionada como: %\santi{Cuando definís $ent(X)$ estaría bueno decir que asumimos dada una distribución $\Pr[]$ sobre el conjunto.}
\[
\Pr\bigl[e' \mid \consistsWith(e,S)\bigr] =
\begin{cases}
\displaystyle \frac{\Pr[e']}{\sum\limits_{e'' \in \consistsWith(e,S)} \Pr[e'']} & \text{si } e' \in \consistsWith(e,S), \\
0 & \text{en caso contrario}.
\end{cases}
\]

%\santi{Ojo que la probabilidad debería ser 0 si $e'$ no está en las consistentes. Te queda una función partida. Sino, te queda mal definida esta probabilidad (no suman 1 las condicionadas).}

%Echu: Esta fórmula surge de P(A | B ) = P (A ∩ B) / P (B). En este caso B = cw(e,s) y A ∩ B = A = e', con e' in B. 
De este modo se define la función característica como:
\begin{align} \label{formula:characteristicFunctionDefinition}
    \charFunML_{M,e,\Pr}(S)
        = \sum_{e'\in\consistsWith(e,S)}
    \Pr\bigl[e'\mid\consistsWith(e,S)\bigr]\;M(e').    
\end{align}


%\santi{Lo dijiste recién esto último}.
Por conveniencia, para un modelo \(M\), una entidad \(e\) y una distribución \(\Pr\), denotaremos los Shapley values cómo:

\[
\Shap_{M,e,\Pr}(x_i) = \sum_{S \subseteq X \setminus \{x_i\}} c_{|S|} \left[ \charFunML_{M,e,\Pr}(S \cup \{x_i\}) - \charFunML_{M,e,\Pr}(S) \right]
\]


%\santi{Agregar la definición de probabilidad condicionada por consistentes}

Nótese que los axiomas que estos valores satisfacen no tienen un significado claro en el contexto de la inteligencia artificial, ya que dependen de la definición de \(\charFunML_{M,e,\Pr}\) \cite{fryer2021shapley}. Además, para algunas nociones simples y robustas de \textit{feature attribution} basadas en \textit{explicaciones abductivas} \cite{marques2023logic}, los Shapley values no logran asignar un puntaje de 0 a features irrelevantes \cite{huang2023inadequacy}.

\subsection{Complejidad de las explicaciones basadas en Shapley values}

\subsubsection{Resultados conocidos sobre los Shapley Values}

Calcular estos valores en tiempo polinomial con respecto al tamaño del modelo es un desafío: por ejemplo, la sumatoria externa itera sobre un conjunto de tamaño exponencial en el número de features \(n\). Sin embargo, para algunas familias específicas de modelos y distribuciones, es posible desarrollar algoritmos eficientes.

%\footnote{Ojo que estás yendo y veniendo en la forma que escribís. A veces usás impersonal (observar que) y a veces te dirigís al lector (observe). Ídem si vas a hablar de "nosotros hicimos" o "Se hizo... Escribite en un itemize tus criterios de redacción y notaciones, asi podes tenerlos de referencia y no variar tanto"}\sergio{Estoy de acuerdo con este footnote}

El primer resultado de este tipo provino de \cite{lundberg2020local}, donde los autores proporcionan un algoritmo en tiempo polinomial para calcular los Shapley values en árboles de decisión bajo la distribución \textit{producto} o \textit{completamente factorizada}. Tal distribución surgiría naturalmente bajo el supuesto poco realista de \textit{independencia de features}. En dicho escenario tendríamos, para cada \(x \in X\), un valor \(p_x\) que indica la probabilidad de que el feature \(x\) tenga valor 1 en una entidad aleatoria. Así, se sigue que:

\[
\Pr[e' | \consistsWith(e, S)] = \prod_{\substack{x \in X \setminus S \\ e'(x) = 1}} p_x \prod_{\substack{x \in X \setminus S \\ e'(x) = 0}} (1-p_x) 
\]

Estos resultados se extendieron en \cite{arenas2021tractability}, donde se demostró que también es posible calcular los Shapley values para distribuciones producto cuando el modelo está condicionado a ser un circuito \textit{determinístico} y \textit{descomponible}. Además, se mostró que eliminar cualquiera de estas condiciones hace que el problema sea \(\sharpPhard\), y en un artículo posterior también obtuvieron resultados de no-aproximabilidad \cite{arenas2023complexity}.

Un resultado más general fue demostrado simultáneamente en \cite{van2022tractability}: es posible calcular los Shapley values para una familia de modelos \(\mathcal{F}\) bajo la distribución producto si y solo si es posible calcular la predicción promedio para ese modelo dado cualquier conjunto de probabilidades \(\{p_x\}_{x \in X}\) en tiempo polinomial. A través de este lema, deducen inmediatamente la factibilidad de calcular los Shapley values para modelos de regresión lineal, árboles de decisión, funciones booleanas d-DNNF y circuitos CNF de anchura acotada. Luego, también demostraron la intratabilidad de este problema para modelos más expresivos como modelos de regresión logística, redes neuronales con funciones de activación sigmoide y funciones booleanas generales en CNF.

En \cite{lundberg2020local} se afirma que es posible calcular los Shapley values para árboles de decisión bajo la \textit{distribución empírica}, que es la dada por los datos de entrenamiento. Más formalmente, dado un multiconjunto de muestras \(D \subseteq \entities(X)\) de tamaño \(m\), la distribución empírica inducida por \(D\) se define como:

\[
\Pr[e'] = \frac{D(e')}{m} 
\]

donde \(D(e')\) indica el número de copias de \(e'\) que contiene \(D\). Observar que la probabilidad de una entidad no vista es 0.

Sin embargo los autores no proporcionan una demostración que respalde la corrección del algoritmo y, además, en \cite{van2022tractability} se demuestra que, para este tipo de distribución, el problema de calcular los Shapley values es \(\sharpPhard\) incluso para modelos extremadamente simples\footnote{Más precisamente, la afirmación de dificultad se aplica a cualquier familia de modelos que contenga funciones dependientes de solo uno de los features de entrada}, y en particular, para árboles de decisión.

%\sidesergio{shots fired. Quizás empezaría la oración con un 'Sin embargo', o suavizaría un poco}
\begin{comment}
    Sacó esto de acá porque literal menciono lo mismo acá: "En \cite{van2022tractability}, se demostró que...", además era NP-Hard, no #P-Hard
    
    Por último, en \cite{van2022tractability} también se demuestra que el problema es \(\sharpPhard\) al considerar el modelo trivial \(f(x_1,\ldots,x_n) = x_1\) y una distribución Naive Bayes. Una distribución Naive Bayes asume que los features son independientes entre sí, dado el valor de la variable objetivo. \sergio{trataría de explicar un poco más o dar un ejemplito}\santi{Aparte, creo que la definición en ese contexto no es esta. Ellos tienen una red bayesiana que modela las correlaciones entre los features. Por otro lado completamente distinto está el valor del modelo con una cierta entrada. El punto es que una red naive bayes es básicamente dos distribuciones independientes, moduladas por una única feature de la cual depende qué indpendiente se usa.}
\end{comment}


\begin{comment}
%Creo que esto no hace falta porque después lo mencionó más adelante. 
Matemáticamente, esto se representa como 
\[
P(X_1, X_2, \ldots, X_n | Y) = \prod_{i=1}^n P(X_i | Y)
\]    
\end{comment}

%\santi{Agregaría el resultado que encontramos donde calculan shapley values para distribuciones markovianas}
%Buscar SHAP MArkovian Distributions en google

\subsubsection{Asymmetric Shapley Values (ASV)}

La definición de los Shapley values en la Ecuación~\ref{eq:shapley_values_by_perm} asigna el mismo peso a todas las posibles permutaciones. En general, podríamos considerar una función de peso \(w:\perm(\players) \to R\) y definir
%\santi{Una pavada, pero estaría bueno que los links se vean en azul, a mi me parece mejor qsy. Debe ser algo que hay que configurar en hyperref}\sergio{apoyo. Puse azul a los links internos y verde a las citas. Se puede cambiar} 

\begin{align}\label{eq:assymetric_shap_def}
   \phi_{i}^{assym}(\charactheristicFunction) = \sum_{\pi \in \perm(\players)} w(\pi) \left[ \charactheristicFunction(\pi_{<i} \cup {i}) - \charactheristicFunction(\pi_{<i}) \right] 
\end{align}

Asumiendo \(\sum_{\pi \in \perm(\players)} w(\pi) = 1\), esta es la expresión más general para cualquier función que satisfaga Eficiencia, Linealidad y Jugador Nulo \cite{frye2019asymmetric}. Para cualquier función de peso diferente de la uniforme, \(\phi_i^{assym}\) no satisface Simetría (y de ahí el nombre).

En \cite{frye2019asymmetric}, se definen los \textit{Asymmetric Shapley Values} considerando la definición de la Ecuación~\ref{eq:assymetric_shap_def} y una función de peso basada en el grafo causal del espacio de features. Más formalmente, se asume que tenemos acceso a un DAG (Directed Acyclic Graph) \(G = (X, E)\), donde los nodos de \(G\) son los features \(X\). El conjunto \(\topo(G)\) de órdenes topológicos de \(G\) es un subconjunto de \(\perm(X)\), y podemos definir una función de peso \(w\) de la siguiente manera\footnote{Es un problema $\sharpPhard$ calcular $|\topo(G)|$  para cualquier DAG \cite{countingLinearExtensions}, pues un orden topológico es una extensión lineal. Pero para algunas familias de dígrafos es posible calcularlo en tiempo polinomial, como los polytrees con grado acotado, como vamos a ver en la sección \ref{subSection:polytreeCountingComplexity} }):

\[
w(\pi) = \begin{cases}
\frac{1}{|\topo(G)|}  & \pi \in \topo(G) \\
0 & \text{en otro caso}
\end{cases}    
\]


y los \textit{Asymmetric Shapley Values} como:

\[
\assym_{M,e,\Pr}(x_i) = \frac{1}{|\topo(G)|} \sum_{\pi \in \topo(G)} [\charactheristicFunction_{M,e,\Pr}(\pi_{<i} \cup \{x_i\}) - \charactheristicFunction_{M,e,\Pr}(\pi_{<i})] 
\]

Intuitivamente, los Asymmetric Shapley Values filtran permutaciones que no respetan la causalidad definida por el DAG $G$. En el ejemplo que vimos en \ref{asvCaseExample}, una permutación que respeta la causalidad sería $\topo$, tal que $\topo[\text{edad}] < \topo[\text{educación}]$, ya que $(\text{edad}, \text{educación})$ es un eje de $G$. Nos importan estas permutaciones ya que queremos evaluar la mejora del modelo sabiendo la edad, y luego cuánto mejora el modelo si conocemos la educación, \emph{además} de la edad. De esta forma si la educación queda fija, la edad también, por lo que $ASV$ va a terminar asignando \emph{más importancia a las causas}, ya que las evalúa primero, y \emph{menos importancia a las consecuencias}, puesto que las evalúa cuando la causa ya fue incluida. Esto resulta deseable en contextos explicativos, donde típicamente nos interesa priorizar las variables que originan un fenómeno, y no aquellas que simplemente son consecuencia del mismo.

%\sidesergio{Creo que aportaría mucho poner un ejemplo desarrollado} % \santi{Queda claro que Sergio quiere ejemplos}

Este grafo causal se introdujo para modelar correlaciones entre las variables a nivel del puntaje mismo, independientemente de la distribución subyacente. Sin embargo, podemos considerar que, en lugar de un grafo causal, se nos proporciona una Red Bayesiana que describe la distribución del espacio de features, la cual en particular contiene un DAG que podemos emplear como grafo de causalidad. Lo que estamos haciendo es \emph{tomar a la red bayesiana como nuestro digrafo causal}.
Esto es clave a la hora de tener en cuenta los distintos experimentos realizados en este trabajo, puesto que tenemos redes bayesianas pero no digrafos causales. Aun así, se podría introducir un grafo causal distinto como input y los algoritmos presentados en las secciones siguientes funcionarían de igual manera. 

%\echu{¿Hace falta hacer más enfasis en esto? ¿Habría que justificarlo de otra forma?}

%\santi{Creo que está bien. Me gusta la itálica para enfatizar que es una suposición que estamos tomando. A lo sumo podemos decir que tiene sentido porque una red bayesiana bien armada debería tener las variables más independientes arriba (creo), pero qsy.}

En \cite{van2022tractability}, se demostró que calcular los Shapley values para una Naive Bayes es \(\NPhard\), al considerar el modelo trivial \(f(x_1,\ldots,x_n) = x_1\). Lo que esto nos dice es que calcular SHAP para una distribución y un modelo de poca complejidad ya resulta intratable. Una Naive Bayes es una red cuyo DAG tiene forma de estrella:  hay un único nodo \(x_1\) tal que el conjunto de aristas es \(E = \{(x_1, x_j) : 2 \leq j \leq n\}\) (\(x_1\) es padre de todos los nodos, y no hay otras aristas, como se puede ver en la Figura \ref{fig:naiveBayesExample}).  
%Esto es el teorema 8 del paper. 

\begin{figure}[ht]
    \centering
    \scalebox{0.75}{
        \begin{tikzpicture}[
            every node/.style={circle, minimum size=1.2cm, font=\small, text=white},
            class/.style={draw, fill=red!70},
            feature/.style={draw, fill=blue!60},
            node distance=1.8cm and 1.8cm,
            ->, thick
        ]
    
            % Central class node
            \node[class] (Disease) {Enfermo};
    
            % Features aligned horizontally below
            \node[feature, below left=of Disease, xshift=-3.6cm] (Fever) {Fiebre};
            \node[feature, right=of Fever] (Cough) {Tos};
            \node[feature, right=of Cough] (Fatigue) {Fatiga};
            \node[feature, right=of Fatigue] (Test) {Dolor};
            \node[feature, right=of Test] (Age) {Edad};
    
            % Edges
            \draw (Disease) -- (Fever);
            \draw (Disease) -- (Cough);
            \draw (Disease) -- (Fatigue);
            \draw (Disease) -- (Test);
            \draw (Disease) -- (Age);
    
        \end{tikzpicture}
    }
    \caption{Distribución Naive Bayes para predicción de enfermedades. La variable \textbf{Enfermo} influencia al resto.}
    \label{fig:naiveBayesExample}
\end{figure}


%\santi{Un dibujo de la red Naive? Digo para también tener un primer dibujo de redes bayesianas. }
Para calcular la probabilidad de una entidad \(e\) en una Naive Bayes, lo que hacemos es calcular la probabilidad de $x_1$ (la raíz) y luego la probabilidad del resto de los nodos condicionados en el valor $x_1$. La fórmula la podemos ver a continuación:
\[
\Pr[e] = \Pr[X_1 = e(x_1)] \prod_{j=2}^n \Pr[X_j = e(x_j) | X_1 = e(x_1)]
\]

Teniendo en cuenta que nuestra distribución es una Naive Bayes, calcular los Asymmetric Shapley Values (considerando la red misma como el DAG de causalidad) puede hacerse en tiempo polinomial para una familia de modelos más grande, a diferencia de los Shapley values usuales. Más específicamente, para cualquier modelo que permita calcular los Shapley values normales para la distribución producto:

%\santi{Ese amplia me parece un poco vendehumo, diría directamente lo que dice el teorema simplificando un toque.}

\begin{theorem}\label{the:assym_naive_equivalent_mean_prediction}
Los Asymmetric Shapley Values pueden calcularse en tiempo polinomial para distribuciones dadas como una Red Bayesiana Naive y para una familia de modelos \(\mathcal{F}\) si y solo si los Shapley values pueden calcularse para la familia \(\mathcal{F}\) bajo una distribución producto arbitraria en tiempo polinomial.
\end{theorem}

La demostración de este teorema se puede encontrar en el apéndice en la sección \ref{subsubSection:proofASVPolynomialNaiveBayes}.

A raíz del resultado obtenido con el Teorema \ref{the:assym_naive_equivalent_mean_prediction}, nos gustaría encontrar modelos en los cuales se pueda calcular la predicción promedio en tiempo polinomial, puesto que si no podemos calcular el promedio en tiempo polinomial, es razonable pensar que calcular ASV tampoco resultaría tratable, ya que en principio habría que evaluar al promedio. Por lo tanto, decidimos enfocarnos puntualmente en los árboles de decisión. En la siguiente sección vamos a introducir más formalmente a las redes bayesianas y cómo calcular el promedio para modelos del tipo \emph{Decision Trees}, con datos que tienen como su distribución a una red bayesiana, siendo esta también su grafo causal. 

%\santi{No entiendo a que va esta oración. El teorema conocido te dice que si podés calcular el promedio de la producto entonces tenes Shap, y ahora nosotros agregamos que tenés Assym para naive Bayes. Dicho eso, no se a que va tu oración.}

\section{Grafos Causales}

%\santi{Para mi esto debería estar antes. Principalmente porque ya aparecio una red bayesiana en la demo anterior. Capaz podes ponerlo pegado a la seccion 3.1, que es también medio introductoria. Sino también puede ir antes}

\begin{comment}
    Una Red Bayesiana para features \(X\) es una tupla \(\aBayesianNetwork = (X, E, \Pr)\), donde \((X, E)\) es un DAG que tiene los features \(X\) como nodos, $E$ sus aristas y \(\Pr\) es una función que codifica, para cada feature \(X\), su distribución de probabilidad condicional \(\Pr(X | \parents(X))\). La semántica topológica especifica que cada variable es condicionalmente independiente de sus no-descendientes, dado sus padres (y es por esto que la información de \(\Pr\) es suficiente para reconstruir la distribución conjunta de \(X\)).

\echu{¿Tiene sentido poner esta definición acá si después vamos a definirlo de vuelta en la siguiente sección? Podemos poner sólo la de la naive bayes sin tener en cuenta la de la red bayesiana. }
\sergio{Creo que es mejor evitar repetición directa, y además es mejor si se introducen los nuevos conceptos de manera más gradual cuando tiene sentido}
\santi{Banco a muerte no repetir, pero soy de la escuela de definir todo al principio. Da igual de todas formas.}

Dada una Red Bayesiana \(\aBayesianNetwork\), sea \(\pi \in perm(X)\) un orden topológico para el DAG de \(\aBayesianNetwork\). Entonces, la probabilidad para alguna entidad \(e\) se calcula como \footnote{A partir de ahora, denotamos por \(X_i\) la variable aleatoria asociada al feature \(x_i\).}

\begin{align}\label{eq:bayesian_probability}
    \Pr[e] = \Pr\left[\bigwedge_{i=1}^n X_i = e(x_i)\right] &= \prod_{i=1}^n \Pr\left[X_{\pi(i)} = e(x_{\pi(i)}) | \bigwedge_{i=1}^{i-1} X_{\pi(i)} = e(x_{\pi(i)})\right]\nonumber \\
    &= \prod_{i=1}^n \Pr\left[X_{\pi(i)} = e(x_{\pi(i)}) | \bigwedge_{x_j \in \parents(x_i)} X_j = e(x_j)\right]
\end{align}

donde en la última desigualdad usamos las restricciones topológicas para condicionar únicamente en los padres de \(x_i\). \cite{Darwiche_2009} Esta fórmula se basa en utilizar el teorema de Bayes para descomponer a la conjunción de valores que representa la entidad $e$ en varias probabilidades condicionales utilizando las distintas dependencias entre las variables que tiene un orden tópologico. 

Este modelo se basa en un grafo acíclico dirigido (DAG) donde la variable objetivo \(Y\) es el único nodo padre y todos los features \(X_i\) son hijos independientes. Dado que cualquier familia razonable de modelos contiene este tipo de funciones, la posibilidad de desarrollar algoritmos eficientes para calcular los Shapley values bajo distribuciones bayesianas parece limitada. Aún así hay familias de modelos para los cuáles se puede computar si tenemos una distribución de Markov \cite{marzouk2024tractabilityshapexplanationsmarkovian}.

%\echu{¿Meto esto que esta en el comment en algún lado o ya lo elimino? No se si suma tanto } Rta Cifu: No suma tanto, afueraa esto

\end{comment}

Como mencionamos en la sección anterior, $ASV$ utiliza el grafo causal asociado al problema para definir una función $w$, que nos va a permitir filtrar permutaciones no deseadas que no respeten la causalidad. Luego, cuando tenemos una permutación $\topo$ que sí nos interesa, vamos a querer realizar la operación $\charactheristicFunction(\pi_{<i})$, la cual consiste en evaluar la esperanza teniendo en cuenta nuestra distribución elegida y un valor fijo para los features en $\pi_{<i}$. Para realizar este cálculo necesitamos entender cómo se calculan probabilidades en una red bayesiana y cuál es su complejidad. 

\subsection{Redes Bayesianas}
Una Red Bayesiana $\aBayesianNetwork$ es un $DAG$, donde cada nodo representa una variable diferente y los arcos representan las dependencias condicionales entre ellas, puesto que para calcular la probabilidad de la cabeza necesitamos la de la cola. Por ejemplo, si tenemos el arco $A \longrightarrow B$, esto representa que la variable $B$ depende\footnote{Aunque no necesariamente significa que $A$ dependa de $B$ causalmente, puesto que podríamos armar una red bayesiana que no se condiga con el DAG causal de las variables. } de $A$, y esto se cuantifica usando la probabilidad condicional $P(B|A)$. Un nodo puede tener múltiples padres, por lo que la distribución de los valores del nodo se definirá en una \emph{Tabla de Probabilidad Condicional} (CPT), que define la probabilidad de que tome algún valor, dado los posibles valores de sus nodos padres. Con esto, podemos definir una Red Bayesiana como:

%\santi{Diría que esto es \textit{Intuitivamente}. Nada impide que armes una red bayesiana con todas las dependencias al revés. El truco es que si respetas la causalidad la red debería quedar con pocos ejes, pero en principio cualquier distribución se puede representar tomando cualquier ordenamiento de las variables aleatorias. La definción correcta correcta es que tenés tablitas, y se definen en función de los padres, y todas las tablas juntas dan una distribución.} Echu : Con dependencias condicionales me refería a que dependen en la probabilidad condicional de sus padres
% \santi{Creo que esto queda mejor en un footnote.}

\begin{definition}
Una Red Bayesiana para variables \(X\) es una tupla \(\aBayesianNetwork = (X, E, \Pr)\), donde \((X, E)\) es un DAG que tiene las variables \(X\) como nodos, $E$ como aristas y \(\Pr\) es una función que codifica, para cada variable \(x \in X\), su distribución de probabilidad condicional \(\Pr(x | \parents(X))\):

\begin{itemize}
    \item Para cada variable $x \in X$, sabemos que $x$ tiene un conjunto finito de estados mutuamente excluyentes. Estos son los posibles valores que puede tomar.
    \item Para cada variable $x \in X$ con padres $B_1, \dots, B_n$, se tiene una tabla de probabilidad condicional (CPT) $Pr(x \mid B_1, \dots, B_n)$. Así es como está definida $Pr$.
\end{itemize}
    
\end{definition}

\begin{figure}[ht]
    \centering
    \includegraphics[scale=0.4]{img/bayesianNetworks/bayesianNetworks.png}
    \caption{Red Bayesiana para determinar la inteligencia de un estudiante. Podemos ver que cada nodo tiene una $CPT$ que se calcula en función de sus padres. Aquí las variables son ternarias o binarias. Fuente: \cite{probabilisticGraphicalModels}}
    \label{fig:bayesian_network_example}
\end{figure}

 
Este tipo de redes nos ayuda a simplificar el cálculo de distribuciones de probabilidad conjunta. Una de sus propiedades más importantes es la \emph{regla de la cadena}, que proporciona una estructura fundamental para entender cómo las probabilidades conjuntas pueden descomponerse en estas redes. Específicamente, establece que si consideramos que $\aBayesianNetwork$ es una red Bayesiana sobre un conjunto de variables \( \{A_1, \dots, A_n\} \), entonces $\aBayesianNetwork$ especifica una única distribución de probabilidad conjunta \( Pr(U) \), donde \( U = \{A_1, \dots, A_n\} \), que puede expresarse como el producto de todas las CPT's especificadas en $\aBayesianNetwork$. Matemáticamente, se representa como:
\begin{align}\label{eq:bayesian_probability}
    Pr(U) = \prod_{i=1}^{n} P(A_i \mid \parents(A_i))    
\end{align}

donde $\parents(A_i)$  denota los padres de \( A_i \) en la red. Esta regla es esencial porque simplifica el cálculo de la distribución de probabilidad conjunta al descomponerla en dependencias locales más simples entre un nodo y sus predecesores directos. Intuitivamente, lo que hace es utilizar las dependencias locales para definir una distribución global, facilitando los procesos de inferencia dentro de estas redes.

%\santi{Acá estamos repitiendo las cosas que ya se dijeron unas páginsa anteriores al hablar por primera vez de redes bayesianas.}

Además, las redes Bayesianas contemplan una variedad de operaciones computacionales cruciales para el razonamiento probabilístico. Una de las operaciones principales es la \textbf{inferencia}, que implica calcular la distribución a priori para alguna variable. Es decir, calcular $Pr(A \mid B_1  \dots B_i)$, siendo $A$ un nodo en la red y $B_i$ otros nodos de la red. Este proceso a menudo se denomina \emph{consulta} a la red. La complejidad de la inferencia depende de la estructura de la red; para redes con forma de polytree (DAG's que su grafo subyacente es un árbol), también conocidas como \emph{simplemente conectadas}, la complejidad es polinomial en el tamaño de las variables de la red \cite{pearl1986bayesianInference}, pero para redes generales la inferencia es $\sharpPhard$. 

%\santi{Hace falta que sean padres? No es en general?} Rta: Es general, pifie

El algoritmo que se utiliza  para realizar la inferencia marginal, que calcula cual es la probabilidad de que una variable tome ciertos valores, es \emph{Variable Elimination} \cite{variableElimination}. Su complejidad depende principalmente del treewidth\footnote{El treewidth (ancho de árbol) de un grafo es una medida de cuán ``cercano´´ está un grafo a ser un polytree. Formalmente, es el tamaño del mayor conjunto de vértices en una descomposición en árbol, menos uno, minimizado sobre todas las posibles descomposiciones.} de la red y puntualmente para redes de un treewidth acotado, su complejidad es polinomial respecto al tamaño de la red. En base a esto es que elegimos trabajar con polytrees. 

%\santi{Hace falta poner la explicación de variable elimination? Al final no lo usamos nunca, no?}

\subsection{Predicción promedio en árboles de decisión}

Los árboles de decisión son ampliamente utilizados en Inteligencia Artificial (IA) debido a su capacidad para realizar predicciones y clasificaciones basadas en features de los datos de entrada. Al descomponer decisiones en una serie de preguntas y respuestas simples, los árboles de decisión permiten a los usuarios entender el razonamiento detrás de las predicciones del modelo, contribuyendo a la transparencia en sistemas complejos \footnote{Aunque no siempre el camino de un árbol es la mejor explicación, ya que puede haber features redundantes en el camino que no sean claves para la predicción realizada. \cite{audemard2021explanatorypowerdecisiontrees}}.
%\santi{Acá agregaría un footnote y una cita comentando que el camino en el árbol no es necesariamente una buena explicación. Hay discusiones sobre esto en papers de abductive explanations para árboles, por ejemplo \url{https://arxiv.org/pdf/2108.05266} (último párrafo pag 2)}

\begin{definition}
Un \textbf{árbol de decisión} es una estructura jerárquica utilizada para representar funciones de decisión sobre un conjunto finito de variables. Formalmente, un árbol de decisión \(T\) sobre un conjunto de variables \(X\) es un árbol enraizado cuyas componentes principales son:

\begin{itemize}
    \item \textbf{Nodos internos}, cada uno asociado a una variable \(x \in X\) y etiquetado con una condición sobre dicha variable. Estos nodos determinan cómo se bifurcan los datos.
    
    \item \textbf{Ramas}, que conectan un nodo padre con sus hijos y representan los posibles valores o resultados de evaluar la condición del nodo padre.
    
    \item \textbf{Hojas}, que son nodos sin hijos y contienen una salida concreta: una clase, un valor numérico, o una distribución de probabilidad, dependiendo del tipo de problema (clasificación, regresión, o probabilístico).
\end{itemize}

Cada instancia \(e \in \entities(X)\) se evalúa recorriendo el árbol desde la raíz hasta una hoja, tomando decisiones en función de las condiciones de los nodos internos. El resultado asociado a la hoja alcanzada es la predicción del modelo para esa instancia.

\end{definition}

Como mencionamos previamente, investigamos si el cálculo del promedio era tratable para árboles de decisión con distribuciones de redes bayesianas, ya que nuestro objetivo era calcular ASV en tiempo polinomial. %\santi{Ídem comentario de más arriba. Hacemos esto porque después lo vamos a necesitar para cacular $v(\pi_{<i})$, no por el Teorema 1, que no dice mucho sobre distribuciones no producto.}%\santi{Reescribiría esta oración.}

%\sidesergio{recordar no poner referencias sueltas, poner Teorema ref. Relacionadamente, por consistencia, capitalizar esas referencias (Teorema, Ecuación, etc.)}

%\santi{Sería un poco más formal en la definición.} 

Para este algoritmo\footnote{Pueden encontrar este algoritmo en \path{\pasantia-BICC\asvFormula\bayesianNetworks\bayesianNetwork.py}} trabajaremos con árboles de decisión binarios y con features binarios, más adelante veremos la extensión a variables no binarias. Cada nodo determinará un valor para un feature $X_i$, donde el hijo izquierdo corresponde al caso $X_i=0$ y el hijo derecho al caso $X_i=1$. Los features son un conjunto $X$, y su distribución de probabilidad será una red bayesiana $\aBayesianNetwork = (V, E, Pr_B)$. Definimos a $ev$ y $pathCondition$ cómo conjuntos de asignaciones $X_i = k$, las cuales definen que el feature $X_i$ toma el valor $k$. Entonces, $Pr_B(pathCondition \mid ev)$ representa la probabilidad de que dada una instancia que tiene los valores de $ev$, la instancia tome los valores de $pathCondition$, dada la red bayesiana $\aBayesianNetwork$. La idea del algoritmo consiste en explorar todas las ramas e ir acumulando las decisiones tomadas en la variable $pathCondition$. Al llegar a la hoja evaluamos la probabilidad de haber llegado a la hoja dada la evidencia y la multiplicamos por el valor que la misma devuelve. 

%\santi{Agregaría un párrafo comentando la idea del algoritmo. Aparte, si asumimos que en ningún camino se repiten features podríamos quitar las líneas 2-4, no? Aunque cuando haya más de un valor para cada feature eso deja de ser cierto. Capaz podemos comentarlo (igual no es muy relevante)}

%\santi{Qué es una evidencia? Estás pisando la $E$ de la definición de la red.}

%\echu{No entendí el comentario de Santi de lo de las líneas 2-4}

%Ya se entendio


\begin{algorithm}
\caption{Predicción promedio para árbol de decisión binario} \label{alg:meanPredBinDT}
\begin{algorithmic}[1]
\Function{Mean}{$node$, $B$, $pathCondition$, $evidence$}
    \If{$evidence$ does not match $pathCondition$}
        \State \Return  $0$
    \EndIf
    
    \If{$node$.isLeaf}
        \State \Return  $Pr_B(pathCondition\ | \ evidence) \cdot node.value$
    \EndIf
    \State $X_i \gets node.feature$
    \State $leftMean \gets$  \Call{Mean}{$node$.left, $B$, $pathCondition \cup \set{X_i=0}, evidence$}
    \State $rigthMean\gets$ \Call{Mean}{$node$.right,$B$,$pathCondition \cup \set{X_i=1}, evidence$} 
    \State \Return  \mbox{$leftMean + rigthMean$}
\EndFunction
\end{algorithmic}
\end{algorithm}

Analicemos la complejidad del Algoritmo \ref{alg:meanPredBinDT}. La condición del primer \texttt{if} puede ser evaluada en tiempo $O(|V|)$. En las hojas solo hacemos un llamado al algoritmo de \textit{variable elimination}, cuya complejidad denotamos como $O(varElim)$. Por otro lado, en los nodos internos solo se hacen los llamados recursivos extendiendo la \textit{pathCondition}, lo cual puede implementarse en $O(1)$. Por lo tanto, si $l$ denota la cantidad de hojas de nuestro árbol de decisión, e $i$ la cantidad de nodos internos, la complejidad de nuestro algoritmo es $O(i|V| + (varElim)l)$. En el caso de los polytrees $varElim$ es polinomial, por lo que nuestro algoritmo realizaría una cantidad de operaciones polinomial en función del tamaño del árbol de decisión, siendo estas ($O(i + l) = O(|V|)$) operaciones polinomiales. %, de costo polinomial $O(varElim) \ y \ O(|V|)$.

%\santi{Lineal?}

%\santi{Definir antes que la red es una tupla (V, E, Pr) y acá directamente poner $O(|V|)$}
%\santi{Ojo que cuando hablaste de VE lo consideraste un algoritmo para evaluar $P(A)$, y ahora querés evaluar $P(A | B)$. Podés directamente introducirlo como algo para calcular $P(A|B)$ y obtenés lo otro como caso particular.}
%\santi{Usaría otra expresión distinta a $VE$}

%\santi{Sobre tu comment en el latex, podés no decir polinomial hasta el final, solo poné las complejidades y aclará que VE es polinomial.}

%¿Polinomial en base a que? ¿Aclaro de vuelta todo lo anterior? Siento que estoy diciendo todo el tiempo lo mismo, necesito sinonimos de polinomial. ¿Lo hago más formal?

\begin{figure}[ht]
    \centering
\begin{tikzpicture}[
    level 1/.style={sibling distance=6cm},
    level 2/.style={sibling distance=3cm},
    every node/.style={minimum size=1cm, font=\footnotesize}
]
\node [circle, draw, red] {Letter}
    child {
        node [circle, draw] {Den}
        edge from parent node[left] {\textcolor{red}{$X_0 = 0$}}
    }
    child {
        node [circle, draw, blue] {SAT}
        child {
            node [circle, draw] {Den}
            edge from parent node[left] {\textcolor{blue}{$X_1 = 0$}}
        }
        child {
            node [circle, draw] {Acc}
            edge from parent node[right] {\textcolor{blue}{$X_1 = 1$}}
        }
        edge from parent node[right] {\textcolor{red}{$X_0 = 1$}}
    };
\end{tikzpicture}

    \caption{Decision tree que define si un estudiante va a ser aceptado (accepted/1) o rechazado (denied/0) en su ingreso a una universidad basado en los features de la red bayesiana de la Figura \ref{fig:bayesian_network_example}. Podría ser generado a partir de un dataset con los mismos features de la red, con un feature extra llamado \texttt{Acceptance}. La red usada puede encontrarse en el repositorio}%en $\backslash pasantia-BICC\backslash networkExamples \backslash student.bif$}
    \label{fig:decision_tree_example}
\end{figure}

%\santi{En el ejemplo deberia haber dos ejes saliendo de Letter (en rojo), o sacar el nodo Letter (en rojo) y solo poner el Den.}

% Tengamos en cuenta que el feature \santi{En general no deberíamos mezclar el target con los features. Entiendo que en la experimentación surgió naturalmente por las redes de ejemplo que tomamos, pero en general para nuestro framework features y valores objetivos son dos cosas complementamente distintas, y no suponemos que tenemos información respecto a su correlación (aunque si la tenemos en nuestros casos de test). Dicho esto, este párrafo no debería estar.} que intentamos predecir no está en nuestra red bayesiana, si este fuera el caso, entonces lo que utilizaríamos para calcular la predicción promedio del feature sería la red misma, no el árbol de decisión.  Rta: Completamente de acuerdo con el comentario, no hace falta esta oración

En la Figura \ref{fig:decision_tree_example} tenemos un ejemplo de un árbol de decisión. Queremos ver cuál es la predicción promedio de nuestro árbol de decisión para un alumno que es inteligente.  Si corremos nuestro algoritmo desde la raíz del árbol, lo que vamos a obtener es
\begin{align*}
        &\Call{Mean}{Letter, B, \{\}, \{INT = 1\}}\\
        &= \Call{Mean}{Den, B, \textcolor{red}{Letter = 0}, \textcolor{brown}{\{INT = 1\}}} + \Call{Mean}{SAT, B, \textcolor{red}{Letter = 1}, \textcolor{brown}{\{INT = 1\}}} \\
        &= 0 + \Call{Mean}{SAT, B, \{Letter = 1\}, \textcolor{brown}{\{INT = 1\}}} \\
        &= \Call{Mean}{Den, B, \{Letter = 0, \textcolor{red}{SAT = 0} \}, \textcolor{brown}{\{INT = 1\}}} + \Call{Mean}{Acc, B, \{Letter = 1, \textcolor{red}{SAT=1}\}, \textcolor{brown}{\{INT = 1\}}} \\
        &=  Pr_B(SAT = 0, Letter = 1\ | \ INT = 1) \cdot 0 +  Pr_B(SAT = 1, Letter = 1\ | \ INT = 1) \cdot 1 \\
        &= Pr_B(SAT = 1, Letter = 1\ | \ INT = 1) \\
        &= 0.61
    \end{align*}

En este ejemplo podemos ver que aunque la inteligencia no sea una variable que se tenga en cuenta en el árbol, afecta la predicción promedio, ya que para calcularla estamos utilizando la red bayesiana, y esta evidencia introducida va a afectar la inferencia realizada en la red. 

%\santi{Comentario sobre que INT no es una feature usada en el árbol y no obstante afecta la predicción promedio. Digo, porque es algo un poco antiintuitivo, pero es la gracia.}


%\echu{Duda, no se si hacer que las hojas sean  "Denied" y "Accepted" o que sean un "SAT" (o cualquier otra variable) y que ahí te devuelva las probabilidades para cada valor, entiendo que la primera es más "fiel" a lo que buscamos representar.}

%\santi{No entiendo este comentario}

%Respuesta: Deberíamos dejar la primera opción, porque estamos calculando ASV usando 0/1 cómo predicción, no una probabilidad. 

%\santi{Agregar el valor de la predicción esta cuando INT = 0, que es 0.39.} Rta: Al final es 0.019430000000000003, porque es la proba de que (SAT=1 y Letter = 1)

\begin{comment}
    "El código para correr la query. Intelligence puede valer i1 o i0"
    studentNetworkPath = networkSamplesPath + "/student.bif"
    BNmodel = BIFReader(studentNetworkPath).get_model()
    BNInference = VariableElimination(BNmodel)
    query = BNInference.query(evidence={'Intelligence':'i1'}, joint=True, variables = ['Letter', 'SAT'] )
    print(query.get_value(**{'Letter' : 'l1', 'SAT' : 's1'}))
\end{comment}

\subsubsection{Expandiendo la predicción promedio a features no binarios} 

%\echu{¿Juega esta sección?  Santi: Sip, pero muy chill. Contar que no podemos hacer la operación que queremos en la red bayesiana de una forma muy simple y mostrar la forma en la que la implementamos nosotros. }

El Algoritmo \ref{alg:meanPredBinDT} funciona para árboles binarios y para variables binarias. Para poder trabajar con features no binarios tuvimos que modificar la inferencia realizada, por lo que su complejidad dejó de ser polinomial en el tamaño de la red, ya que la implementación actual depende de la cardinalidad de cada feature.
%\santi{No lo sabemos esto. Habría que saber más de implementaciones de inferencia}. \scc{\sout{Puesto que ahora la cardinalidad de cada feature influye el tiempo que toma}}.

Si cada feature admite más de un valor, cuando llegamos a un nodo $n$ obtenemos su feature $f$ y su umbral de decisión $v$. Luego para los valores $i,d \in Dominio(f)$ se le agregan a $pathCondition$ los $i$ tal que $i<v$ en el lado izquierdo de la recursión y los $d$ tal que $d \geq v$ en el lado derecho. Seguimos realizando la inferencia al llegar al nodo hoja a través de una suma de la unión de todas las consultas generadas, por lo que la inferencia no es polinomial. Por ejemplo, si llegamos con $pathCondition = \set{x=\set{1,2},y=\set{3}}$ vamos a tener que evaluar $Pr_B(pathCondition) = Pr_B(x=1,y=3) + Pr_b(x=2,y=3)$. Para mejorar esta inferencia, una posibilidad sería implementar la consulta modificando el algoritmo de Variable Elimination o creando nodos intermedios que representen la evidencia introducida; pero no tomamos este camino debido a que no es el objetivo principal de esta tesis. 

%\echu{¿Así queda menos en ladri? En respuesta al comentario de Cifu}

%\santi{Hay que sacar esta oración o comentar que existe esta posibilidad. Decir que lo pensamos y no poner nada de resultado me parece ladri.} 

%https://stackoverflow.com/questions/76365165/create-variable-elimination-with-multiple-possible-values-in-pgmpy#:~:text=Unfortunately%2C%20there%20is%20no%20direct,the%20probability%20in%20such%20cases Acá preguntaron lo mismo que nosotros

%\santi{O hacer la query más compleja a la red bayesiana. Yo diría que el algoritmo ahora se reduce a esta query más compleja, que no sabemos si se puede resolver. La oración siguiente a esto solo es la observación de que descomponerla en queries tradicionales no es eficiente, pero capaz no hace falta y se puede manejar de otra forma.}



\begin{comment}
\subsection{Exact mean prediction computation in Decomposable Circuits}

Can we extend the previous algorithm to work in more general models? We can easily prove an upper bound on model complexity related to the satisfiability problem.

\begin{proposition}
    Let $M$ be a model over entities $\entities(X)$ and $\aDistribution$ a distribution over $\entities(X)$ such that no entity has probability 0. Then, deciding if $\expectancy_{e\sim \aDistribution}[M(e)]$ is positive is equivalent to deciding if $M$ is satisfiable.
\end{proposition}

This does not rule out the tractability of exactly computing the average of models for which the satisfiabilily problem is tractable, such as d-DNNF circuits \cite{arenas2021tractability}. For them, we can prove an intractability result exploiting the correlations that one can impose using the Bayesian Network, which allows us to turn a normal Boolean circuit into a d-DNNF one without altering the average prediction.

\begin{proposition}
    The problem of deciding, given a Bayesian distribution $\aBayesianNetwork$ and a d-DNNF circuit $\aCircuit$ whether $\expectancy_{e \sim \aBayesianNetwork}[\aCircuit(e)] > 0$ is \NPhard{}. Moreover, the results holds for Bayesian Networks which are union of disjoint paths.
\end{proposition}

\begin{proof}
    The problem clearly belongs to $\NP{}$, since we can solve it by guessing an entity $e$ such that $\aCircuit(e)$ and $\Pr[e] > 0$. For the hardness, we reduce \textsc{Circuit-SAT} to our problem.

    Let $\aCircuit$ be a Boolean circuit. Without loss of generality we may assume that it is deterministic (i.e. for each $\vee$ gate the two subcircuits $\aCircuit_1$ and $\aCircuit_2$ cannot be simultaneously satisfied). We are going to design a simple Bayesian distribution such that $\aCircuit$ can be understood also as a decomposable circuit.

    We proceed top-down. Let $\wedge$ be the \textit{and} node closest to the top, and let $\aCircuit_1$ and $\aCircuit_2$ the two subcircuits of this node. In principle, $var(\aCircuit_1) \cap var(\aCircuit_2) \neq \emptyset$. To fix this, let $x_1,\ldots, x_k$ be the variables shared by both subcircuits, and replace the variables $x_1,\ldots,x_k$ from $\aCircuit_2$ with the variables $x_1',\ldots,x_k'$. At the same time, we add to the Bayesian network nodes $x_1,x_1',\ldots,x_k,x_k'$ conditioning that whenever $x_i = 1$ then $x_i'=1$ with probability one, and similarly for the case $x_i = 0$, for all $i$.

    Observe that some entities have probability 0: they are exactly those that pick a different value for $x_i$ and $x_i'$, for some $i$. Therefore, if the expected value of this new circuit is positive then there is an assignment of the original circuit such that it is satisfied.

    To complete the reduction we continue working recursively: at each $\wedge$ gate we detect the shared variables $y_1,\ldots,y_\ell$, change them with variables $y_1',\ldots,y_\ell'$ and add the correlations in the Bayesian network $\aBayesianNetwork$.

    Note that the final structure of the Bayesian network is a set of disjoint paths, over which the inference problem be solved easily.
\end{proof}

\subsection{Approximate mean prediction computation}

Even though exact computation is intractable, approximate calculation is straightforward when considering additive precision.

\begin{proposition}
    Let $\aDistribution$ be any distribution over $\entities(X)$ that can be sampled efficiently, and $\mathcal{F}$ a family of models such that evaluating, given $M \in \mathcal{F}$ and $e \in \entities(X)$, the value $M(e)$,q can be done in polynomial time. Then, there is a \textit{Fully Polynomial-time Approximation Scheme} (FPRAS) for $\expectancy_{e \in \aDistribution}[M(e)]$ under additive error\footnote{I don't think we can achieve multiplicative error, looks like satisfiability.}.
\end{proposition}

\begin{proof}
    Note that $M(e) : \aDistribution \to \{0,1\}$ is a random variable bounded between $0$ and $1$ that can be sampled efficiently, and thus the result follows by considering the Hoeffding's inequality.
\end{proof}

\santi{Esto lo podemos escibir juntos la otra semana y ya queda como ejemplo para el otro algoritmo (el de sampleo de ordenes topológicos).}

%Tiene sentido poner que no se puede? Y la demo esa de convertir una red no deterministica en una deterministica en polinomial? O alguna otra justificación?

\end{comment}

\section{Heuristíca ASV}

%\subsection{Clases de equivalencia en \dtrees}


\begin{frame}{Heurística ASV: Recordando la Fórmula}
	\small
	\dificultyLevel{2}
	Recordemos la definición de ASV:
	\[
	    \assym_{M,e,\Pr}(x_i)
	    \;=\;
	    \frac{1}{|\topo(G)|}\sum_{\pi \in \topo(G)} 
	      \Bigl(\charactheristicFunction(\pi_{<i} \cup \{x_i\})
	      \;-\;\charactheristicFunction(\pi_{<i})\Bigr).
	\]
	\begin{itemize}[<+- | alert@+>]
	    \item Nuestro objetivo es minimizar la cantidad de veces que se evalúa $\charactheristicFunction$.
	    \item La idea de la heurística es identificar los órdenes topológicos que devuelven el mismo resultado al ser evaluados por $\charactheristicFunction$. 
	    \item Así solo tendremos que evaluar $\charactheristicFunction$ una vez por cada orden perteneciente a cada clase de equivalencia. 
	\end{itemize}
	% \only<4>{\begin{mydefinition}[Clase de equivalencia $\rel$] 	Sea \(A\) un conjunto y \(R \subseteq A \times A\) una relación de equivalencia. La clase de equivalencia $[a]_\rel$ son todos los elementos de $A$ relacionados con $a$.
	%Para cada \(a \in A\), la \emph{clase de equivalencia} de \(a\) bajo \(R\) se define como
	%\[	[a]_R \;=\; \{\,x \in A \mid a \ R \ x \}.	\] 	\end{mydefinition}}
	\only<2>{ 
		\alert{Podemos agruparlos según la relación $\rel^\star$, en la cual dos órdenes topológicos \(\pi^1,\pi^2\) están relacionados si evaluán a lo mismo.} %($\charactheristicFunction(\pi^1_{<i}) \;=\; \charactheristicFunction(\pi^2_{<i})$).
		}
\end{frame}

\begin{comment}
	\begin{frame}{Introducción a $\rel$}
		\dificultyLevel{2}
		\begin{itemize}[<+- | alert@+>]
			\item En la siguiente sección vamos a introducir la relación $\rel$, se define sobre los órdenes topológicos de nuestro grafo causal $G$. 
			\item Luego, para calcular las distintas clases de equivalencia vamos a tener que establecer ciertas relaciones entre los nodos y los órdenes de $G$. 
			\item Más adelante vamos a ver cómo contar los órdenes topológicos, ya que nos va a ayudar a calcular los tamaños de estas clases. 
			\item Por último, vamos a encontrar una cota para el número de clases de equivalencia de un \dtree. 
		\end{itemize}
	\end{frame}
\end{comment}


% ---------------------------------------------------------
\begin{frame}{Relación de Equivalencia \(\rel\)}
\dificultyLevel{3}
%\only<1>{Como \(\rel^\star\) es díficil de calcular, definimos una relación \(\rel\) más manejable.}
Definimos \(\rel\) para distinguir clases de equivalencia \(\equivalenceClass\)
para el feature \(x_i\):
\[
  \pi^1 \rel \; \pi^2 
  \quad\Longleftrightarrow\quad
  \{\pi^1_{<i}\} \;=\; \{\pi^2_{<i}\}
\]
\only<1>{%es decir, dos órdenes topológicos están en la misma clase si,  previo a la posición de \(x_i\), tienen el mismo conjunto de nodos.
	Ambos órdenes evaluan igual: $\alert{\charactheristicFunction(\pi_{<i})}$. }


\only<2>{
	\begin{figure}
		\centering
		\includegraphics[width=0.8\linewidth]{pic/img/SHAP/equivalenceClassTopoSortNotEqual.png}
	\end{figure}
}

\only<3>{
	\begin{figure}
		\centering
		\includegraphics[width=0.8\linewidth]{pic/img/SHAP/equivalenceClassTopoSortEqual.png}
	\end{figure}
}

\end{frame}

\begin{frame}{Heuristíca ASV}
	\dificultyLevel{2}
	Denotemos por 
	\[
	eqCl(G,x_i) \;=\; \{\,[\pi]_{\rel} \mid \pi \in \topo(G)\}
	\]
	al conjunto de clases de equivalencia respecto de \(\rel\). \pause Entonces nuestra fórmula original nos queda:
	\[
	%\assym_{M,e,\Pr}(x_i)
	%\;=\; 
	\frac{1}{|\topo(G)|}\sum_{\alert{[\pi]_{\rel} \,\in\, eqCl(G,x_i)}} 
	\Bigl(\charactheristicFunction(\pi_{<i} \cup \{x_i\}) 
	- \charactheristicFunction(\pi_{<i})\Bigr) 
	\;\cdot\; \alert{\bigl|\,[\pi]_{\rel}\bigr|}
	\]    
	en vez de: 
	\[
	%\assym_{M,e,\Pr}(x_i)
	%\;=\;
	\frac{1}{|\topo(G)|}\sum_{\pi \in \topo(G)} 
	\Bigl(\charactheristicFunction(\pi_{<i} \cup \{x_i\})
	\;-\;\charactheristicFunction(\pi_{<i})\Bigr)
	\]
\end{frame}


\begin{comment}
	\begin{mydefinition}[Clase de equivalencia de \(\rel\)]
		Sea \(D=(V,E)\) y \(x_i\in V\). Para \(\pi\in \topo(G)\), definimos 
		$f_{\pi}$ tal que $f_{\toOr}(n)$ es una función que identifica si el nodo $n$ está a la derecha o izquierda de $x_i$ en $\toOr$. \pause
		\begin{itemize}[<+- | alert@+>]
			\item La clase de equivalencia \(\equivalenceClass\) se representa con $\equivalenceClassRep$ por $V$ etiquetado por $f_{\pi}$. 
			%\(   \equivalenceClassRep     = \{\,v_{f_{\pi}(v)} \mid v \in V\setminus\{x_i\}\}.  \) 
			%\item $L(\equivalenceClass)$ y $R(\equivalenceClass)$ denotan el conjunto de nodos a la izquierda y a la derecha en la clase de equivalencia, respectivamente.
		\end{itemize}
	\end{mydefinition}
	
	\begin{frame}{Relación de Equivalencia $\rel$}
		\dificultyLevel{3}
		\begin{mydefinition}[Pertenencia a una clase]\label{def:belongsEquivClass}
			Sea $G$ un digrafo $G = (V, E)$. Un orden topológico $\toOr'$ pertenece a la clase de equivalencia $\equivalenceClass$ si se cumple que tiene los mismos nodos a la izquierda de $x_i$ que $\toOr$.
			\small
			%$$  (\forall v \in V \setminus \{x_i\}) (f_{\toOr'}(v) = left \land v \in L(\equivalenceClass) ) \lor (f_{\toOr'}(v) = right \land v \in R(\equivalenceClass) )    $$
			
		\end{mydefinition}
		
	\end{frame}
	\pause
	\smallskip
	\emph{Observación:} \(\rel\) es más fina que \(\rel^\star\). 
	Si dos órdenes tienen el mismo conjunto antes de \(x_i\), 
	no necesitamos evaluar \(\charactheristicFunction\) para agruparlos.
	
	% ---------------------------------------------------------
	%Si esto hubiera sobrevivido hubiera habido que se hacer más enfasis en que esto se utiliza para calcular los tamaños de las clases
	\begin{frame}{Número de Órdenes Topológicos en un DAG}
		\dificultyLevel{2}
		\small
		Queremos una fórmula para \(\lvert \topo(G)\rvert\) en familias especiales de DAGs, 
		porque en general es \(\sharpPhard\) \cite{countingLinearExtensions}. \only<1>{Esto lo vamos a necesitar para calcular los tamaños de las clases de equivalencia.} \pause 
		
		\medskip
		\textbf{Caso base: grafo vacío con \(r+1\) nodos (sin aristas).}
		\[
		\lvert \topo(G)\rvert = (r+1)!.
		\]
		\begin{figure}
			\centering
			\begin{tikzpicture}[scale=0.9, transform shape,
				nodo/.style={circle, draw, minimum size=0.8cm}]
				\node[nodo] (r)   at (0, 0) {$r$};
				\node[nodo] (h1)  at (-2,-2) {$h_1$};
				\node[nodo] (h2)  at (-0,-2) {$h_2$};
				\node[nodo] (hi)  at (2,-2) {$h_i$};
				\node[nodo] (hr) at (4,-2) {$h_r$};
				\node[draw=none, fill=none]  (dotsL) at (-1,-2) {$\ldots$};
				\node[draw=none, fill=none]  (dotsR) at (1,-2) {$\ldots$};
				\node[draw=none, fill=none]  (dotsRR) at (3,-2) {$\ldots$};
			\end{tikzpicture}
			\caption*{DAG sin aristas: todos los nodos son permutables.}
			\label{fig:emptyGraphExample}
		\end{figure}
	\end{frame}
	
	% ---------------------------------------------------------
	\begin{frame}{Conteo de Órdenes en \dtrees}
		\dificultyLevel{2}
		
		\small
		Ahora añadimos aristas para formar un \dtree. Sea \(D\) un \dtree\ con raíz \(r\). 
		
		
		
		\begin{figure}[ht]
			\centering 
			\begin{tikzpicture}[scale=.6, transform shape]
				
				% ---- NODOS ----
				\node[nodo] (r) at (0, 0) {$r$};
				\node[nodo] (s1) at (-4, -2) {$h_1$};
				\node[nodo] (s2) at (-2, -2) {$h_2$};
				\node[nodo] (si) at (1, -2) {$h_i$};
				\node[nodo] (sn-1) at (4, -2) {$h_{r}$};
				
				\node[draw=none, fill=none] (dots) at (-0.2, -2) {$\ldots$}; % Ellipsis
				\node[draw=none, fill=none] (dots) at (2.2, -2) {$\ldots$}; % Ellipsis
				
				\node[draw=none, fill=none] (h1) at (-4, -4) {};
				\node[draw=none, fill=none] (h2) at (-2, -4) {};
				\node[draw=none, fill=none] (hi) at (1, -4) {};
				\node[draw=none, fill=none] (hn-1) at (4, -4) {};
				
				
				\path [->] (r) edge[arista]  (s1);
				\path [->] (r) edge[arista]  (s2);
				\path [->] (r) edge[arista]  (si);
				\path [->] (r) edge[arista]  (sn-1);
				
				\path [->] (s1) edge[arista,  mySnake]  (h1);
				\path [->] (s2) edge[arista,  mySnake]  (h2);
				\path [->] (si) edge[arista,  mySnake]  (hi);
				\path [->] (sn-1) edge[arista,  mySnake]  (hn-1);
			\end{tikzpicture}
			\caption*{Polytree con nodos con grados de entrada menores o iguales a 1, lo cual definimos como \dtree.}
			\label{fig:dtreeExample}
		\end{figure}
	\end{frame}
	
	\begin{frame}{Conteo de Órdenes en \dtrees}
		\dificultyLevel{3}
		\textbf{Fórmula general:}
		
		\only<1>{
			\begin{mydefinition}
				Sean \(k_i\) la cantidad de nodos del subárbol \(t_i\), con 
				\(n = \sum_{i=1}^r k_i\). La cantidad de órdenes topológicos es:
			\end{mydefinition}
		}
		
		\only<2>{
			\begin{mydefinition}
				Sean \(k_i\) la cantidad de nodos del subárbol \(t_i\), con 
				\(n = \sum_{i=1}^r k_i\). La cantidad de órdenes topológicos es:
				\[
				\numTopo(t) 
				= \alert<2>{\binom{n}{k_1,\dots,k_r}}
				\]
			\end{mydefinition}
		}
		
		\only<3->{
			\begin{mydefinition}[Órdenes Topológicos en un \dtree]
				Sean \(k_i\) la cantidad de nodos del subárbol \(t_i\), con 
				\(n = \sum_{i=1}^r k_i\). La cantidad de órdenes topológicos es:
				\[
				\numTopo(t) 
				= \binom{n}{k_1,\dots,k_r}
				\;\cdot\;
				\alert<3>{\prod_{i=1}^{r} \numTopo(t_i)}
				\]
			\end{mydefinition}
		}
		
		\begin{itemize}
			\item<2-> \alert<2>{Coeficiente multinomial: 
				\(\binom{n}{k_1,\dots,k_r} = \frac{n!}{k_1!\cdots k_r!}\)} 
			cuenta las maneras de intercalar nodos de subárboles sin alterar su orden interno.
			\item<3-> \alert<3>{Producto de subárboles: 
				\(\prod_{i=1}^{r} \numTopo(t_i)\)} 
			corresponde a las combinaciones posibles dentro de cada subárbol.
			\item<4-> \alert<4>{Combinación final: la fórmula multiplica ambas partes para obtener el total de órdenes topológicos.}
		\end{itemize}
	\end{frame}
	
	\begin{frame}{Extensión a \textit{Polyforests} con Raíz Virtual}
		\dificultyLevel{2}
		\small
		Si nuestro DAG es un \emph{polyforest} (varias raíces \(r_1,\dots,r_\ell\)), 
		podemos agregar una raíz virtual \(r_0\) que conecte a todas las raíces originales. 
		\(r_0\) actúa como raíz de un \dtree\ que engloba todo el polyforest:
		
		\begin{figure}[ht]
			\centering 
			\begin{tikzpicture}[scale=.6, transform shape]
				
				% ---- NODOS ----
				\node[nodo, blue] (r) at (0, 0) {$r_0$};
				\node[nodo] (s1) at (-4, -2) {$r_1$};
				\node[nodo] (s2) at (-2, -2) {$r_2$};
				\node[nodo] (si) at (1, -2) {$r_i$};
				\node[nodo] (sn-1) at (4, -2) {$r_{l}$};
				
				\node[draw=none, fill=none] (dots) at (-0.2, -2) {$\ldots$}; % Ellipsis
				\node[draw=none, fill=none] (dots) at (2.2, -2) {$\ldots$}; % Ellipsis
				
				\node[draw=none, fill=none] (h1) at (-4, -4) {};
				\node[draw=none, fill=none] (h2) at (-2, -4) {};
				\node[draw=none, fill=none] (hi) at (1, -4) {};
				\node[draw=none, fill=none] (hn-1) at (4, -4) {};
				
				
				\path [->] (r) edge[arista, blue]  (s1);
				\path [->] (r) edge[arista, blue]  (s2);
				\path [->] (r) edge[arista, blue]  (si);
				\path [->] (r) edge[arista, blue]  (sn-1);
				
				\path [->] (s1) edge[arista,  decorate, decoration={snake, amplitude=.4mm, segment length=4mm, post length=1mm}]  (h1);
				\path [->] (s2) edge[arista,  decorate, decoration={snake, amplitude=.4mm, segment length=4mm, post length=1mm}]  (h2);
				\path [->] (si) edge[arista,  decorate, decoration={snake, amplitude=.4mm, segment length=4mm, post length=1mm}]  (hi);
				\path [->] (sn-1) edge[arista,  decorate, decoration={snake, amplitude=.4mm, segment length=4mm, post length=1mm}]  (hn-1);
			\end{tikzpicture}
			\caption*{El DAG original son los nodos y aristas en negro, el nodo virtual y sus aristas están en azul.}
			\label{fig:topoSortCalcForForests}
		\end{figure}
		
		Así, usamos la misma fórmula multinomial 
		\(\numTopo(t)\) vista para \dtrees, ahora aplicable a polyforests.
	\end{frame}
\end{comment}


\begin{comment}
	
	% ---------------------------------------------------------
	\begin{frame}{Más definiciones: Ancestros, Descendientes y Nodos No Relacionados}
		\dificultyLevel{2}
		\begin{figure}[ht]
			\centering 
			\begin{tikzpicture}[scale=.8, transform shape]
				
				% ---- NODOS ----
				\node[nodo] (a1) at (0, 0) {$a_1$};
				\node[nodo] (a2) at (2.5, 2) {$a_2$};
				\node[nodo] (a3) at (2.5, -2) {$a_3$};
				\node[nodo] (xi) at (5, 0) {$x_i$};
				\node[nodo] (d1) at (7.5, 2) {$d_1$};
				\node[nodo] (d2) at (10, -2) {$d_2$};
				\node[nodo] (d3) at (10, 0) {$d_3$};
				
				\node[draw=none, fill=none] (hijoa3) at (4.5, -4) {};
				\node[draw=none, fill=none] (hijod2) at (8, -4) {};
				
				% ---- ARISTAS ----
				\path [->] (a1) edge[arista, mySnake]  (xi);
				\path [->] (a2) edge[arista, mySnake]  (xi);
				\path [->] (a3) edge[arista, mySnake]  (xi);
				\path [->] (xi) edge[arista, mySnake]  (d1);
				\path [->] (xi) edge[arista, mySnake]  (d2);
				\path [->] (xi) edge[arista, mySnake]  (d3);
				\path [->] (a3) edge[arista, mySnake] node[above right] {descendientes de $a_3$ } (hijoa3);
				\path [->] (hijod2) edge[arista, mySnake] node[below right] {ancestros de $d_2$ } (d2);
				
			\end{tikzpicture}
			\caption*{Al fijar un nodo $x_i$, podemos dividir el resto de los nodos en tres grupos: \textit{ancestros} (todos los nodos que pueden alcanzar a $x_i$), \textit{descendientes} (todos los nodos alcanzables desde $x_i$) y aquellos \textit{no relacionados} con $x_i$.} %Los \textit{no relacionados} son los que no pertenecen a los ancestros ni a los descendientes, por lo que pueden estar a la derecha o la izquierda de $x_i$ en un orden topológico.}
		\label{fig:unrelatedNodesDefinition}
	\end{figure}
	\end{frame}
	
	\begin{frame}{Ancestros, Descendientes y Nodos No Relacionados}
		\dificultyLevel{2}
		\begin{itemize}[<+->]
			\item Sea \(A\) el conjunto de ancestros de \(x_i\) y \(D\) el de descendientes.
			\item Para cualquier \(\pi\in \topo(G)\), los nodos de $A$ siempre van a estar a la izquierda de $x_i$ y los nodos de $D$ a la derecha.
			\item Los \emph{nodos no relacionados} \(U = V \setminus (A\cup D\cup\{x_i\})\) 
			pueden ubicarse a la izquierda o a la derecha de \(x_i\), por lo tanto son los que definen las clases de equivalencia.
		\end{itemize}
	\end{frame}
\end{comment}

% ---------------------------------------------------------
\begin{frame}{Clases de Equivalencia en \dtrees: Ejemplo}
	\dificultyLevel{2}
	\begin{figure}[ht]
		\centering 
		\begin{tikzpicture}[scale=.5, transform shape, 
			unrelated/.style={circle, draw=red},
			wiggly/.style={decorate, decoration={snake, amplitude=.2mm, segment length=2mm}}  % Define wiggly line style
			]
			
			% ---- NODOS ----
			\node[nodo, blue] (r) at (0, 0) {$r$};
			\node[unrelated] (a1) at (-1, -2) {$a_1$};
			\node[nodo, blue] (a2) at (1, -2) {$a_2$};
			
			\node[unrelated] (b1) at (-1, -4) {$b_1$};
			\node[nodo, blue] (b2) at (1, -4) {$b_2$};
			\node[unrelated] (b3) at (3, -4) {$b_3$};
			
			\node[unrelated] (c1) at (0, -7) {$c_1$};
			\node[nodo, blue] (c2) at (3, -6) {$c_2$};
			
			% Nodo x_i ampliado
			\node[nodo, font=\Large, minimum size=1cm] (xi) at (3, -8) {$x_i$};
			
			\node[draw=none, fill=none] (hi) at (3, -10) {};
			
			\path [->] (r) edge[arista]  (a1);
			\path [->] (r) edge[arista]  (a2);
			
			\path [->] (a2) edge[arista]  (b1);
			\path [->] (a2) edge[arista]  (b2);
			\path [->] (a2) edge[arista]  (b3);
			
			\path [->] (b2) edge[arista]  (c1);
			\path [->] (b2) edge[arista]  (c2);
			
			\path [->] (c2) edge[arista]  (xi);
			
			
			\node[text=red, font=\Large] at (-2, -8) {$c_1$ subárbol}; 
			\draw[red, wiggly] (-1, -9) -- (0,-7.4) -- (1, -9) -- cycle;  % Draw the triangle
			
			\node[text=red, font=\Large] at (-3, -5) {$b_1$ subárbol}; 
			\draw[red, wiggly] (-2, -6) -- (-1,-4.4) -- (0, -6) -- cycle;  % Draw the triangle
			
			\path [->, teal] (xi) edge[arista,  decorate, decoration={snake, amplitude=.4mm, segment length=4mm, post length=1mm}] node[right, font=\Large] {descendientes de $x_i$} (hi);
		\end{tikzpicture}
		\caption*{Ejemplo de un \dtree{} con $x_i$ en negro, sus nodos no relacionados marcados en rojo y sus ancestros marcados en azul.} %Llamemos $UR$ al conjunto de raíces de los árboles no relacionados. }
		\label{fig:dtreeForestForEquivalenceClasses}
	\end{figure}
	
\end{frame}

\begin{comment}
	\begin{frame}{Clases de Equivalencia en \dtrees: Ejemplo}
		\dificultyLevel{2}
		\begin{itemize}[<+->]
			\item Solo los nodos en rojo (\(b_1,\,b_3,\,c_1,\dots\)) pueden variar 
			izquierda/derecha respecto a \(x_i\) y definir nuevas clases.
			\item Si no hubiera aristas entre subárboles rojos, habría \(2^{|U|}\) clases 
			(cada nodo de \(U\) puede ir a la izquierda o derecha). 
			\item Sin embargo, las aristas internas en cada subárbol imponen restricciones.
		\end{itemize}
	\end{frame}
\end{comment}


% ---------------------------------------------------------
\begin{frame}{Subárbol de Nodos No Relacionados: Ejemplo}
	\dificultyLevel{2}
	\small
	Tomemos el subárbol de \(b_1\) del ejemplo. 
	Queremos ver las clases de equivalencia posibles dentro de dicho subárbol:
	%\[	b_1	\;\to\; 	\bigl\{\,1_1,\;1_2\,\bigr\}	\;\to\; \bigl\{\,2_1,\;2_2,\;2_3\,\bigr\}.	\]
	\begin{figure}
		\centering
		\begin{tikzpicture}[scale=0.9, transform shape,
			unrelated/.style={circle, draw=red, thick, minimum size=0.8cm},
			arista/.style={->, thick, >=Stealth},
			wiggly/.style={decorate, decoration={snake, amplitude=.2mm, segment length=2mm}}]
			
			% ---- NODOS ----
			\node[unrelated] (b1) at (0,  0) {$b_1$};
			\node[unrelated] (11) at (-1,-2) {$1_1$};
			\node[unrelated] (12) at (1, -2) {$1_2$};
			
			\node[unrelated] (21) at (-1,-4) {$2_1$};
			\node[unrelated] (22) at (1, -4) {$2_2$};
			\node[unrelated] (23) at (3, -4) {$2_3$};
			
			% ---- ARISTAS ----
			\path[arista, draw=red] (b1) edge (11);
			\path[arista, draw=red] (b1) edge (12);
			\path[arista, draw=red] (11) edge (21);
			\path[arista, draw=red] (12) edge (22);
			\path[arista, draw=red] (12) edge (23);
			
			% ---- LÍNEA DE CORTE ----
			\draw[dashed, thick] (-2, -3.1) -- (4, -3.1);
			
			% ---- ETIQUETAS ----
			\node at (-2.2, -2.5) {\small \textbf{Izquierda}};
			\node at (-2.2, -3.7) {\small \textbf{Derecha}};
		\end{tikzpicture}
		\caption*{Clase del subárbol. En este caso la clase sería $L(\equivalenceClass) = \set{b_1, 1_1, 1_2}, R(\equivalenceClass) = \set{2_1, 2_2, 2_3}$.}
		\label{fig:unrelatedSubtree}
	\end{figure}
\end{frame}

\begin{comment}
	\begin{frame}{Intuición de la fórmula \numEqCl}
		\dificultyLevel{3}
		\begin{itemize}[<+- | alert@+>]
			\item Si \(b_1\) está a la derecha de \(x_i\), todos sus descendientes 
			\(1_1,1_2,2_1,2_2,2_3\) \textbf{caerán a la derecha}. Eso define \emph{una} clase.
			\item Si \(b_1\) está a la izquierda, sus hijos pueden \textbf{repartirse 
				libremente}.
			\item A partir de esto se puede \textbf{definir recursivamente} el cálculo de \(\numEqCl\).
		\end{itemize}
	\end{frame}
\end{comment}

\begin{comment}
	% ---------------------------------------------------------
	\begin{frame}{Fórmula Recursiva para \(\numEqCl(n)\)}
		\dificultyLevel{3}
		\small
		\begin{mylemma}[Número de clases de equivalencia]\label{formula:number_of_equiv_classes}
			Sea \(n\) un nodo de un subárbol de nodos no relacionados respecto a \(x_i\). 
			Entonces:
			\[
			\numEqCl(n) 
			=
			\begin{cases}
				2, 
				&\text{si \(n\) es hoja};\\
				\displaystyle
				\Bigl(\prod_{h \in hijos(n)} \numEqCl(h)\Bigr)
				\;+\; 1, 
				&\text{de lo contrario}.
			\end{cases}
			\]
			\begin{itemize}
				\item Si \(n\) es hoja, puede ir a la izquierda o derecha de \(x_i\) 
				(\(2\) clases).
				\item Si no es hoja, al estar a la derecha toda su rama cae a la derecha 
				(\(1\) clase), y si está a la izquierda, cada hijo \(c\) genera 
				\(\numEqCl(c)\) posibilidades, combinables.
			\end{itemize}
		\end{mylemma}
	\end{frame}
	
	% ---------------------------------------------------------
	\begin{frame}{Combinando Subárboles No Relacionados}
		\dificultyLevel{2}
		\small
		Para contar las clases de equivalencia \emph{globales} de todos los 
		nodos no relacionados \(U\), introducimos un nodo virtual \(r_0\) que 
		conectamos únicamente a cada nodo de $UR$. 
		Luego:
		\[
		\numEqCl(r_0) 
		= 
		\prod_{\substack{\text{subárboles}\\ur \in UR}} 
		\numEqCl(ur) 
		\;+\; 1.
		\]
		
		\medskip
		Así, \(\numEqCl(r_0)\) cuenta todas las clases de equivalencia de nuestro grafo $G$.
	\end{frame}
\end{comment}

% ---------------------------------------------------------
\begin{frame}{Cota Superior para \(\numEqCl(r)\) en \dtrees}
	\dificultyLevel{3}
	\small
	Queremos una cota teórica para el número de clases de equivalencia 
	en un \dtree.
	
	\begin{lemma}[Cota superior de clases de equivalencia]\label{lemma:upper_bound_equivalence_classes}
		Sea \(T\) un árbol con raíz \(r\), número de hojas \(l\), número de nodos $n$ y altura \(h\) se cumple que:
		\[
		\numEqCl(r) \;\le\; h^{\,l} \;+\; 1.
		\]
	\end{lemma}
	\pause 
	
	\medskip
	\textbf{Observaciones:}
	\begin{itemize}[<+->]
		\item A menor número de hojas \(l\), menor cota para \(\numEqCl(r)\).
		\item La cota de las clases depende de \(h\) y \(l\), mientras que 
		la de los órdenes topológicos depende de \(n\) (\(O(n!)\)).
		
	\end{itemize}
\end{frame}

\begin{frame}{Spoiler : It works!}
	\dificultyLevel{1}
	\begin{figure}
		\centering
		\includegraphics[width=1\linewidth]{pic/img/equivalentClassesVsToposorts.png}
	\end{figure}
\end{frame}



\subsection{Algoritmo para las clases}

\begin{comment}
Algoritmo exacto para clases de equivalencias en “árboles”
    Forma Naive
    Cálculo de clases de equivalencia para cada unrelated tree
    Combinar clases de los unrelated tree con ancestros y descendientes
        Combinación con descendientes
        Combinación con ancestros
    Complejidad total
        Complejidad de #UnrEC
        Complejidad de #eqClass
        Complejidad total del algoritmo
\end{comment}

Nuestro objetivo es computar $\assym_{M,e,\Pr}(x_i)$ dado un grafo causal $G$ (que es un DAG). Recordemos que:

$$\assym_{M,e,\Pr}(x_i) = \heuristicASVFormula$$

%\santi{Agregar paréntesis a los primeros términos de la sumatoria.}

Una forma de obtener este valor es calculando primero el conjunto de clases de equivalencia $eqCl(G, x_i)$, para luego tomar un representante de cada una (es decir, un orden topológico), con el cual evaluar la expresión dentro de la sumatoria. 
%\santi{Usar referencias a Figuras y no decir ``a continuación'' o ``En el grafo de abajo''.}

\subsection{Solución Naive} \label{alg:naiveAlgorithmEquivalenceClass}

%\santi{Capaz esto puede ir antes del dibujo anterior y la descripción, como una forma sencilla de calcular $eqCl(G, x_i)$. Después proponés la idea de calcular las clases de equivalencia mediante combinaciones de subclases de equivalencia.}

La manera más sencilla de obtener las clases y sus representantes consiste en: calcular todos los órdenes topológicos del bosque utilizando algún algoritmo de generación de los mismos \cite{KNUTH1974153}. Luego, iterar sobre los ordenes topológicos y asignarles a una clase de equivalencia, en base a cuáles nodos no relacionados están antes de $x_i$. Una vez hecho esto, tendremos nuestras clases, un representante para cada una de ellas y sus tamaños. El problema de este algoritmo es que necesitamos calcular explícitamente cada orden topológico de $G$, lo cual puede ser $O(n!)$ en el peor caso, ya que esa es la cota para todos los ordenes topológicos posibles.

\subsection{Algoritmo recursivo}

%\santi{Es raro este subtítulo porque hace parecer que el algoritmo Naive no es exacto.}

En este trabajamos proponemos un algoritmo recursivo para calcular las clases de equivalencia sin calcular explícitamente todos los órdenes topológicos. El algoritmo para encontrar este conjunto de clases se divide en dos partes: en la primera obtenemos las clases de equivalencia para los árboles no relacionados (es decir, aquellos subárboles que contienen nodos que no son ni descendientes ni ancestros de $x_i$), y en la segunda fusionamos estas clases con los ancestros y descendientes. En la Figura~\ref{fig:ASV_forest_example} podemos ver como  quedan etiquetados los distintos nodos:

\begin{figure}[H]
    \centering
    \begin{tikzpicture}[scale=.65, transform shape, 
    unrelated/.style={circle, draw=red},
    ancestor/.style={circle, draw=blue},
    wiggly/.style={decorate, decoration={snake, amplitude=.2mm, segment length=2mm}}  % Define wiggly line style
    ]
        \node[text=blue] at (-5, 0) {Ancestors};
        \node[text=teal] at (-5, -0.5) {Descendants};
        \node[text=red] at (-5, -1) {Unrelated};        
        
        % ---- NODOS ----

        \node[ancestor] (a1) at (0, 0) {$a_1$};
        \node[unrelated] (u1) at (-1, -2) {$u_1$};
        \node[ancestor] (a2) at (1, -2) {$a_2$};

        \drawUnrelatedTree{u2}{-1}{-4}{$u_2$}
        \node[ancestor] (a3) at (1, -4) {$a_3$};
        \node[unrelated] (u3) at (3, -4) {$u_3$};

        \drawUnrelatedTree{u4}{0}{-7}{$u_4$}
        \node[ancestor] (a4) at (3, -6) {$a_4$};

        \node[nodo] (xi) at (3, -8) {$x_i$};

        \drawUnrelatedTree{r1}{6}{0}{$u_5$} 
        \drawUnrelatedTree{r2}{10}{0}{$u_6$} 
        
        \node[draw=none, fill=none] (hi) at (3, -10) {};


         \path [->] (a1) edge[arista]  (u1);
         \path [->] (a1) edge[arista]  (a2);

         \path [->] (a2) edge[arista]  (u2);
         \path [->] (a2) edge[arista]  (a3);
         \path [->] (a2) edge[arista]  (u3);

         \path [->] (a3) edge[arista]  (u4);
         \path [->] (a3) edge[arista]  (a4);

         \path [->] (a4) edge[arista]  (xi);      
         \path [->, teal] (xi) edge[arista,  decorate, decoration={snake, amplitude=.4mm, segment length=4mm, post length=1mm}] node[right] {descendants of $x_i$} (hi);
    \end{tikzpicture}
    \caption{Ejemplo de un grafo causal $G$, con sus respectivas etiquetas en base a $x_i$}
    \label{fig:ASV_forest_example}
\end{figure}

% \santi{Hace falta este dibujo? No podemos linkear al anterior que es idéntico?} Rta: Para mi hace falta porque tiene los nodos bien etiquetados y esta completo, es la referencia de  se etiquetan. 

\subsubsection{Clases de equivalencia para árboles unrelated}

Sea $UR$ (unrelated roots) el conjunto de nodos que son raíces de un árbol no relacionado. Formalmente, $ur \in UR$ sii $ur$ es una raíz y un nodo no relacionado, o el padre de $ur$ es un ancestro de $x_i$ y $ur$ no es un ancestro de $x_i$. El algoritmo se ejecutará sobre cada una de estas raíces para obtener las clases de equivalencia de estos subárboles. Cada clase de equivalencia puede representarse con un conjunto $\equivalenceClassRep$, como vimos en la Definición \ref{equivalenceClassDefinition}. 

%\santi{No se entiende lo de ``si es una raiz''}
%\santi{No entiendo a qué te referís con representación. Creo que es más adecuado decir que una clase de equivalencia es una función $f$ de los nodos unrelated a $\{left, right\}$.} 

%\echu{Definir esto antes, no puede ser que esto recién aparezca acá. Esto es la forma de identificar  dos órdenes pertenecen a la misma clase de equivalencia}

Nuestro algoritmo devolverá un conjunto de tuplas con el formato $(\equivalenceClassRep, leftTopos, rightTopos)$, donde $leftTopos$ es el número de órdenes topológicos que podemos generar con los nodos no relacionados previos a $x_i$, y $rightTopos$ lo mismo, pero con los que están después. El tamaño de la clase $\equivalenceClass$ se puede calcular mediante esta tupla como $leftTopos * rightTopos$.\\

%\santi{En el fondo es medio confuso que usas la misma notación $[\pi]R$ para denotar clases de equivalencia y también para su ``representación''.}

Para la fórmula a continuación, dado un nodo $n \in V$ notamos como $n_i$ al $i-$ésimo hijo de $n$, y como $|n|$ a su número de hijos. La Ecuación\footnote{Pueden encontrar este algoritmo en \path{\pasantia-BICC\asvFormula\classesSizes\recursiveFormula.py}} \ref{formula:unrelated_equiv_classes} calcula todas las clases de equivalencia posibles en un árbol no relacionado, dada la raíz del mismo. Aplicándola sobre cada nodo $ur \in UR$, obtendremos el conjunto de clases de equivalencia para cada subárbol no relacionado con $x_i$. La función $\unrEqCl(n)$ consiste en obtener todas las clases de equivalencias de los hijos del nodo $n$, para luego unificar cada combinación posible de las mismas. El caso base es cuando $n$ es una hoja, ahí sólo hay dos opciones: $n$ puede estar a la derecha o a la izquierda, y va a tener un sólo orden topológico. 

\begin{align}\label{formula:unrelated_equiv_classes}
    \unrEqCl(n) = 
    \begin{cases} 
    \set{(\set{n_l}, 1, 1), (\set{n_r},1,1)} & \text{si $n$ es una hoja} \\[1ex]
    \begin{aligned}
    &\left( \bigcup_{\forall mix \in \unrEqCl(n_1) \times \dots \times \unrEqCl(n_{|n|})} \hspace{-6em} \union(mix,n_{left})\right) \\
    &\cup \union(right,n_{right})
    \end{aligned} & \text{cc}
    \end{cases}
\end{align}

%\santi{Acá hay un tema de notación, o estoy entendiendo todo mal: $\union$ toma un nodo y una tupla de la forma $(\equivalenceClass, L(\equivalenceClass), R(\equivalenceClass))$ por cada hijo de $n$. Con esta info, hay una \textbf{única} clase de equivalencia que podés armar en la que $n$ esté a la izquierda de $x_i$ y se cumplan las clases de cada uno de los hijos. Si es asi, el problema notacional viene de que vos estás escribiendo $eqCl_j$ para denotar estas clases de los hijos, cuando $eqCl$ denotaba el conjunto de clases de equivalencia. Si estoy en lo cierto, estaría bueno declarar las aridades de las funciones para que sea más fácil entenderlas. El \union también está flojo de papeles. Capaz podemos escribir todo lindo en la próxima reunión. Además estaría bueno agregarle una explicación/semántica a la función (tipo para n que hace), y explicar que devuelve. } Rta: Ahí modifique el texto y la función. 

%\santi{Me parece que la mejor forma de escribir esto sería simplemente contar la relación recursiva de los conjuntos y listo, mandamos al apéndice esta función. Tratemos de escribirlo bien la próxima reunión. También, tratemos de elegir mejores nombres, $UnrEC$ es muy largo.}
La función \textit{union} nos sirve para combinar las distintas clases de equivalencia de cada subárbol, para una explicación más detallada de la misma pueden ir al apéndice a la sección \ref{subsubSection:auxiliaryFormulasUnrelatedTrees}. Para tener una intuición respecto a la correctitud de la función, se pude observar que hay una correspondencia directa entre las clases de equivalencia de un nodo y las de sus hijos. Dado un nodo $n$, cada clase de equivalencia que lo incluye puede descomponerse de manera única en una combinación de clases de equivalencia de sus hijos. Análogamente, si conocemos todas las clases de equivalencia posibles de los hijos de $n$, podemos combinarlas para reconstruir todas las clases de equivalencia de $n$. Esta biyección es la que permite realizar el proceso recursivo definido en la ecuación \ref{formula:unrelated_equiv_classes}.

Por fuera de la unión, tenemos una clase de equivalencia más, la cuál tiene al nodo $n$ a la derecha de $x_i$. Utilizamos $right$ pues es la única unión de clases de equivalencia en la cual todos sus nodos aparecen después de $x_i$, y si $n$ está a la derecha, entonces todos sus descendientes deben estar a la derecha también. Observar que $right$ pertenece al mismo producto cartesiano generado en la unión. %\santi{No se a que se refiera la última oración, la sacaría.} Rta: Se explico en persona
%\echu{Falta esto: "Agregar un parrafito explicando que hay una biyección entre cómo podes -> a través de una clase del padre inferir la de los hijos y cómo a través de las de los hijos podes mezclarlas y obtener la de los padres. " pero no sabría donde meterlo}
%Así se define $right$: \\  $right \in \unrEqCl(n_1) \times \dots \times \unrEqCl(n_{|n|}), \forall (repEC, \_ , \_) \in right, L(repEC) = 0$ \\ Lo saco y no lo pongo porque no tiene mucho sentido
%\santi{Ok, hoy lo escribimos. Quedó medio suelto el $right$, no se cual es tu objetivo.} Lo flete, no pega mucho con la onda de la sección

\paragraph{Optimizaciones}\label{slight_improvement}

Más adelante en este algoritmo vamos a tener que combinar cada uno de nuestros árboles no relacionados con los ancestros, esa parte del proceso es la más costosa en tiempo. Por lo tanto, nuestro objetivo es terminar con la menor cantidad de subárboles para combinar. En base a esta idea, encontramos esta optimización. Imaginemos que tenemos los resultados para $\unrEqCl(ur_1)$ y $\unrEqCl(ur_2)$. Si $ur_1$ y $ur_2$ tienen el mismo padre o ambos son raíces, entonces los vamos a unificar. El procedimiento es análogo a agregar un nodo virtual $v$ en la Figura \ref{fig:ASV_forest_example}, que es el padre de $u_5$ y $u_6$, y luego ejecutamos $\unrEqCl$ desde este nodo, utilizando únicamente el $\union$ (ya que este nodo no existe y nunca podría estar a la derecha de $x_i$). De esta forma sólo vamos a tener cómo máximo un árbol no relacionado por cada ancestro de $x_i$. Otra forma más visual de interpretar esta optimización es que en vez de correr nuestro algoritmo sobre cada $ur$, lo vamos a ejecutar sobre cada ancestro $a$. Teniendo el cuidado de no recorrer, el eje que conecta al ancestro $a$ a $x_i$.
%\santi{No entendí cuál es la optimización expliacada acá.} 
%\echu{¿Ahora se entiende más la idea?}
%\santi{Si. Pero, ¿No es equivalente decir que se ejecuta el algoritmo desde cada ancestro? Teniendo el cuidado de no recorrer, al ejecutarlo sobre un cierto ancestro $a$, el eje que lo acerca a $x_i$ (o sea, el eje que va a otro ancestro). }

\begin{comment}
    Al pedo esta figura, igual que la anterior
    
\begin{figure}[H]
    \centering
    \begin{tikzpicture}[scale=.65, transform shape, 
    unrelated/.style={circle, draw=red},
    ancestor/.style={circle, draw=blue},
    wiggly/.style={decorate, decoration={snake, amplitude=.2mm, segment length=2mm}}  % Define wiggly line style
    ]

        \node[ancestor] (a1) at (0, 0) {$a_1$};
        \node[unrelated] (u1) at (-1, -2) {$u_1$};
        \node[ancestor] (a2) at (1, -2) {$a_2$};

        \drawUnrelatedTree{u2}{-1}{-4}{$u_2$}
        \node[unrelated] (u3) at (3, -4) {$u_3$};

        \drawUnrelatedTree{u4}{0}{-7}{$u_4$}
        \node[ancestor] (a4) at (3, -6) {$a_3$};

        \node[nodo] (xi) at (3, -8) {$x_i$};

        \drawUnrelatedTree{r1}{6}{0}{$u_5$} 
        \drawUnrelatedTree{r2}{10}{0}{$u_6$} 
        
        \node[draw=none, fill=none] (hi) at (3, -10) {};


         \path [->] (a1) edge[arista]  (u1);
         \path [->] (a1) edge[arista]  (a2);

         \path [->] (a2) edge[arista]  (u2);
         \path [->] (a2) edge[arista]  (u3);

         \path [->] (a2) edge[arista]  (u4);
         \path [->] (a2) edge[arista]  (a4);

         \path [->] (a4) edge[arista]  (xi);      
         \path [->, teal] (xi) edge[arista,  decorate, decoration={snake, amplitude=.4mm, segment length=4mm, post length=1mm}] node[right] {descendants of $x_i$} (hi);
    \end{tikzpicture}
    \caption{En este caso, los nodos que unificaríamos con esta optimización son $u5$ y $u6$ pues ambos son raíces, y $u2$ y $u4$ pues comparten un padre.}
    \label{fig:equivClass_optimization_example}
\end{figure}

\end{comment}

\subsubsection{Fusión de clases de unrelated trees con ancestros y descendientes}

Ahora nuestro objetivo es combinar las clases obtenidas previamente con los ancestros y descendientes de $x_i$. ¿Por qué no podemos usar la fórmula anterior y simplemente combinarlas como hacíamos antes? Imaginemos que lo hacemos y usamos directamente $\union$ con los ancestros. La tupla que los representa sería  $(\set{ a_l | a \in A}, 1, 1)$, $A$ = $ancestros$. Esto se debe a que solo tienen un orden posible, y todos ellos deben aparecer antes de $x_i$. Ahora bien, si usáramos $\union$, estaríamos haciendo el cálculo $\binom{|A| + |U|}{|A|}$, lo que significa que podríamos insertar los elementos de $A$ entre cualquier elemento de $U$ (nodos no relacionados) en el orden topológico, ¡pero eso es incorrecto! Porque en este escenario, hay dependencias entre los nodos. En la Figura \ref{fig:ASV_forest_example} no podemos poner $u_1$ antes de $a_1$ o ninguno de los nodos en el subárbol de $u_2$ antes de $a_2$. Esto significa que necesitamos otra función para calcular los órdenes posibles de los nodos que aparecen antes de $x_i$ y los ancestros. Vamos a llamar a esta función $leftOrders$. \\

Ahora la pregunta es, ¿qué hacemos con los descendientes? Aquí no estamos restringidos como lo estábamos antes con los ancestros, por lo que podemos usar una fórmula similar a la del algoritmo \ref{for:topoCountingDTrees}, porque no hay dependencias entre los nodos de $D$ y $UR$. Teniendo esto en cuenta, finalmente podemos definir nuestra fórmula\footnote{Pueden encontrar este algoritmo en \path{\pasantia-BICC\asvFormula\classesSizes\recursiveFormula.py}} \ref{formula:equiv_classes_sizes}. Dado un \dtree{} $G$, un nodo $x_i \in V(G)$, $ur_i$ la raíz del $i-$ésimo árbol no relacionado, $A$ y $D$ los ancestros y descendientes, respectivamente, de $x_i$ en $G$. Utilizando las funciones $eqCl$ y $\eqClassSize$, las cuáles nos devuelven la representación de la clase $\equivalenceClassRep$ y el tamaño de la clase, respectivamente. Con estas funciones auxiliares, las cuáles están definidas en el apéndice en la sección \ref{subsubSection:auxiliaryFormulasEquivalenceClasses}, la fórmula para calcular todas las clases de equivalencia y sus tamaños en $G$ es:

%\echu{Pasar esta formula a una definición y aclarar todas las variables y letras que aparecen + especificar la notación falopa}

\begin{align}\label{formula:equiv_classes_sizes}
    \eqClassSizes(G, x_i) = 
    \bigcup_{\forall mix \in \unrEqCl(ur_1) \times \dots \times \unrEqCl(ur_{|UR|})} \left( eqCl(A,D, mix) , \eqClassSize(A,D,mix) \right) 
\end{align}

%\santi{No diría que esto es una definición.} Concuerdo

\subsubsection{Combinando los ancestros y los nodos no relacionados}

Estas son las variables que vamos a tomar en cuenta para calcular las combinaciones: los $ancestros$, el valor de $L(eqCl)$ (la cantidad de nodos a la izquierda en la clase de equivalencia) para cada $eqCl$ de cada unrelated tree, y la raíz de cada árbol no relacionado. Lo que queremos hacer es definir el número de órdenes topológicos que podemos generar, combinando los nodos a la izquierda de cada unrelated tree con los ancestros. Veamos un ejemplo con el grafo de la Figura \ref{fig:ASV_forest_example}. ¿Qué ordenes topológicos podemos generar con los ancestros posicionados como vemos en la Figura \ref{fig:order_of_ancestors}?

\begin{figure}[ht]
    \centering
    \begin{tikzpicture}
        % Draw the main line
        \draw[thick] (0,0) -- (10,0);
        
        % Draw and label the points
        \foreach \x/\label in {2/$a_1$, 4/$a_2$, 6/$a_3$, 8/$a_4$} {
            \pgfmathsetmacro{\xnew}{\x-1}
            \node at (\xnew,0.2) {$\cdots$};
            \draw (\x,0.2) -- (\x,-0.2); % Ticks
            \node[below] at (\x,-0.2) {\label}; % Labels
        };

    \node at (9,0.2) {$\cdots$};
    \end{tikzpicture}
    \caption{Posible orden topológico con los ancestros ya colocados}
    \label{fig:order_of_ancestors}
\end{figure}

Ahora la pregunta es, ¿dónde podemos colocar los elementos de cada unrelated tree? Antes de $a_1$, solo podemos colocar nodos que estén en los subárboles de $u_5$ o $u_6$. Antes de $a_2$, podemos colocar los mismos elementos, más $u_1$. En general, antes de un nodo $a_i$ podremos colocar todos los nodos que estén incluidos en un $unrelated \ tree$ que tenga como raíz $ur$, tal que $ur$ sea una raíz, o $a_j$ sea el padre de $u_r$ con $j<i$. Ahora queremos convertir esto en una fórmula recursiva, ya que habrá una superposición de problemas utilizamos programación dinámica al implementar el algoritmo. La fórmula más detallada la pueden encontrar en el apéndice en la sección \ref{subsubSection:leftOrdersImplementation}, este es el algoritmo para contabilizar las combinaciones posibles entre los nodos no relacionados y los ancestros:\\

\begin{algorithm}
\caption{leftOrders($A$, $\textit{actual ancestor}$, $\textit{nodes to place}$, $position$)} \label{alg:leftOrdersAlgorithm}
\begin{enumerate}
    \item Definimos donde colocar $\textit{actual ancestor}$ en base a $position$ y a cuántos nodos tenemos disponibles en $\textit{nodes to place}$, generando $\textit{new position}$.
    \item Luego seleccionamos cuántos nodos de cada unrelated tree vamos a usar para llenar todas las posiciones entre $position$ y $\textit{new position}$, generando $\textit{new nodes}$.
    \item Eliminamos los $\textit{new nodes}$ de los $\textit{nodes to place}$, puesto que ya los colocamos, actualizando nuestros nodos disponibles.
    \item Realizamos el llamado recursivo actualizando la posición, nuestros nodos disponibles y nuestro ancestro actual. 
\end{enumerate}
\end{algorithm}

\subsection{Complejidad del algoritmo}
%\echu{ ¿Esta es la idea? ¿Cuánto más formal debería ser? ¿Cuánto mejor lo debería escribir? }
Sea $n$ el número de nodos del digrafo causal $G$, $r$ la raíz de $G$ y $equivalenceClasses$ el conjunto de clases de equivalencia en $G$ para el feature $x_i$.

\subsubsection{Complejidad temporal de \unrEqCl}

Para calcular la complejidad de \unrEqCl, definida en la fórmula \ref{formula:unrelated_equiv_classes}, vamos a calcular el costo de cada nodo del árbol de llamadas recursivas y el tamaño de este árbol. La complejidad de $union$ es $O(n)$, su justificación se encuentra en el apéndice en la sección \ref{subsubSection:auxiliaryFormulasUnrelatedTrees}. Luego necesitamos acotar el número de $mix$ que se generará en cada llamada, pero debido a que estamos generando las clases de equivalencia, sabemos que estará acotado \footnote{En el apéndice, en la sección \ref{subsubSection:leftOrdersImplementation}, se puede encontrar una justificación más detallada de esta cota} por $|equivalenceClasses|$. Por lo que la cota para la complejidad de calcular esta función en cada estado es $O(n^2 * |equivalenceClasses|)$. Luego esta función se ejecuta una sola vez por cada nodo, por lo que se llama $O(n)$ veces, con lo cual la complejidad temporal total es $O(n^3 * |equivalenceClasses|)$.

%\echu{ Es una cota poco fina, porque sólo en el último nodo mix va a tener tamaño |equivalenceClasses| y además podría usar $d_out(node)$ en vez de n, cota para cada nodo. Creería que no hace falta porque la complejidad posta es la de la segunda parte. }
%\santi{Para mí quedó bien. Capaz aclararía algo sobre la cuenta de clases de equivalencia.} Rta: Esa cota esta explicada en el apéndice. 

%\echu{ Esta bien este formato? O intento hacer esto cómo una "demo" o al menos poner un lema y cómo quiero mostrar que la complejidad del algoritmo es tal. }
\subsubsection{Complejidad temporal total}

 La complejidad de \eqClassSizes \ es  $O(\unrEqCl) + |equivalenceClasses| * O(\eqClassSize)$. Ya que primero, necesitamos ejecutar $\unrEqCl$ para obtener las clases de equivalencia que vamos a fusionar. Luego, para cada clase de equivalencia que creemos, necesitamos calcular su tamaño y sus elementos. Para  $\eqClassSize$ sabemos que su complejidad temporal es de $O(n^5 * |equivalenceClasses|^2)$, en la sección \ref{subsubSection:auxiliaryFormulasEquivalenceClasses} del apéndice se encuentra la justificación de esta complejidad. Por lo tanto, la complejidad total del algoritmo completo será $O(n^3 * |equivalenceClasses|)$ + $O(n^5 * |equivalenceClasses|^3)$ = $O(n^5 * |equivalenceClasses|^3)$.







\section{ASV end to end}


\subsection{ASV exacto}

Recordemos la fórmula introducida en la sección \ref{Section:HeuristicaASV} para ASV, utilizando nuestra heurística.

\begin{align*}
   \phi_{i}^{assym}(\charactheristicFunction) = \sum_{\pi \in \perm(\players)} w(\pi) \left[ \charactheristicFunction(\pi_{<i} \cup {i}) - \charactheristicFunction(\pi_{<i}) \right] &= \\
   \heuristicASVFormula
\end{align*}

Dado un DAG $G$, un nodo $x_i \in V(G)$ y una función característica $\charactheristicFunction$. Nuestro algoritmo completo queda así entonces: 

\begin{itemize}
    \item Calculamos $eqClass$ a través de $\eqClassSizes(G,x_i)$, el algoritmo que introducimos en la sección~ \ref{Section:AlgoritmoEquivClasses}
    \item Luego para cada clase de equivalencia vamos a calcular el promedio teniendo en cuenta los features previos a $x_i$. Para realizar el promedio vamos a utilizar el algoritmo de la sección \ref{Section:RedesBayesianas}. (En el caso de que el modelo no sea un árbol, podemos aproximarlo utilizando Monte Carlo, pero el algoritmo dejaría de ser exacto )
    \item Por último sumamos los resultados para obtener él $\assym$ para el feature $i$. 
\end{itemize}

Las modificaciones que introducimos son para mejorar la performance del mismo, sin modificar los resultados obtenidos a diferencia del enfoque aproximado. En la sección a continuación vamos a ver cuál es la mejora respecto al enfoque naive y el tiempo que tarda para distintas redes.  

\subsection{ASV aproximado}

Este algoritmo es igual al anterior. La única diferencia que tiene es respecto a cómo se calculan las clases de equivalencia, $eqClass(G, x_i)$, puesto que ahora las vamos a aproximar. Además, vamos a tener que agregar un parámetro extra, para definir la cantidad de órdenes topológicos que queremos generar. 
Primero utilizamos el algoritmo \ref{alg:topoSortSampling} para samplear los órdenes topológicos. Esto nos va a permitir calcular él $ASV$ para polytrees y no simplemente \dtrees. Luego procesamos estos órdenes al igual que en la sección \ref{alg:naiveAlgorithmEquivalenceClass}, para obtener las clases de equivalencia. Una vez obtenidas las clases de equivalencia, repetimos el mismo proceso del algoritmo exacto. 


\section{Sampleo} % en polytrees}

% --- Sección: Muestreo aproximado de órdenes topológicos ---
\begin{frame}{Sampleo aproximado de órdenes topológicos}
	\dificultyLevel{2}
	En esta sección presentamos un algoritmo probabilístico para \textbf{muestrear órdenes topológicos} de un DAG causal $G$.
	\begin{itemize}[<+- | alert@+>]
		\item A través de este muestreo vamos a poder \textbf{aproximar $ASV$} con precisión creciente según el número de muestras.
		\item La cantidad de muestras necesaria \textbf{crece lentamente} con la precisión deseada.
		\item Nuestro objetivo inicial: \textbf{devolver un orden topológico aleatorio}.
	\end{itemize}
\end{frame}

\begin{frame}{Algoritmo de sampleo}
	\dificultyLevel{3}
		\begin{figure}[ht]
		\centering
		\begin{tikzpicture}
			% Define the right set of nodes
			\foreach \i in {1,2,3,4,5}
			\node[draw=none, circle, minimum size=5mm, inner sep=0pt] (L\i) at (4, -\i) {};
			
			% Define the left set of nodes
			\foreach \j in {1,2,3}
			\node[draw, circle, red, minimum size=5mm, inner sep=0pt] (source\j) at (0, -\j*2) {\scalebox{0.6}{$Source_\j$}};
			
			% Draw edges between nodes (example edges)
			\foreach \i in {1,2}
			\foreach \j in {1,2}
			\draw[->]  (source\j) -- (L\i); 
			
			\foreach \i in {4,5}
			\foreach \j in {2,3}
			\draw[->]  (source\j) -- (L\i); 
			
			\draw[->]  (source2) -- (L3);
			
			\draw [decorate, blue, decoration={random steps, segment length=10pt, amplitude=2pt}, thick]
			(4,-3) circle (2.4);
			
			%Number of órdenes topológicos for each node
			
			\node[draw=none,minimum size=3mm, inner sep=0pt] () at (0, -1) {\small \textcolor{orange}{$n_1$=100}};
			
			\node[draw=none,minimum size=3mm, inner sep=0pt] () at (0, -3) {\small \textcolor{orange}{$n_2$=300}};
			
			\node[draw=none,minimum size=3mm, inner sep=0pt] () at (0, -5) {\small \textcolor{orange}{$n_3$=200}};
		\end{tikzpicture}
		%\caption{Posible comienzo del algoritmo de sampleo, con los candidatos a ser el primer nodo del orden en rojo y el resto del grafo en azul. Cada node fuente (source) tiene sus respectivas cantidades de órdenes topológicos en los que está primero.}
		\label{fig:topoSortSamplingExample}
	\end{figure}
	\pause
	$p(source_1)=\frac{1}{6}, p(source_2)=\frac{1}{2}, p(source_2)=\frac{1}{3}$
\end{frame}

\begin{comment}
	Ejemplo de polytree:
	Los descendientes en común y disjuntos de cada raíz están marcados de un color distinto. Acá hay intersección entre los descendientes por lo que no se los puede mezclar libremente. Por eso tenemos que encontrar otro orden para recorrer el grafo. 
	Conteo de órdenes:
	DFS (sobre el grafo subyacente) enraizado en \(r_i\): los nodos grises ya fueron visitados; los azules está en proceso y los blancos no fueron procesados todavía.
	Calculamos recursivamente para cada componente conexa (del grafo subyacente) no visitada cuáles son sus órdenes topológicos. Necesitamos saber posicion del nodo, tamaño del subárbol y órdenes totales para combinar estos resultados. 
	Luego combinamos los resultados utilizando una función muy similar a la utilizada en el algoritmo recursivo de las clases en la cuál fusionamos ancestros y unrelated. 
	Mencionar que procesamos a los padres e hijos a la vez, por lo que a la hora de fusionarlos tenemos que tener en cuenta esto, lo cuál no es tan sencillo. 
	
	\begin{frame}{Expandiendo el conteo de órdenes}
		\dificultyLevel{2}
		\begin{figure}[ht]
			\centering 
			\begin{tikzpicture}[scale=.9, transform shape]
				
				% ---- NODOS ----
				\node[nodo, red] (r1) at (-2,0) {$r_1$};
				\node[nodo, green] (r2) at  (2,0) {$r_2$};
				\node[nodo, orange] (r3) at  (4,0) {$r_3$};
				
				\node[nodo, red] (s1) at (-2,-2) {$s_1$};
				\node[nodo, blue] (s2) at (0,-2) {$s_2$};
				\node[nodo, green] (s3) at (2,-2) {$s_3$};
				\node[nodo, orange] (s4) at (4,-2) {$s_4$};
				
				
				\node[draw=none, fill=none] (h1) at (-2, -4) {};
				\node[draw=none, fill=none] (h1Parent) at (-2, 2) {};
				\node[draw=none, fill=none] (h2) at (0, -4) {};
				\node[draw=none, fill=none] (h3) at (2, -4) {};
				\node[draw=none, fill=none] (h3Parent) at (2, 2) {};
				\node[draw=none, fill=none] (h4) at (4, -4) {};
				\node[draw=none, fill=none] (h4Parent) at (4, 2) {};
				
				
				\path [->] (r1) edge[arista, red]  (s1);
				\path [->] (r1) edge[arista]  (s2);
				\path [->] (r2) edge[arista]  (s2);
				\path [->] (r2) edge[arista, green]  (s3);
				\path [->] (r3) edge[arista]  (s3);
				\path [->] (r3) edge[arista, orange]  (s4);
				
				\path [->] (s1) edge[arista, red, mySnake] (h1);
				\path [->] (h1Parent) edge[arista, red, mySnake] (r1);
				\path [->] (s2) edge[arista, blue,  mySnake]  (h2);
				\path [->] (s3) edge[arista, green,  mySnake]  (h3);
				\path [->] (h3Parent) edge[arista, green,  mySnake]  (r2);
				\path [->] (s4) edge[arista, orange,  mySnake]  (h4);
				\path [->] (h4Parent) edge[arista, orange,  mySnake]  (r3);
				
			\end{tikzpicture}
			\caption{Polytree para el cual no funciona el algoritmo original de conteo.}
			\label{fig:polytreeMultipleIntersections}
		\end{figure}
	\end{frame}
	
	\begin{frame}{Algoritmo de Conteo}
		\dificultyLevel{3}
		\begin{figure}
			\centering
			\begin{tikzpicture}[scale=.7, transform shape]
				
				% ---- ESTILOS ----
				\tikzstyle{nodo}      =[circle, draw, minimum size=18pt, font=\small] % estilo base
				\tikzstyle{visited}   =[nodo, fill=gray!30]   % ya visitado
				\tikzstyle{visiting}  =[nodo, fill=blue!20]   % en visita
				\tikzstyle{arista}    =[->, >=stealth]        % aristas
				\tikzstyle{mySnake}   =[decorate, decoration={snake, amplitude=0.7pt}] % para líneas onduladas
				
				% ---- NODOS ----
				\node[visited]  (a1)  at (0,   0) {$a_1$};
				\node[visited]  (a2)  at (2.5,  2) {$a_2$};
				\node[visited]     (a3)  at (2.5, -2) {$a_3$};
				\node[visiting] (xi)  at (5,   0) {$r_i$};
				\node[nodo]  (d_1) at (7.5, 2) {$d_1$};
				\node[visiting]     (d_2) at (10, -2) {$d_2$};
				\node[nodo]     (d_3) at (10,  0) {$d_3$};
				
				% nodos “fantasma” para las líneas onduladas
				\node[draw=none, fill=none] (hijo_a3) at (4.5, -4) {};
				\node[draw=none, fill=none] (hijo_d2) at (8,   -4) {};
				
				% ---- ARISTAS ----
				\path (a1) edge[arista] (xi);
				\path (a2) edge[arista] (xi);
				\path (a3) edge[arista] (xi);
				
				\path (xi) edge[arista] (d_1);
				\path (xi) edge[arista] (d_2);
				\path (xi) edge[arista] (d_3);
				
				\path (a3)      edge[arista, mySnake] node[above right] {} (hijo_a3);
				\path (hijo_d2) edge[arista, mySnake] node[below right] {} (d_2);
			\end{tikzpicture}
			\caption{DFS sobre el grafo subyaciente enraizado en \(r_i\)}
		\end{figure}
	\end{frame}
\end{comment}

\section{Experimentos}

Habiendo definido nuestros algoritmos, lo que nos queda es ver cuál es la ventaja que otorgan los mismos en comparación a la solución naive de cada uno de estos problemas. Esta ventaja puede ser en precisión, tiempo o espacio. 

Para los experimentos utilizamos dos redes bayesianas, $\cancerNetwork$ y $\childNetwork$, tomadas del paquete de R \href{https://www.bnlearn.com/bnrepository/}{bnlearn}, podemos ver su estructura en la Figura \ref{fig:bayesian_networks_combined}. Utilizamos estas redes, ya que tenían un tamaño manejable y una topología similar a la de un polytree. Además de ser la distribución de nuestros datos, las empleamos como nuestros grafos causales, asumiendo que fueron generadas basándose en la causalidad \footnote{Esto no necesariamente siempre se cumple, pero entendemos que es una suposición razonable para la mayoría de los casos.}.
 En el caso de la red \childNetwork, al no ser un polytree tuvimos que remover algunas aristas para lograr obtener esa estructura. Analizamos distintas heurísticas para remover ejes de la red convirtiéndola en un polytree. Utilizamos la más sencilla, que consiste en remover los ejes que generan ciclos en el grafo subyacente de la red bayesiana. Una opción más refinada consiste en elegir las aristas a remover, en base a cuáles minimizan la divergencia entre las distribuciones marginalizadas. Esto lo podríamos hacer comparando las distintas distribuciones que nos quedan al remover distintas aristas, utilizando una métrica como la divergencia de Kullback-Leibler, pero no era el objetivo de esta tesis. 

Para generar los datasets de entrenamiento y test, sampleamos instancias de ambas redes. Luego entrenamos al DT usando este dataset. A la hora de calcular el ASV, removimos la variable a predecir de nuestra red. Removemos la variable, puesto que si tuviéramos la red bayesiana completa con nuestra variable a predecir en la misma, haríamos la inferencia directamente en la red bayesiana \footnote{Tampoco entrenaríamos un árbol de decisión, sino que simplemente utilizaríamos a la red para clasificar las instancias}. Las variables que definimos para predecir fueron \variableNetwork{Smoker} y \variableNetwork{Age} para sus respectivas redes. En el caso de la red $\cancerNetwork$ no elegimos \variableNetwork{Cancer}, puesto que nuestro DAG causal iba a perder todas sus dependencias, por lo cual ASV no iba a poder detectar relaciones significativas. Luego en el caso de \childNetwork{} el criterio fue no utilizar una variable que se encuentre muy arriba en el árbol, puesto que no iba a tener muchas variables que la influencien, ni tan abajo que cueste distinguir las variables que la impactan mayormente. Por lo que nuestro input va a ser una red bayesiana $\aBayesianNetwork$, un feature a predecir $p$, un dataset $data$, un árbol de decisión $DT$ y una instancia $x \in data$. 

\begin{figure}[ht]
    \centering

    \begin{subfigure}[b]{0.3\textwidth}
        \centering
        \includegraphics[width=\textwidth]{img/bayesianNetworks/cancerNetwork.png}
        \caption{Red Bayesiana $\childNetwork$ para analizar las enfermedades de niños de recién nacidos. Fuente: \cite{cancerNetwork}}
        \label{fig:cancer_network}
    \end{subfigure}
    \hfill
    \begin{subfigure}[b]{0.6\textwidth}
        \centering
        \includegraphics[width=\textwidth]{img/bayesianNetworks/childNetwork.png}
        \caption{Red Bayesiana \cancerNetwork para determinar la probabilidad de tener cáncer de distintos pacientes. Fuente: \cite{childNetwork}}
        \label{fig:child_network}
    \end{subfigure}

    \caption{Ejemplos de redes bayesianas utilizadas en los experimentos.}
    \label{fig:bayesian_networks_combined}
\end{figure}

\paragraph{Especificaciones de los experimentos}

\begin{itemize}
    \item \textbf{Procesador:} Intel(R) Core(TM) i5-7500 CPU @ 3.40GHz
    \item \textbf{Memoria RAM:} 16 GB
    \item \textbf{Sistema operativo:} Ubuntu 22.04 LTS
    \item \textbf{Python:} Versión 3.12
    \item  \textbf{Paquetes:} pgmpy (inferencia bayesiana), sklearn, networkx, shap
\end{itemize}

\subsection{Clases de equivalencia vs Órdenes Topológicos}

Para este experimento, vamos a comparar la implementación clásica de $ASV$ con nuestra idea de utilizar las clases de equivalencia para reducir los términos de la sumatoria. Para eso vamos a comparar la forma original de calcular ASV: 

$$\frac{1}{|topos(G)|} \sum_{\pi \in topos(G)} w(\pi) \left[ \charactheristicFunction(\pi_{<i} \cup {i}) - \charactheristicFunction(\pi_{<i}) \right] $$
con nuestra heurística:
$$\heuristicASVFormula$$

Hay dos métricas a tener en cuenta para ver cuál de estas dos estrategias es mejor. Primero, ver cuánto es el tiempo que se tarda en obtener los conjuntos sobre los que efectuar la sumatoria, que son $eqCl(G, x_i)$ y $topos(G)$. Luego comparar el tamaño de cada uno de esos conjuntos, puesto que por cada elemento de ese conjunto vamos a tener que evaluar a $\charactheristicFunction$ dos veces. Podría ocurrir que la construcción de las clases de equivalencia resulte computacionalmente costosa, y su cardinalidad no necesariamente presente una reducción significativa en comparación con $topos$. En tales casos, el costo adicional de calcularlas puede superar el beneficio esperado, incrementando el tiempo total de cómputo.

\begin{figure}[ht]
    \centering
    \includegraphics[width=1\linewidth]{img/equivalentClassesVsToposorts.png}
     \caption[Caption for image]{Comparación de clases de equivalencias y órdenes topológicos de distintas clases de grafos. Se utiliza un promedio, puesto que la cantidad de clases de equivalencia depende del nodo que elijamos para calcularlas. \footnotemark }
    \label{fig:equivalenceClassesVsToposortsNumberPlot}
\end{figure}

\footnotetext{La función no es monótona, ya que no se utilizaron todos los grafos posibles para cada una de las clases mencionadas, se realizó una estimación a partir de una muestra significativa de distintos grafos de cada clase. }

%\echu{¿Es mejor tener un gráfico para cada familia? A mi me parecía mejor unificarlos} Queda unificado

Las distintas clases de grafos mencionados en la Figura \ref{fig:equivalenceClassesVsToposortsNumberPlot} son: 
\begin{itemize}
    \item Naive Bayes: Una red Naive Bayes con $n$ nodos, que tiene $n/2$ hojas y que tiene un camino de longitud $n/2-1$ en una de sus hojas. %\santi{No dan las cuentas de la cantidad de nodos} \echu{¿Ahí si, no? Me faltaba la raíz}
    \item Multiple paths: Un bosque compuesto de múltiples caminos de igual longitud. 
    \item Balanced tree: Un árbol binario perfecto balanceado.  
    \item Child network: La red bayesiana \childNetwork, sin algunos de sus ejes para ser un polytree. 
    
\end{itemize}

En la Figura \ref{fig:equivalenceClassesVsToposortsNumberPlot}, podemos observar como el número de clases de equivalencia crece significativamente más lento que el número de órdenes topológicos. Por ejemplo, en el caso de la red $\childNetwork$, la red tiene $7.41\times10^{11}$ órdenes topológicos y 2003 clases de equivalencia en promedio, por lo que si utilizamos las clases disminuimos enormemente la cantidad de llamadas a $\charactheristicFunction$. Luego, para los árboles balanceados, la cantidad de órdenes topológicos es $10^{50}$ veces mayor, una diferencia muy significativa. A partir de estos ejemplos, queda claro que es una mejora hacer el cálculo sobre las clases de equivalencia. Ahora solo queda ver el costo de calcularlas. 

El costo de calcular las clases lo podemos ver en la Figura \ref{fig:equivalenceClassesTimePlot}. Para la mayoría de los grafos de ejemplo que utilizamos tarda menos de 10 segundos. Pero para grafos de mayor tamaño, el tiempo que tarda comienza a crecer exponencialmente, al igual que la cantidad de clases de equivalencia. En el caso puntual de la red \childNetwork tarda menos de 1 segundo en calcular todas sus clases. 


\begin{figure}[ht]
    \centering
    \includegraphics[width=1\linewidth]{img/equivalentClasses_time.png}
    \caption[Caption for image]{Comparación del tiempo que tarda el algoritmo para calcular las clases de equivalencia}
    \label{fig:equivalenceClassesTimePlot}
\end{figure}

\begin{figure}[ht]
    \centering
    \includegraphics[width=1\linewidth]{img/equivalentClassesVsAllToposorts_time.png}
    \caption[Caption for image]{Comparación del tiempo que tarda el algoritmo para calcular las clases de equivalencia y calcular todos los órdenes topológicos, utilizando una \href{https://networkx.org/documentation/stable/reference/algorithms/generated/networkx.algorithms.dag.all_topological_sorts.html}{implementación} de la librería networkx. \footnotemark }
    \label{fig:equivalenceClassesVsToposortsTimePlot}
\end{figure}


\footnotetext{Para estimar el tiempo que tardaría se calcularon los primeros 1000 órdenes que devuelve el algoritmo exacto. Luego se utilizó ese tiempo para hacer una aproximación del tiempo total, sabiendo la cantidad de órdenes de cada grafo. Esto se realizó por simplicidad, ya que no era viable correr el algoritmo durante tanto tiempo.}

La Figura \ref{fig:equivalenceClassesVsToposortsTimePlot} nos muestra para distintos tipos de grafos cuánto tiempo toma cada algoritmo. En uno se obtienen todas las clases de equivalencia y en el otro todos los órdenes topológicos. A partir de $10^{15}$ órdenes topológicos el problema de calcularlos todos tardaría días, en cambio, calcular las clases de equivalencia sigue siendo una estrategia eficiente. Esto sucede así, puesto que, como vimos en el lema \ref{lemma:upper_bound_equivalence_classes}, la cantidad de clases de equivalencia depende de la estructura del grafo, y no de la cantidad de órdenes.
%\sergio{revisar esta frase, no tiene sentido que solo a partir de $10^{15}$ no es tratable} Rta: Ahí le puse la referencia al gráfico correspondiente y tuvo más sentido, si no era cualquier cosa. 

Por ende, a través de este experimento pudimos ver que la cantidad de clases de equivalencias es significativamente menor que la cantidad de órdenes, por lo que se reduce la cantidad de llamadas a $\charactheristicFunction$. Además, en la figura \ref{fig:equivalenceClassesVsToposortsTimePlot}, podemos ver cómo el tiempo para calcular las clases de equivalencia crece más lentamente que el tiempo de calcular todos los órdenes. Por lo tanto, podemos concluir que el algoritmo proporciona una mejora del tiempo para calcular todas las clases respecto a la implementación naive.
%\echu{Medio raro lo de doblemente efectiva, tal vez habría que expresarlo de otra forma}
%\sergio{Decir algo de que cambia el slope en la escala logarítmica notablemente}

\subsection{ASV vs SHAP}

%\echu{¿Uso emph, bold o alguna otra notación para las redes y las variables?} Sip, emph para redes y texttt para las variables

Este experimento consiste en calcular el valor del $ASV$ y $SHAP$, para cada uno de los features de los modelos utilizados en el experimento de la sección \ref{subSection:experimentoAlgoritmoPromedio}. Vamos a correr este experimento para 5 seeds distintas, ya que los valores obtenidos pueden depender de la aleatoriedad de los datos y queremos contrastar múltiples resultados. Aun así, vamos a elegir una seed para analizar para cada una de las redes, utilizando como criterio para elegir la seed el modelo con la mejor accuracy, ya que si el modelo no aprendió correctamente los patrones subyacentes de los datos, entonces los valores de $ASV$ y $SHAP$ pueden no correlacionarse con el grafo causal original. Los resultados de todas las corridas pueden verse en el apéndice en las Figuras ~\ref{fig:multipleSeedsASVvsShapleyChild} y \ref{fig:multipleSeedsASVvsShapleyCancer}. Correr el $ASV$ para todos los features de la red $\cancerNetwork$ tarda 1 segundo, y correrlo para todos los features de la red $\childNetwork$ tarda 3 minutos. 

Para obtener una intuición acerca de ambas redes y las distintas relaciones entre sus nodos nos basamos en \cite{childNetwork} para la red $\childNetwork$ y \cite{cancerNetwork} para la red $\cancerNetwork$.Además, generamos una Tabla \ref{tab:phi_smoker_with_shift} para ambas redes, para analizar el impacto de modificar cada una de las variables de la red. Por ejemplo, la probabilidad original de la variable \variableNetwork{Smoker} es $P(Smoker=True)=0.3$ pero si sabemos que el paciente tiene cáncer pasa a ser $P(Smoker=True|Cancer=True)=0.8255$. Por lo que (como era de esperarse), \variableNetwork{Cancer} es una variable significativa a la hora de calcular si un paciente es fumador o no. En este caso, vemos que la probabilidad de que sea fumador se vio modificada en 0.5255 al introducir la evidencia de que tenía \variableNetwork{Cancer}. Esta variación de la probabilidad es la que vamos a ver en la columna \emph{Probability Shift} en la Tabla \ref{tab:phi_smoker_with_shift}.

\begin{table}[ht]
    \centering
    \begin{tabular}{|c|c|c|c|}
        \hline
        \textbf{Variable} & \textbf{Smoker} & \textbf{New smoker probability} & \textbf{Probability Shift} \\
        \hline
        \multirow{2}{*}{Pollution = Low} & True  & 0.3 & \multirow{2}{*}{0.0} \\
                                         & False & 0.7 & \\
        \hline
        \multirow{2}{*}{Pollution = High} & True  & 0.3 & \multirow{2}{*}{0.0} \\
                                          & False & 0.7 & \\
        \hline
        \multirow{2}{*}{Cancer = True} & True  & 0.8255 & \multirow{2}{*}{0.5255} \\
                                       & False & 0.1745 & \\
        \hline
        \multirow{2}{*}{Cancer = False} & True  & 0.2938 & \multirow{2}{*}{0.0062} \\
                                        & False & 0.7062 & \\
        \hline
        \multirow{2}{*}{Xray = Positive} & True  & 0.3206 & \multirow{2}{*}{0.0206} \\
                                         & False & 0.6794 & \\
        \hline
        \multirow{2}{*}{Xray = Negative} & True  & 0.2946 & \multirow{2}{*}{0.0054} \\
                                         & False & 0.7054 & \\
        \hline
        \multirow{2}{*}{Dyspnoea = True} & True  & 0.3070 & \multirow{2}{*}{0.0070} \\
                                         & False & 0.6930 & \\
        \hline
        \multirow{2}{*}{Dyspnoea = False} & True  & 0.2969 & \multirow{2}{*}{0.0031} \\
                                          & False & 0.7031 & \\
        \hline
    \end{tabular}
    \caption{Variación de la probabilidad de ser fumador en base a la nueva evidencia}
    \label{tab:phi_smoker_with_shift}
\end{table}

Comencemos analizando el caso de la red $\cancerNetwork$. En la Figura \ref{fig:shapleyVsASVSingleSeedCancer} podemos ver que para $ASV$ la única variable significativa es \variableNetwork{Cancer}. Esto tiene sentido con lo visto en la Figura \ref{fig:cancer_network}, ya que es la única variable conectada directamente con \variableNetwork{Smoker}. Pero si nos basáramos en los resultados obtenidos en los Shapley Values, creeríamos que \variableNetwork{Xray} y \variableNetwork{Dyspnoea} también tienen un impacto significativo en si es fumador o no el paciente. La forma que tenemos para ver cuál de los métodos está detectando correctamente las variables relevantes es la Tabla \ref{tab:phi_smoker_with_shift}, en la cual podemos ver que la variable que más impacta es \variableNetwork{Cancer} y que la \variableNetwork{Dyspnoea} tiene un impacto mucho menor. Esta nueva métrica que vamos a utilizar es igual de arbitraria que Shap, pero nos permitió realizar una comparación y un análisis cuantitativo más allá del significado de cada una de las variables.
%\santi{Nunca definís como se calcula el probability shift}
%\echu{Esto quedo medio raro, lo que quiero decir es que utilizamos esta métrica para que la justificación meramente no sea "Tiene sentido esto, por lo que significan las variables en el mundo real" ¿Se entiende la idea?}.
%\santi{Se entiende, creo.}
Esto ocurre, ya que $ASV$ tiene en cuenta el grafo causal a la hora de realizar estos cálculos, a diferencia de $SHAP$. 

\begin{figure}
    \centering
    \includegraphics[width=1\linewidth]{img/asvResults/cancerASVAndShapleyExactASVAndShapley.png}
    \caption{Resultados del ASV y Shapley para el modelo con mayor accuracy de los 5 seeds para la red $\cancerNetwork$, con un 81\% de accuracy. }
    \label{fig:shapleyVsASVSingleSeedCancer}
\end{figure}

Luego en el caso de la red $\childNetwork$, estas son las 5 variables más relevantes para las distintas métricas propuestas:

\begin{itemize}
    \item \textbf{Mayor valor de Probability Shift}: Disease, Duct Flow, Sick, Cardiac Mixing, LVH
    \item \textbf{Mayor valor de ASV}: Disease, Duct Flow, Sick, LVH, LVH Report
    \item \textbf{Mayor valor de SHAP}: Disease, ChestXRay, CO2, RUQO2, LVH Report
\end{itemize}

Estos resultados\footnote{Los datos completos se pueden encontrar en \path{\pasantia-BICC\results}, para ver la tabla completa para todas las variables.} se basan en la Figura \ref{fig:shapleyVsASVSingleSeedChild}. Lo que podemos ver es que la intersección entre los features más relevantes de Probability Shift y $ASV$, es mayor a la de $SHAP$ con la misma métrica. Ya que aunque ambos logran identificar a los nodos \variableNetwork{Disease} y a \variableNetwork{LVH/LVH Report} como relevantes, sólo $ASV$ encuentra la relación con \variableNetwork{Sick} y \variableNetwork{DuctFlow}. Aún así esto podría variar según el modelo, ya que si el modelo no logró identificar correctamente las relaciones entre los datos, los valores de $ASV$ y $SHAP$ tampoco van a correlacionarse con el grafo causal. Analizar estos valores nos puede ayudar a ver si tiene sentido la elección de features más relevantes que está utilizando nuestro algoritmo para realizar sus predicciones. 
%\echu{Acá lo polémico es que uso la nueva métrica, Probability Shift, cómo un tipo de ground truth para ver cuáles son los features más relevantes}
%\santi{Si, se ve la complicación. Capaz estaría bueno decir algo más o comparar con otras métricas, pero buneo. Creo igual que el primer ejemplo ya convence de que hay algo raro en SHAP.} Rta: Lo dejamos así también, es lo polémico de usar métricas para comparar. Ninguna es mejor que todas

\begin{figure}
    \centering
    \includegraphics[width=1\linewidth]{img/asvResults/childASVAndShapleyExactASVAndShapley.png}
    \caption{Resultados del ASV y Shapley para el modelo con mayor accuracy de los 5 seeds para la red $\childNetwork$, con un 68\% de accuracy. }
    \label{fig:shapleyVsASVSingleSeedChild}
\end{figure}

En base a lo observado en estos dos casos, podemos ver que $ASV$ puede detectar relaciones entre los features que $SHAP$ no logra encontrar. Esto se debe a que utiliza la información del grafo causal, para ver cuáles features priorizar a la hora de realizar estos cálculos. 

\subsection{ASV exacto sin EqClasses vs ASV aproximado}

El objetivo de este experimento es comparar la performance de obtener los órdenes topológicos de un grafo que es un polytree, pero no es un \dtree. Por lo tanto, solo los podemos obtener sampleándolos con nuestro Algoritmo \ref{alg:topoSortSampling} o generándolos con el algoritmo de Knuth, puesto que las clases de equivalencia solo las podemos obtener para los \dtrees.

\begin{table}[h]
\centering
\begin{tabular}{|l|c|}
\hline
\textbf{Cantidad de órdenes sampleados} & \textbf{Tiempo de sampleo (s)}\\
    \hline
    100 & 0.9270 \\
    1000 & 1.1170 \\
    10000 & 4.7170 \\
    20000 & 7.7170 \\
    30000 & 10.8387 \\
    \hline
    \end{tabular}

\caption{Tiempos de ejecución para muestreo y generación de órdenes topológicos de la red $\childNetwork$}
\label{table:exactVsApproximateTopoSorts}
\end{table}

En la Tabla \ref{table:exactVsApproximateTopoSorts} podemos ver que el enfoque aproximado toma un tiempo tratable para samplear los órdenes. Aunque en nuestro análisis de la complejidad del mismo, habíamos llegado a una cota cuasi polinomial, el algoritmo se comporta de mejor manera en la práctica. También podemos observar que, como se mencionó al introducir el algoritmo de sampleo, su complejidad no es lineal en la cantidad de órdenes sampleados. Esto ocurre, pues vamos obteniendo múltiples candidatos en cada llamado recursivo de la función. 

En base a estos resultados, uno podría creer que la mejor opción es generarlos con el algoritmo de Knuth, el cual tarda menos de 1 segundo en generar los 30000 órdenes. El problema de este algoritmo es que no respeta la distribución de los órdenes.  Por ejemplo, si tuviéramos un grafo como el de la Figura \ref{fig:badExactToposortExample}, podría ocurrir que los primeros 1000 órdenes que nos devuelva el algoritmo tengan como primer nodo a $n_1$. Pero en realidad $n_1$ es el primer nodo en menos del 0.05\% de los casos, por ende no sería una muestra representativa. Para lograr esto deberíamos generar todos los órdenes, pero ya generar meramente un 1\% de los órdenes de la red $\childNetwork$ tomaría más de 2 días. 

%\echu{Este ejemplo, no tan detallado, lo mencionó también en el segundo parrafo de la sección de sampleo. ¿Vale la pena este experimento? Para mi esta bueno para ilustrar porque no garpa usar el algoritmo de Knuth, aunque no sume tantooo}

%\santi{Me parece bien.}

\begin{figure}[ht]
    \centering
    \begin{tikzpicture}
        % Define the rigth set of nodes
        \foreach \i in {1,2,3,4,5}
            \node[draw=none, circle, minimum size=5mm, inner sep=0pt] (L\i) at (4, -\i) {};

        % Define the left set of nodes
        \foreach \j in {1,2,3}
            \node[draw, circle, red, minimum size=5mm, inner sep=0pt] (source\j) at (0, -\j*2) {\scalebox{0.6}{$Source_\j$}};

        % Draw edges between nodes (example edges)
        \foreach \i in {1,2}
            \foreach \j in {1,2}
                \draw[->]  (source\j) -- (L\i); 

        \foreach \i in {4,5}
            \foreach \j in {2,3}
                \draw[->]  (source\j) -- (L\i); 

        \draw[->]  (source2) -- (L3);
        

         \draw [decorate, blue, decoration={random steps, segment length=10pt, amplitude=2pt}, thick]
        (4,-3) circle (2.4);

        %Número de ordenes topológicos para cada nodo

        \node[draw=none,minimum size=3mm, inner sep=0pt] () at (0, -1) {\small \textcolor{orange}{$n_1$=1000}};

        \node[draw=none,minimum size=3mm, inner sep=0pt] () at (0, -3) {\small \textcolor{orange}{$n_2$=100000}};

        \node[draw=none,minimum size=3mm, inner sep=0pt] () at (0, -5) {\small \textcolor{orange}{$n_3$=100000}};
    \end{tikzpicture}
    \caption{Posible comienzo del algoritmo exacto, con los candidatos a ser el primer nodo del orden en rojo y el resto del grafo en azul. Cada node fuente (source) tiene sus respectivas cantidades de órdenes topológicos en los que está primero.}
    \label{fig:badExactToposortExample}
\end{figure}

Por último, realizamos un experimento para calcular el error al calcular $ASV$ con el método aproximado, sampleando 1000 órdenes topológicos de la red $\childNetwork$. En la Figura \ref{fig:boxplotASVApproximateDifferences} podemos ver los resultados de esta corrida. La mayoría de los valores aproximados obtenidos tiene un error del 8\% con respecto a su valor exacto. Las diferencias más altas se corresponden a valores de $ASV$ muy pequeños, como 0.005, por lo que una pequeña diferencia en su valor calculado relativamente es más significativa. Esto es esperable, ya que el error mencionado en el Teorema \ref{theorem:asvSamplingError} es un error absoluto, no relativo. Para los features con valores mayores a 0.01, su diferencia es menor al 5\%.
%\santi{Esto es esperable: nuestro teorema habla de error absoluto, no del relativo. Podrías decirlo.}
Con estos resultados, podemos concluir que no es necesario obtener todos los ordenes y con una buena aproximación podemos obtener resultados medianamente precisos. 


\begin{figure}
    \centering
    \includegraphics[width=0.8\linewidth]{img/asvResults/ChildAllSeedsASVBoxplot.png}
    \caption{Diferencia relativa entre los valores del ASV aproximado y el exacto para la red $\childNetwork$, se utilizaron las diferencias de cada una de las 5 seeds.}
    \label{fig:boxplotASVApproximateDifferences}
\end{figure}

% \echu{¿Tiene sentido hacer un experimento en el cuál comparo para distintos grafos, los árboles por ejemplo, cuán cercana es la distribución original a la sampleada? Puedo samplear y ver como me quedan distribuidas las clases de equivalencia. Intentando hacer algún calcúlo para ver cuán cercanas son esas clases de equivalencia a las originales. Respuesta de Santi: Noup, es justamente lo que hace el algoritmo, buscar samplear de las clases de equivalencia}

\subsection{Algoritmo de promedio para DT binarios}
\label{subSection:experimentoAlgoritmoPromedio}
Para este experimento vamos a comparar la forma naive de obtener la predicción promedio para una permutación y el algoritmo \ref{alg:meanPredBinDT}, que utiliza la estructura del árbol para calcularlo con complejidad $O(i|V| + (varElim)l)$. La idea es comparar el tiempo que tardan ambos algoritmos y ver cómo se asemejan estos promedios a las probabilidades originales de la red. Dentro del cálculo de $\assym(x,i)$, para la instancia $x$ y el feature $i$, lo que queremos calcular es:
$$\charactheristicFunction(\toOr) = \mathbb{E}_{\aBayesianNetwork(x' | x)}[f_y(x_{\toOr \leq i} \cup x'_{\toOr > i})]$$

%\echu{La segunda es la fórmula que utilizo en la sección 2 para introducir al promedio, pero me gusta más la que utilizo acá porque siento que se entiende más para explicar la idea. ¿Puedo decir que ambas significan lo mismo y explicar porque? Además a M le falta el $_y$, aunque podría usar $M_y$ en vez de $f_y$ tal vez.}

A través de una permutación $\toOr$ de los features de $x$ definimos qué features quedan fijos y cuáles varían. La función de probabilidad que se utiliza es $\aBayesianNetwork(x' | x)$, la cual utilizamos para calcular la probabilidad de los valores de $x'$ dados los valores de $x$. Esta predicción promedio la vamos a calcular para cada uno de los posibles valores\footnote{Recordemos que podemos tener variables no binarias, por lo que $y$ puede tomar más valores que 0 o 1.} $y$ de $p$, el feature a predecir. $f_y(x)$ es un clasificador binario que devuelve $1$ si nuestro árbol de decisión le asigna la clase $y$ a la instancia $x$ y $0$ en el caso contrario. A continuación presentamos la implementación naive para calcular $\charactheristicFunction$. 
%\santi{El $y$ es el resultado entonces? ¿No alcanza entonces con poner $f_1()?$} Rta: No se había entendido que era para variables no binarias
%\echu{No entendí, $y$ es la clase que queres clasificar. Cómo no necesariamente son binarios los párametros entonces puede tomar todos los valores del dominio de la variable a predecir} 

Para calcular $\mathbb{E}_{\aBayesianNetwork(x' | x)}[f_y(x_{\toOr \leq i} \cup x'_{\toOr > i})]$ lo que vamos a hacer es generar todas las instancias $x_{prom} \in (x_{\toOr \leq i} \cup x'_{\toOr > i})$, en las cuales los valores de los features que aparece luego de $x_i$ en $\toOr$ van a ser variables, y el resto van a ser fijos. Por fijos nos referimos a que van a tener los mismos valores que $x$, y sus otros features van a tomar todos los valores posibles. A partir de estas instancias vamos a calcular la función característica\footnote{La notación para la función característica es distinta a la utilizada al introducirla en la fórmula \ref{formula:characteristicFunctionDefinition}, pero su significado es el mismo. En este caso $(x_{\toOr \leq i} \cup x'_{\toOr > i})$ son nuestras instancias consistentes y $f_y$ es nuestro clasificador binario $M$. Para cada valor de $y$, $f_y$ devuelve 1 si la etiqueta clasificada es $y$ y 0 en el caso contrario.} cómo:

$$\charactheristicFunction(\toOr) = \sum_{x_{prom} \in (x_{\toOr \leq i} \cup x'_{\toOr > i})} p_\aBayesianNetwork(x_{prom} | x) f_y(x_{prom}) $$

%\santi{¿Para cuál valor de $y$?}
%\echu{Para cada valor de $y$ tenes una nueva función, lo aclare en el footnote. }

Por lo tanto, para esta cuenta vamos a necesitar generar todas las instancias $(x_{\toOr \leq i} \cup x'_{\toOr > i})$, y luego evaluar nuestro árbol de decisión $DT$ y a la red bayesiana $\aBayesianNetwork$ para cada una de estas instancias. Evaluar él $DT$ cuesta $O(d)$, siendo $d$ la profundidad del $DT$. Si tomamos a $c$ como la cardinalidad máxima de un feature y a $vars$ como el tamaño del conjunto de features variables, la complejidad temporal de la implementación naive es $O(vars^c(varElim + d)$, siendo $O(vars^c)$ la cantidad de instancias generadas. 

Así las complejidades que nos quedan son $O(vars^c(varElim + d))$ y $O(i|V| + (varElim)l)$, para cada uno de nuestros algoritmos. Podemos ver que la solución naive depende de la cantidad de features variables del $\toOr$, a diferencia del otro algoritmo, que corre el promedio directamente sobre la estructura del $DT$.

%\santi{$v$ no debería ser $|V|$? } Rta: Sip

%\echu{¿Tiene sentido lo de poner el valor que da la predicción del feature? Para mi un poco si para explicar que tipo de feature es y porque da esos valores. } Rta: Sip, tiene sentido

\begin{figure}[ht]
    \centering
    \includegraphics[width=0.7\linewidth]{img/cancerDecisionTree.png}
    \caption{Árbol de decisión generado a partir de los datos de la red $\cancerNetwork$. Las hojas contienen el valor de la predicción que devuelve el modelo (0 o 1)}
    \label{fig:cancerDecisionTree}
\end{figure}

Para la red $\cancerNetwork$ se generaron 600 instancias con un árbol de decisión de altura 3 y para la red $\childNetwork$ se generaron 10000 instancias con un árbol de decisión de altura 9. Las probabilidades de la red bayesiana son los valores que devuelve la consulta $P(X = z)$ para la red $\aBayesianNetwork$ y los distintos valores $z$ de cada feature $X$. El método \emph{Algoritmo promedio (probabilidades)} consiste de la predicción promedio, si en el árbol en vez de devolver una predicción de 1 o 0 se devolviera una probabilidad en las hojas. Este método es distinto al algoritmo evaluado en esta tesis, pero nos pareció interesante agregarlo para analizar la diferencia entre los promedios al usar una predicción probabilística. 
%\santi{Medio feo usar $y$ para dos cosas distintas.} Rta: Concuerdo

\begin{table}[ht]
    \centering
    \begin{tabular}{l c c}
        \toprule
        \textbf{M\'etodo} & \textbf{Predicci\'on} & \textbf{Tiempo (segundos)} \\
        \midrule
        Algoritmo promedio & [0.98255, 0.01745] & 0.0165 \\
        Algoritmo promedio (probabilidades) & [0.7213, 0.2787] & 0.0043 \\
        Implementaci\'on naive & [0.98255, 0.01745] & 0.0043 \\
        Probabilidades red bayesiana & [0.7, 0.3] & - \\
        \bottomrule
    \end{tabular}
    \caption{Valor promedio de la predicci\'on del feature \textbf{Smoker} en la red bayesiana $\cancerNetwork$, dejando variables todos los features}
    \label{table:cancerMeanResults}
\end{table}
%\sergio{Cuidado con la coherencia de CANCER, cancer, \it{cancer}} 

\begin{table}[ht]
    \centering
    \begin{tabular}{l c c}
        \toprule
        \textbf{M\'etodo} & \textbf{Predicci\'on} & \textbf{Tiempo (segundos)} \\
        \midrule
        Algoritmo promedio & [0.9916, 0.0, 0.0084] & 0.0258 \\
        Algoritmo promedio (probabilidades) & [0.7577, 0.0746, 0.1677] & 0.0258 \\
        Implementaci\'on naive & [0.9916, 0.0, 0.0084] & 19.6774 \\
        Probabilidades red bayesiana & [0.6490, 0.1715, 0.1795] & - \\
        \bottomrule
    \end{tabular}
    \caption{Valor promedio de la predicci\'on del feature \textbf{Age} en la red bayesiana $\childNetwork$, dejando variables 11 de los 20 features y utilizando a $x\in data$ t.q $f(x)=0$.}
    \label{table:childMeanResults}
\end{table}

Al analizar la Tabla \ref{table:cancerMeanResults} podemos ver que en ambos casos se tiende a sobrerrepresentar una clase. Esto ocurre ya que el modelo entrenado predice 0 para la mayoría de los inputs y la probabilidad de los inputs para los cuales predice 1 es más baja. En la Tabla \ref{table:childMeanResults} ocurre lo mismo respecto a la sobrerrepresentación. Por lo que, en realidad, la predicción promedio solo va a ser tan buena como el modelo que haya sido entrenado. Esto no depende del algoritmo, sino del entrenamiento del modelo.
Además, podemos ver que la implementación naive es más lenta. Esto se debe a que la mediana de la cardinalidad de cada feature es 3. Por lo que cada feature agregado va a hacer que se tarde 3 veces más en promedio. Para órdenes topológicos que dejaran los 19 features variables la implementación naive tardaría \textbf{más de 1 día}. En cambio, la performance del algoritmo promedio no se ve tan afectada por la cantidad de features variables, sino que depende del tamaño del árbol de decisión. Finalmente, se destaca que las predicciones más cercanas a las probabilidades generadas por la red bayesiana son aquellas que utilizan directamente las probabilidades como output, en lugar de predicciones binarias (0/1). Esto se debe a que estas predicciones son más granulares y, por ende, reflejan con mayor fidelidad las distribuciones probabilísticas subyacentes. Podemos ver en la Figura~\ref{fig:cancerDecisionTree} que el patrón aprendido por el árbol es muy simple $f(x) = $ \textbf{If} $x_{cancer} > 0.5$, \textbf{then} 1, \textbf{else} 0. Luego como $P(Cancer = 1) = 0.01745$, el valor del promedio que vemos en la Tabla \ref{table:cancerMeanResults} va a representar esa predicción. 

Se puede contemplar que los valores de la implementación naive y del algoritmo promedio son idénticos, puesto que ambos están calculando lo mismo. Solo que mientras nuestro algoritmo calcula la probabilidad de llegar a una hoja, la implementación naive genera todas las instancias que pueden llegar a la misma, para luego clasificarlas y calcular su probabilidad. Así que teniendo en cuenta que ambos calculan el mismo valor, podemos concluir que el algoritmo introducido ofrece una mejora significativa respecto a la implementación naive. 
%Esto ocurre ya que el algoritmo del promedio en $DT$ utiliza las hojas del $DT$ para realizar su cálculo, en cambio, la implementación naive genera instancias nuevas a partir de los features variables. Por lo que no necesariamente van a tener el mismo valor estos dos algoritmos, aun así su diferencia no va a ser significativa. Esto puede significar un problema para nuestro algoritmo si resulta que el árbol de decisión no tiene instancias representativas en sus hojas, ya que el valor del promedio no va a tener en cuenta a una muestra significativa. 
%\sergio{Revisar, repensar, repent. No deberia dar distinto, revisar el caso en el cual da y evaluar porque. Si es que no es un bug y tiene sentido, meteer una oraci[on que lo explique mejor. } RTA: Había flasheado, al final si eran iguales
%\santi{Adem[as en la seccion hablas del valor de la prediccion ylos comparas. Cuando en la intro solo hablas del tiempo, aclarar que se va a tener eso en cuenta tambi[en. Si no lo vas a hacer, remover el valor de las tablas.}


%\santi{Mover esta primer expeimentación al final, es la menos importante.}

%\sergio{recordar de unificar capitalización Tablas, Sección, Figura, etc.}

\section{Conclusión}
\begin{frame}{Conclusiones}
	\dificultyLevel{1}
	\begin{itemize}[<+- | alert@+>]
		\item Se optimiz\'o el c\'alculo de ASV en datos con distribuciones bayesianas y árboles de decisión.
		\item Se demostr\'o la tratabilidad para Naive Bayes.
		\item Se desarroll\'o un algoritmo exacto eficiente para la predicci\'on promedio en árboles de decisión.
		\item Se defini\'o una heur\'istica basada en clases de equivalencia para reducir las evaluaciones.
		\item Se construy\'o un algoritmo de sampleo de órdenes topológicos con performance tratable en grafos con grados acotados.
		%\item En la pr\'actica: la heur\'istica reduce llamadas a $\charactheristicFunction$, aunque requiere optimizaci\'on.
	\end{itemize}
\end{frame}

\begin{frame}{Conclusiones}
	\dificultyLevel{1}
	\begin{itemize}[<+- | alert@+>]
		\item Se implement\'o una versi\'on exacta y otra aproximada para ASV.
		\item Se comprobó empiricamente que las clases de equivalencia proporcionan una mejora significativa. 
		\item El principal aporte es la optimizaci\'on de ASV mediante clases de equivalencia respecto de los órdenes topológicos.
	\end{itemize}
\end{frame}

\begin{frame}{Trabajo Futuro}
	\dificultyLevel{1}
	\begin{itemize}[<+- | alert@+>]
		\item Generalizar algoritmo de clases de equivalencia a \emph{polytrees}.
		%\item Optimizar el algoritmo de conteo de órdenes topológicos.
		\item Implementar nuevas estrategias de sampleo y conteo. % \cite{HUBER2006420, efficientCountingOfToposorts}.
		\item Extender la implementación de ASV para modelos y distribuciones arbitrarios.
		\item Estudiar propiedades de complejidad del sampleo y conteo de órdenes tópologicos.
		\item Explorar algoritmos alternativos para enumerar órdenes tópologicos.
		
	\end{itemize}
\end{frame}


\begin{frame}[plain]  % plain quita cabecera/pie
	\begin{tikzpicture}[remember picture,overlay]
		% nodo con la imagen de fondo
		\node at (current page.center) {
			\includegraphics[width=\paperwidth,height=\paperheight]{pic/img/fotitoAgradecimientos.jpg}
		};
    \node at (current page.center) {
	{\fontsize{80}{90}\selectfont\color{blue}\bfseries Gracias!}
		};
	\end{tikzpicture}
\end{frame}

\section{Extra}





%Apéndice con cosas que no entraron por tiempo pero ya había hecho 

\begin{frame}[noframenumbering]{Algoritmo de promedio}
	\dificultyLevel{3}
	\begin{enumerate}[<+- | alert@+>]
		\item Recorrer todas las ramas del árbol de decisión, acumulando las decisiones tomadas.
		\item Al llegar a una hoja:
		\begin{itemize}
			\item Evaluar la probabilidad de haber alcanzado esa hoja, dada la evidencia. 
			%\[ Pr_B\bigl(pathCondition \mid ev\bigr).\]
			\item Luego multiplicar dicha probabilidad por el valor de salida que retorna la hoja.
		\end{itemize}
		\item Sumar todas las contribuciones de cada hoja para obtener la predicción promedio. 
	\end{enumerate}
\end{frame}


\begin{frame}[noframenumbering,fragile]{Algoritmo: Predicción Promedio}
	\dificultyLevel{3}
	\begin{algorithm}[H] % ← evita que flote
		\caption{Predicción promedio para árbol de decisión binario}
		\footnotesize            % opcional: letra más chica
		\begin{algorithmic}[1]
			\Function{Mean}{$node$, $B$, $pathCondition$, $evidence$}
			\If{$evidence$ no coincide con $pathCondition$}
			\State \Return $0$
			\EndIf
			\If{$node$.isLeaf}
			\State \Return $Pr_B(pathCondition \mid evidence)\cdot node.value$
			\EndIf
			\State $X_i \gets node.feature$
			\State $left \gets$  \Call{Mean}{$node$.left,  $B$, $pathCondition \cup\{X_i=0\}, evidence$}
			\State $right\gets$  \Call{Mean}{$node$.right, $B$, $pathCondition \cup\{X_i=1\}, evidence$}
			\State \Return $left + right$
			\EndFunction
		\end{algorithmic}
	\end{algorithm}
	\vspace{0.3cm}
	Complejidad: $O(i|V| + l \cdot varElim)$, polinomial si $varElim$ lo es (e.g. polytrees).
\end{frame}


\begin{frame}[noframenumbering]{Fórmula Shapley}
	\dificultyLevel{3}
	\textbf{Fórmula general:}
	
	% Parte visual progresiva de la fórmula
	\only<1,2>{%
		\begin{mydefinition}[Shapley Value]
			\[
			\phi_i(v) = \text{Shapley Value del jugador $i$ para la función $v$}
			\]
		\end{mydefinition}
	}
	\only<3>{%
		\begin{mydefinition}[Shapley Value]
			\[
			\phi_i(v) = \alert<3>{\frac{1}{|X|!} \sum_{S \subseteq X \setminus \{i\}}} \cdots
			\]
		\end{mydefinition}
	}
	\only<4>{%
		\begin{mydefinition}[Shapley Value]
			\[
			\phi_i(v) = \frac{1}{|X|!} \sum_{S \subseteq X \setminus \{i\}} \alert<4>{|S|! (|X|-|S|-1)!} \cdot \cdots
			\]
		\end{mydefinition}
	}
	\only<5,6>{%
		\begin{mydefinition}[Shapley Value]
			\[
			\phi_i(v) = \frac{1}{|X|!} \sum_{S \subseteq X \setminus \{i\}} |S|! (|X|-|S|-1)! \cdot \alert<5>{(v(S \cup \{i\}) - v(S))}
			\]
		\end{mydefinition}
	}
	
	\only<2>{
		\begin{mydefinition}
			La función característica se define como:
			\[
			v : \mathcal{P}(X) \to \mathbb{R}
			\]
			Asigna un valor real a cada posible \textit{coalición} de jugadores, es decir, a cada subconjunto de \( X \).
		\end{mydefinition}
	}
	
	% Lista explicativa
	\begin{itemize}
		\item<3-> \alert<3>{Se suman todos los subconjuntos $S$ que no contienen a $i$, para ver cuánto colabora $i$ a cada uno.}
		\item<4-> \alert<4>{El término $|S|!(|X|-|S|-1)!$ cuenta cuántas veces \( i \) puede llegar justo después de \( S \) en un orden.}
		\item<5-> \alert<5>{Se calcula el aporte marginal de $i$ a $S$: $v(S \cup \{i\}) - v(S)$.}
		\item<6-> \alert<6>{Se divide todo por $|X|!$, porque se está promediando sobre todas las permutaciones posibles.}
	\end{itemize}
\end{frame}

\begin{frame}[noframenumbering]{Fórmula función característica en ML}
	\dificultyLevel{3}
	% Parte visual progresiva de la fórmula
	\only<1>{%
		\begin{mydefinition}[Función característica]
			\scriptsize
			\[
			v_{M,e,\Pr}(S) = \text{Predicción promedio de $M$ cuando los features $S$ toman los valores de $e$}
			\]
		\end{mydefinition}
	}
	\only<2>{%
		\begin{mydefinition}[Función característica]
			\[
			v_{M,e,\Pr}(S) = \alert<2>{\sum_{e' \in \consistsWith(e,S)}} \cdots
			\]
		\end{mydefinition}
	}
	\only<3>{%
		\begin{mydefinition}[Función característica]
			\[
			v_{M,e,\Pr}(S) = \sum_{e' \in \consistsWith(e,S)} \alert<3>{\Pr[e'|\consistsWith(e,S)]} \cdot \cdots
			\]
		\end{mydefinition}
	}
	\only<4,5>{%
		\begin{mydefinition}[Función característica]
			\[
			v_{M,e,\Pr}(S) = \sum_{e' \in \consistsWith(e,S)} \Pr[e'|\consistsWith(e,S)] \cdot \alert<4>{M(e')}
			\]
		\end{mydefinition}
	}
	
	% Lista explicativa
	\begin{itemize}
		\item<2-> \alert<2>{Se consideran las instancias $e'$ que coinciden con la entidad $e$ en los atributos de $S$: $\consistsWith(e,S)$.}
		\item<3-> \alert<3>{Se pondera cada $e'$ según su probabilidad condicional dado que coincide con $e$ en $S$: $\Pr[e'|\consistsWith(e,S)]$.}
		\item<4-> \alert<4>{Se evalúa el modelo $M$ sobre cada $e'$.}
		\item<5-> \alert<5>{En resumen: $v(S)$ es la predicción promedio del modelo dejando fijos los features de $S$.}
	\end{itemize}
\end{frame}


\begin{frame}[noframenumbering]{Extensión a Features No Binarios}
	\dificultyLevel{2}
	\begin{itemize}[<+- | alert@+>]
		\item El algoritmo original funciona con árboles y variables \textbf{binarios}.
		\item Para admitir \textbf{features no binarios} adaptamos la inferencia.
		\item Al llegar a un nodo con umbral \(v\), dividimos el dominio de \(f\):
		\begin{itemize}
			\item Lado izquierdo: \(f = i\) con \(i < v\)
			\item Lado derecho: \(f = d\) con \(d \geq v\)
		\end{itemize}
		\item La probabilidad condicional requiere una \textbf{suma de múltiples consultas}. 
		\begin{itemize}
			\item  Si tuviéramos la consulta \set{$X_1 \in \set{1,2}, X_2 = 3$} la resolvemos cómo:
			\[
			Pr_B(X_1=1, X_2=3) + Pr_B(X_1=2, X_2=3)
			\]
		\end{itemize}
		\item Esto hace que la complejidad ya no sea polinomial.
		\item No optimizamos esta inferencia, ya que excede los objetivos de la tesis.
	\end{itemize}
\end{frame}


\begin{frame}[noframenumbering]{Conteo de Órdenes en \dtrees}
	\dificultyLevel{3}
	\textbf{Fórmula general:}
	
	\only<1>{
		\begin{mydefinition}
			Sean \(k_i\) la cantidad de nodos del subárbol \(t_i\), con 
			\(n = \sum_{i=1}^r k_i\). La cantidad de órdenes topológicos es:
		\end{mydefinition}
	}
	
	\only<2>{
		\begin{mydefinition}
			Sean \(k_i\) la cantidad de nodos del subárbol \(t_i\), con 
			\(n = \sum_{i=1}^r k_i\). La cantidad de órdenes topológicos es:
			\[
			\numTopo(t) 
			= \alert<2>{\binom{n}{k_1,\dots,k_r}}
			\]
		\end{mydefinition}
	}
	
	\only<3->{
		\begin{mydefinition}[Órdenes Topológicos en un \dtree]
			Sean \(k_i\) la cantidad de nodos del subárbol \(t_i\), con 
			\(n = \sum_{i=1}^r k_i\). La cantidad de órdenes topológicos es:
			\[
			\numTopo(t) 
			= \binom{n}{k_1,\dots,k_r}
			\;\cdot\;
			\alert<3>{\prod_{i=1}^{r} \numTopo(t_i)}
			\]
		\end{mydefinition}
	}
	
	\begin{itemize}
		\item<2-> \alert<2>{Coeficiente multinomial: 
			\(\binom{n}{k_1,\dots,k_r} = \frac{n!}{k_1!\cdots k_r!}\)} 
		cuenta las maneras de intercalar nodos de subárboles sin alterar su orden interno.
		\item<3-> \alert<3>{Producto de subárboles: 
			\(\prod_{i=1}^{r} \numTopo(t_i)\)} 
		corresponde a las combinaciones posibles dentro de cada subárbol.
		\item<4-> \alert<4>{Combinación final: la fórmula multiplica ambas partes para obtener el total de órdenes topológicos.}
	\end{itemize}
\end{frame}


\begin{frame}[noframenumbering]{Fórmula de \eqClassSizes}
	\dificultyLevel{3}
	\only<1>{%
		Habiendo realizado estos cálculo estamos listos para definir nuestra fórmula para el \textbf{conjunto de clases de equivalencia y sus tamaños}.
		\[
		\eqClassSizes(G,x_i) = \,\cdots
		\]
		
		%\textbf{Definición:} 		\begin{itemize}			\item $UR$: raíces de subárboles \alert{no relacionados} con $x_i$.		\end{itemize}
	}
	\only<2>{%
		\[
		\eqClassSizes(G,x_i)
		= \bigcup_{\displaystyle mix\in
			\prod_{j=1}^{|UR|}\unrEqCl(ur_j)} \,\cdots
		\]
		\vspace{1em}
		\textbf{Nota:}
		\begin{itemize}
			\item Cada $mix$ es una combinación (producto cartesiano) de 
			las clases de cada $ur_j\in UR$.
		\end{itemize}
	}
	\only<3>{%
		\[
		\eqClassSizes(G,x_i)
		= \bigcup_{mix}
		\Bigl(\,eqCl(A,D,mix),\,\eqClassSize(A,D,mix)\Bigr)
		\]
		\vspace{1em}
		\textbf{¿Qué hace?}
		\begin{itemize}
			\item $eqCl(A,D,mix)$ fusiona \alert{ancestros $A$}, 
			\alert{descendientes $D$} y la combinación $mix$.
		\end{itemize}
	}
	\only<4->{%
		\[
		\eqClassSizes(G,x_i)
		= \hspace{-3em} \bigcup_{mix\in\prod_{j=1}^{|UR|}\unrEqCl(ur_j)} \hspace{-3em}
		\Bigl(eqCl(A,D,mix),\,\eqClassSize(A,D,mix)\Bigr)
		\]
		\vspace{1em}
		\textbf{Componentes finales:}
		\begin{itemize}
			\item<4-> $eqCl(A,D,mix)$: representa la clase resultante tras fusionar.
			\item<5-> $\eqClassSize(A,D,mix)$: cantidad de órdenes topológicos de esa clase.
		\end{itemize}
	}
	
\end{frame}

% ------------------------------------------------------------------------
\begin{frame}[noframenumbering][fragile]{Algoritmo \texttt{leftOrders}}
	\dificultyLevel{4}
	\begin{algorithm}[H]
		\caption*{leftOrders($A$, $\textit{actual ancestor}$, $\textit{nodes to place}$, $position$)} \label{alg:leftOrdersAlgorithm}
		\begin{enumerate}
			\item Definimos donde colocar $\textit{actual ancestor}$ en base a $position$ y a cuántos nodos tenemos disponibles en $\textit{nodes to place}$, generando $\textit{new position}$.
			\item Luego seleccionamos cuántos nodos de cada unrelated tree vamos a usar para llenar todas las posiciones entre $position$ y $\textit{new position}$, generando $\textit{new nodes}$.
			\item Eliminamos los $\textit{new nodes}$ de los $\textit{nodes to place}$, puesto que ya los colocamos, actualizando nuestros nodos disponibles.
			\item Realizamos el llamado recursivo actualizando la posición, nuestros nodos disponibles y nuestro ancestro actual. 
		\end{enumerate}
	\end{algorithm}
\end{frame}

% ------------------------------------------------------------------------
\begin{frame}[noframenumbering]{Intuición de \texttt{leftOrders}}
\dificultyLevel{3}
\begin{itemize}[<+- | alert@+>]
	\item Recorre los ancestros en orden.
	\item En cada paso reparte los nodos no relacionados en los “huecos” antes del ancestro:
	\[
	[\,\underbrace{\;\; }_{a_0}\;|\;\underbrace{\;\; }_{a_1}\;|\;\dots\;|\;\underbrace{\;\; }_{a_{|A|}}\;]
	\]
	%\item Al recorrer recursivamente, se reutilizan combinaciones parciales (DP). Puesto que podemos llegar a la misma configuración por varios caminos.
\end{itemize}
\end{frame}

	% ------------------------------------------------------------------------
\begin{frame}[noframenumbering]{Iteración: Nodos Disponibles}
	\dificultyLevel{3}
	\begin{figure}[H]
		\centering
		\begin{tikzpicture}[scale=.45, transform shape, 
			unrelated/.style={circle, draw=red},
			ancestor/.style={circle, draw=blue},
			wiggly/.style={decorate, decoration={snake, amplitude=.2mm, segment length=2mm}}  % Define wiggly line style
			]
			
			\node[draw=none, fill=none] (a1) at (0, 0) {};
			\node[ancestor] (a2) at (1, -2) {$a_{i-1}$};
			
			\drawUnrelatedTreeWithTag{u2}{-1}{-4}{$u_{i-1}$}{orange}{Available nodes: npa[i-1]}
			
			\drawUnrelatedTree{u4}{0}{-7}{$u_{i}$}
			\node[ancestor] (a3) at (3, -6) {$a_i$};
			
			\node[draw=none, fill=none] (xi) at (3, -8) {};
			
			\drawUnrelatedTreeWithTag{r1}{6}{0}{$u_5$}{orange}{Available nodes: npa[0]};
			\drawUnrelatedTreeWithColor{r2}{10}{0}{$u_6$}{orange};
			
			
			\path [->] (a1) edge[arista,  decorate, decoration={snake, amplitude=.4mm, segment length=4mm, post length=1mm}] (a2);
			
			\path [->] (a2) edge[arista]  (u2);
			\path [->] (a2) edge[arista]  (a3);
			
			\path [->] (a3) edge[arista]  (u4);
			\path [->] (a3) edge[arista,  decorate, decoration={snake, amplitude=.4mm, segment length=4mm, post length=1mm}] (xi);
		\end{tikzpicture}
		%\caption*{Los nodos pintados en naranja son los nodos disponibles para ser colocados en el paso $i$. Para cada conjunto de subárboles $npa$ (nodes to place) tiene la cantidad de nodos disponibles.}
		\label{fig:leftOrdersIterationGraph}
	\end{figure}
	\begin{itemize}
		\item Solo los nodos de $u_{i-1}$ pueden rellenar el hueco antes de $a_i$.
		%\item Tras fijar $a_i$, ampliamos el conjunto de nodos disponibles.
	\end{itemize}
\end{frame}

\begin{frame}[noframenumbering]{Sampleo Toposorts}
	\begin{algorithm}[H]
		\caption{SampleoTopoSort($D$)} \label{alg:topoSortSampling}
		\begin{enumerate}
			\item \textbf{Calculamos una probabilidad} $p$ para cada uno de los nodos fuente del DAG.
			\begin{enumerate}
				\item Para cada $s \in S$ lo removemos del DAG $D$, y contamos la cantidad de órdenes topológicos en $D-\set{s}$ ($toposorts_s$), este valor es la cantidad de órdenes que comienzan con $s$.
				\item Luego a cada $s \in S$ le asignamos una probabilidad $p(s)= \frac{toposorts_s}{\#topos(D)}$. 
			\end{enumerate}
			\item \textbf{Sampleamos} sobre $S$ utilizando $p$ para obtener nuestro primer nodo $start$.
			\item \textbf{Eliminamos} a $start$ de $D$ y llamamos al algoritmo recursivamente con  SampleoTopoSort($D-\set{start}$), guardando el resultado en $orden$. 
			\item \textbf{Devolvemos} $start + orden$ como el orden topológico sampleado. 
		\end{enumerate}
	\end{algorithm}
\end{frame}



%\begin{frame}[allowframebreaks]{Referencias}   
%\bibliography{biblio}
%\end{frame}

\end{document}

