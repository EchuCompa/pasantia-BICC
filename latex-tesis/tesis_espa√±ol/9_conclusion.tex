En el presente trabajo se optimizó el cálculo de las explicaciones brindadas a través de Assymetric Shapley Values (ASV), aplicadas a datos con distribuciones de redes bayesianas y modelos de árboles de decisión. Se demostró la tratabilidad del problema para el caso de la Naive Bayes y se buscó una solución polinomial para el caso más general. 

Para el caso más general, se definió una noción de clase de equivalencia en base a los órdenes topológicos, para disminuir la cantidad de evaluaciones de la predicción promedio. Para la predicción promedio en árboles de decisión, se implementó un algoritmo exacto más eficiente que la implementación naive. En el análisis teórico se identificó que el algoritmo diseñado para obtener las clases de equivalencia en árboles, tiene una complejidad temporal polinomial respecto a la cantidad de clases de equivalencia. 

Luego, se realizó un algoritmo aproximado para calcular ASV. Para el mismo, se llevó a cabo el cómputo del número de órdenes topológicos, el cual, en el caso general, se encuentra fuera de la clase de problemas polinomiales (\#P-completo). No obstante, se mostró que para los polyforest con grado acotado (algo razonable en las redes bayesianas), es posible desarrollar algoritmos que logran una complejidad tratable. 

Desde el punto de vista práctico, se realizó la implementación de ASV utilizando las ideas desarrolladas en esta tesis. Se pudo observar que la cantidad de clases de equivalencia era  menor que la cantidad de órdenes y que la heurística presentaba una mejora significativa. Aún así, dicha implementación resulta considerablemente más lenta en comparación con SHAP. Se debe optimizar la implementación de ASV, para lograr un desempeño que sea competitivo. La optimización podría orientarse hacia el uso de lenguajes más eficientes (cómo $C++$), técnicas de paralelización o encontrar nuevas heurísticas para calcular ASV. En lo que respecta al enfoque aproximado, para los grafos utilizados en la experimentación el problema era tratable y la aproximación tenía una precisión adecuada. Por último, al contrastarse los valores de SHAP y ASV, se encontró que ASV lograba identificar relaciones entre los datos que SHAP no contemplaba. 

En síntesis, se obtuvo una implementación exacta de ASV para distintos modelos y distribuciones, con la posibilidad de expandir el framework a una mayor cantidad de familias. Además de un algoritmo de conteo y sampleo para órdenes topológicos en polytrees, el cual permite aproximar el valor de ASV. El aporte principal del trabajo es optimizar esta métrica, a través de la heurística encontrada que utiliza la noción de clases de equivalencia aplicada a los órdenes topológicos. 


\paragraph{Trabajo futuro}

A continuación se detallan algunas líneas de trabajo futuro que surgieron a lo largo del desarrollo de esta tesis:

\begin{itemize}
  \item \textbf{Generalizar el algoritmo de clases de equivalencia a polytrees:} adaptar el algoritmo actual, que solo admite árboles dirigidos, para que también funcione sobre estructuras más generales como \emph{polytrees}, utilizando un enfoque similar al del conteo exacto de órdenes topológicos en estos grafos.

  \item \textbf{Optimizar el algoritmo de conteo:} investigar mejoras que permitan reducir la complejidad del algoritmo de conteo de órdenes topológicos, buscando una solución polinomial en el tamaño del grafo, independientemente del grado de sus vértices.

  \item \textbf{Extender el framework a modelos causales arbitrarios:} permitir el uso de modelos causales que no necesariamente sean redes bayesianas. Para ello, modificar el código para que pueda operar sobre modelos más generales, así como optimizar la ejecución mediante cacheo de resultados intermedios en las predicciones promedio.

  %\item \textbf{Comparar con Monte Carlo:} comparar la técnica de sampleo de órdenes tópologicos con las clases de equivalencia  vs samplear utilizando Monte Carlo y calcular el ASV con esas permutaciones, para ver que enfoque es más eficiente.  Rta: Al final es medio humo, porque no es claro que es samplear órdenes con Monte Carlo. Así que lo sacamos.

  \item \textbf{Incorporar consultas de unión sobre evidencia parcial:} estudiar una implementación eficiente de la operación de unión de consultas en redes bayesianas mediante \emph{Variable Elimination}, aprovechando la posibilidad de reutilizar eliminaciones intermedias. Esta operación resulta útil al introducir evidencia parcial sobre variables y la realizamos múltiples veces en el algoritmo de predicción promedio.
  En \href{https://rdrr.io/github/vspinu/bnlearn/man/cpquery.html#:~:text=When%20,be%20treated%20as%20a%20list}{bnlearn} se implementa esta consulta, pero esta implementado cómo la suma. 

  \item \textbf{Estudiar propiedades de complejidad de órdenes topológicos:} formalizar y demostrar si existe una equivalencia entre la posibilidad de contar órdenes topológicos en tiempo polinomial y la de muestrearlos eficientemente en dicho tiempo.

  \item \textbf{Explorar algoritmos alternativos para enumerar órdenes topológicos:} implementar y adaptar ideas provenientes de trabajos previos, cómo \cite{efficientToposort}, con el objetivo de acelerar el muestreo de órdenes topológicos en nuestro enfoque.

  \item \textbf{Implementar nuevas estrategias de sampleo:} durante la investigación de trabajos similares realizados encontramos un algoritmo de sampleo \cite{HUBER2006420} y un algoritmo de conteo \cite{efficientCountingOfToposorts}, para ordenes topológicos. Al implementar estos algoritmos podríamos utilizarlos mejorar la complejidad de nuestra solución. 
\end{itemize}

%\santi{Comentario de Sergio: que las citas estén bien, mayúsculas en palabras como Boolean o SHAP.}
